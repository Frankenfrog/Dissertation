%!TEX root = ../main.tex

\chapter{Introduction}
\label{sec:Introduction}

From latest measurements, based on the Lambda cold dark matter ($\Lambda$CDM)
model, which is a parametrisation of the Big Bang cosmological
model~\cite{BigBang}, the age of our universe is calculated to be \num{13.8}
billion years~\cite{Ade:2015xua}. In principle, theoretical models exist that
describe the evolution of the universe during this incredibly long period of
time, starting from directly after the Big Bang until today. But although
there are models describing phenomena ranging from the largest (cosmological
models) down to the smallest scales (Standard Model of particle physics (SM)),
high energy physics is far from being fully understood. One of the most
striking observations is that we are surrounded by matter, while there are no
antimatter clusters. Though, according to big bang theories the origin of the
universe has been energy, which creates the same amount of matter and
antimatter. Sakharov has proposed three conditions~\cite{Sakharov:1967dj}
that need to be fulfilled to explain this so called baryogenesis. The thermal
equilibrium has to be imbalanced, the baryon number is required to be
violated, \ie the proton, whose lifetime is determined to be greater than
\SI{5.9e33}{years}~\cite{Abe:2014mwa}, has to be unstable and decay, and the
$C$ (and even \CP) symmetry has to be violated as well. Indeed, $C$-violating
and shortly thereafter \CP-violating processes have already been discovered
fifty years ago~\cite{CPV_discovery}. After so many years this topic is still
of great interest. On the one hand, the size of \CP violation in the SM is
orders of magnitude below what is required to explain the matter-antimatter
asymmetry~\cite{Huet:1994jb}. On the other hand, \CP-violating processes are
an excellent test bed for the predictions in the quark-flavour sector of the
Standard Model of particle physics. The unitarity of the
Cabibbo-Kobayashi-Maskawa (CKM) matrix, which describes the probability of
quark transitions, is a fundamental requisite of the SM. It can be tested by
studying the unitarity triangle, which represents one of the unitarity
conditions. The determination of the angle $\beta$ of this triangle, or more
precisely of the derived quantity \sintwobeta, is the common theme of this
thesis. Precision measurements of \CP violation in charmonium decays, \ie in
decay modes involving a \ccbar resonance, and in open-charm decays, \ie in
decay modes with at least one hadron containing exactly one \cquark quark, are
performed. More specifically, \CP violation is studied in \BdToJPsiKS and in
\BdToDD decays. The two analyses exploit the full Run~I data sample, which is
the world's largest sample of \Bz mesons. It corresponds to an integrated
luminosity of \SI{3}{\invfb} collected with the \lhcb detector in
proton-proton collisions at centre-of-mass energies of \SIlist{7;8}{\TeV}.
While the decay mode \BdToJPsiKS offers a very clean determination of
\sintwobeta at a very high precision, the main purpose of the study of \BdToDD
decays is to constrain higher-order Standard Model corrections occurring in
measurements of the \CP-violating phase. These contributions need to be
controlled to distinguish them from effects caused by physics beyond the
Standard Model of particle physics, often referred to as "New Physics". There
are several reasons to believe that the SM needs to be extended. The
measurements of rotation curves of galaxies~\cite{AndromedaNebula} have lead
to the assumption of the presence of dark matter. This is not a small effect.
Around \SI{26}{\percent} of the energy density in the universe are assigned to
dark matter compared to around \SI{5}{\percent} for normal baryonic
matter~\cite{Ade:2015xua}. However, dark matter is not accounted for in the
SM, though no dark matter candidate has been found so far. Furthermore, in the
SM the neutrinos are set to be massless, which is disproved by the observation
of neutrino oscillations~\cite{Fukuda:1998mi,Ahmad:2001an,*Ahmad:2002jz},
awarded with the Nobel prize in 2015~\cite{NobelPrize2015}.

The thesis is structured as follows: First, the basics of the
Standard Model of particle physics are shortly introduced (see
\cref{sec:standardmodel}). In \cref{sec:cpviolation} a more detailed
description of the origin and nature of \CP violation is given as well as ways
to measure it. The \lhcb experiment, namely the detector and the associated
software, is described in \cref{sec:detector}. Some relevant techniques
applied in data analysis of high energy physics are presented in
\cref{sec:dataanalysis}. After these prerequisites are introduced, the analysis
strategies of the measurements of \CP violation in \BdToJPsiKS decays (see
\cref{sec:bd2jpsiks}) and in \BdToDD decays (see \cref{sec:b02dd}) are
discussed. The results of the two analyses are compared with previous
measurements and with each other in \cref{sec:discussion} and a summary of the
outcome of this thesis is given in \cref{sec:conclusion}.
