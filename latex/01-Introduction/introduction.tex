%!TEX root = ../main.tex

\chapter{Introduction}
\label{sec:Introduction}

From latest measurements, based on the Lambda cold dark matter ($\Lambda$CDM)
model, which is a parametrisation of the Big Bang cosmological
model, \todo{references for big bang or LCDM} the age of our universe is
calculated to be \num{13.8} billion years~\cite{Ade:2015xua}. In principle,
theoretical models exist that describe the evolution of the universe during
this incredibly long period of time, starting from directly after the Big Bang
until today. But although there are models describing phenomena ranging from
the largest (cosmological models) down to the smallest scales (Standard Model
of particle physics (SM)), high energy physics is far from being fully
understood. One of the most striking observations is that we are surrounded by
matter, while there are no anti-matter clusters. Though, according to big bang
theories the origin of the universe has been energy, which creates the same
amount of matter and anti-matter. Sakharov has proposed three
conditions~\cite{Sakharov:1967dj} that need to be fulfilled to explain this so
called baryogenesis. The thermal equilibrium has to be imbalanced, the baryon
number is required to be violated, \ie the proton, whose lifetime is
determined to be greater than \SI{5.9e33}{years}~\cite{Abe:2014mwa}, has to be
unstable and decay, and the $C$ (and even \CP) symmetry has to be violated as
well. Indeed, $C$-violating and shortly thereafter \CP-violating processes
have already been discovered fifty years ago~\cite{CPV_discovery}. In this
thesis the main subject is the (precise) measurement of \CP violation in
charmonium decays, \ie in decay modes involving a \ccbar resonance, and in
open-charm decays, \ie in decay modes with at least one hadron containing
exactly one \cquark quark. More specifically, \CP violation is studied in
\BdToJPsiKS and in \BdToDD decays. After so many years this topic is still of
great interest. On the one hand, the amount of \CP violation is measured to be
orders of magnitude below what is required to fulfil Sakharov's
conditions\todo{reference}. On the other hand, \CP-violating processes are an
excellent test bed for the predictions in the quark-flavour sector of the
Standard Model of particle physics. The unitarity of the
Cabibbo-Kobayashi-Maskawa (CKM) matrix, which describes the probability of
quark transitions, is a fundamental requisite of the SM. It can be tested by
studying the unitarity triangle, which represents one of the unitarity
conditions. The determination of the angle $\beta$ of this triangle, or more
precisely of the derived quantity \sintwobeta, is the common theme of this
thesis. The world's largest sample of \Bz mesons is exploited for measurements
of \sintwobeta. The two analyses, presented in this thesis, exploit the full
Run~I data sample corresponding to an integrated luminosity of \SI{3}{\invfb}
collected with the \lhcb detector in proton-proton collisions at
centre-of-mass energies of \SIlist{7;8}{\TeV}. The \lhcb experiment is
specifically designed to study \CP violation and rare decays of hadrons
containing $\bquark$ and $\cquark$ quarks.

Before discussing the analysis strategies of the measurements of \CP violation
in \BdToJPsiKS decays (see \cref{sec:bd2jpsiks}) and in \BdToDD decays (see
\cref{sec:b02dd}), some prerequisites are introduced. First, the basics of the
Standard Model of particle physics are shortly introduced (see
\cref{sec:standardmodel}). In \cref{sec:cpviolation} a more detailed
description of the origin and nature of \CP violation is given as well as ways
to measure it. The \lhcb experiment, namely the detector and the associated
software, is described in \cref{sec:detector}. Some relevant techniques
applied in data analysis of high energy physics are presented in
\cref{sec:dataanalysis}. The results of the two analyses are compared with
previous measurements and with each other in \cref{sec:discussion} and a
summary of the outcome of this thesis is given in \cref{sec:conclusion}.
