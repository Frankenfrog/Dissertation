%!TEX root = ../main.tex

\section{Studies of systematic effects}
\label{sec:bd2jpsiks:systematics}

To check if and how various effects systematically influence the measurement
of the \CP violation parameters, the likelihood fit is performed
\begin{itemize}
    \item with a second independent fitter~\cite{Cauet-PhDThesis},
    \item without a parametrisation of the background using sWeights extracted
    from the \sPlot technique,
    \item  on subsamples split by the \KS track type, the trigger
    requirements, the tagging algorithms, the magnet polarity, and the year of
    data-taking.
\end{itemize}
All results show good agreement with the nominal results. Additionally, the
results from a pure time-dependent and from a pure time-integrated fit are
compatible with each other and with the nominal fit, which comprises both
effects.

Systematic uncertainties from several effects, especially from possible
mismodelling of PDFs and from systematic uncertainties on external input
parameters, are considered. Pseudoexperiments are generated using PDFs that
contain a slight modification compared to the nominal PDF, which is used for
the subsequent fit of the samples. Whenever the mean of the pull distribution
exceeds zero by more than \num{0.032}, a systematic uncertainty is assigned.
Pull distributions show the difference between the individual fit result and
the generation value for the \CP observable normalised by the fit uncertainty.
The value of the criterion is defined by one standard deviation according to
the statistics of \num{1000} pseudoexperiments performed for each study. The
size of the systematic uncertainty is taken from the residual distributions,
which show the same as the pull distributions apart from not being normalised.

The possible tagging asymmetry of the background contribution is treated as a
source for a systematic uncertainty on the \CP observables using the asymmetry
parameters determined from an sweighted fit to the background decay time
distribution (see last part of \cref{sec:bd2jpsiks:backgrounds}). It is found
to account for \SI{83}{\percent} of the systematic uncertainty on \SJpsiKS and
\SI{8}{\percent} for \CJpsiKS. The systematic uncertainties on the
flavour-tagging calibration parameters (see
\cref{sec:dataanalysis:taggingcalibration:jpsixcalibration}) are transferred
to systematic uncertainties on \SJpsiKS and \CJpsiKS by shifting all
calibration parameters related to $p_0$ upwards by one systematic uncertainty
and those related to $p_1$ downwards in the generation of the
pseudoexperiments while setting them to their nominal values in the fit. This
yields \SI{9}{\percent} of the systematic uncertainty on \SJpsiKS, and
\SI{21}{\percent} for \CJpsiKS. The assumption $\DGd = 0$ is responsible for
\SI{6}{\percent} of the systematic uncertainty on \SJpsiKS. It is determined
generating the pseudoexperiments setting $\DGd = \SI{0.007}{\invps}$, the
current experimental uncertainty~\cite{PDG2014}. The largest contribution
(\SI{42}{\percent}) to the systematic uncertainty on \CJpsiKS arises from the
systematic uncertainty on the world average of \dmd. Further effects, which
have been analysed concerning causing systematic uncertainties, are the decay
time resolution model, the uncertainty on the length scale of the vertex
detector, the decay time acceptance model, the correlation between the
invariant mass and the decay time, and the production asymmetry. They are all
small or even negligible compared to the previously described effects.

The values of all individual systematic uncertainties and of the total
systematic uncertainty, which is calculated as the sum of all contributions in
quadrature, are listed in \cref{tab:bd2jpsiks:systematics}. The systematic
uncertainty on \SJpsiKS of \num{\pm0.020} is more than \SI{40}{\percent}
smaller than the statistical uncertainty. For \CJpsiKS the total uncertainty
even increases by only \SI{1}{\percent} through the systematic uncertainty.

\begin{table}[htb]
  \caption{Systematic uncertainties on \SJpsiKS and \CJpsiKS. Entries marked
  with a dash represent studies where no significant effect is observed.}
  \label{tab:bd2jpsiks:systematics}
  \centering
    \begin{tabular}{lSS}
      \toprule
      Origin & {\param{\sigma}{}{$\SJpsiKS$}} & {\param{\sigma}{}{$\CJpsiKS$}} \\
      \midrule
      Background tagging asymmetry            &  0.018  &  0.0015 \\
      Tagging calibration                     &  0.006  &  0.0024 \\
      $\DGd$                                  &  0.005  &  {---} \\
      Fraction of wrong PV component          &  0.0021 &  0.0011 \\
      $z$-scale                               &  0.0012 &  0.0023 \\
      $\dmd$                                  &  {---}  &  0.0034 \\
      Upper decay time acceptance             &  {---}  &  0.0012 \\
      Correlation between mass and decay time &  {---}  &  {---} \\
      Decay time resolution calibration       &  {---}  &  {---} \\
      Decay time resolution offset            &  {---}  &  {---} \\
      Low decay time acceptance               &  {---}  &  {---} \\
      Production asymmetry                    &  {---}  &  {---} \\
      \midrule
      Sum                                     &  0.020  &  0.005 \\
      \bottomrule
    \end{tabular}
\end{table}
