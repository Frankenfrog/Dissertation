%!TEX root = ../main.tex

\section{Backgrounds (2 pages)}
\label{sec:bd2jpsiks:backgrounds}

Although \BdToJPsiKS is an experimentally very clean decay channel care has to
be taken to properly identify, suppress or even reject backgrounds. While the
two muons can be identified quite effectively the pions of the \KS decay might
actually be kaons or protons which have been mis-identified. This would lead
to background contributions from \BdToJPsiKst and \LbToJPsiL. To analyse the
$\proton \to \pion$ mis-ID the proton mass hypothesis is assigned to one of
the pions and the invariant mass of the proton-pion pair $m_{\proton\pion}$ is
recalculated. An excess of candidates at the \Lz mass $M_{\Lz} =
\SI{1115.683}{\MeVcc}$~\cite{PDG2014} can be seen which is reduced by applying
a tighter requirement on the difference of the proton-pion log-likelihood for
candidates close to $M_{\Lz}$. With \LbToJPsiL signal MC it is checked that
after reconstruction, stripping and all offline selection requirements,
including the veto described above, the expected yield is a sub-percent
effect. For $\kaon \to \pion$ mis-ID the broad width of the \Kstarz does not
allow an analogous approach. But studies on \BdToJPsiKst MC show that the
expected contribution is even lower than for \LbToJPsiL. The main reason is
the short lifetime of the \Kstarz which is exploited by the lifetime
significance cut on the \KS. So, it can basically be assumed that besides the
signal candidates almost only combinatorial background is present in the data
sample. Nevertheless, it has to be studied whether the background shows any
tagging-dependent asymmetry which would dilute the measured \CP asymmetry.

By performing a fit to the invariant mass distribution the \sPlot technique
provides a possibility to study the tagging-dependent distributions of the
combinatorial background. First of all, the time-integrated asymmetry
\begin{align}
  \mathcal{A}_\text{bkg}^{\text{int}}= \frac{\param{N}{\Bzb}{bkg} - \param{N}{\Bz}{bkg}}{\param{N}{\Bzb}{bkg} + \param{N}{\Bz}{bkg}}
\end{align}
is calculated for both track type categories and separately for the OS
combination and the SS\pion tagger. Out of the four values listed in
\cref{tab:bkgtimeintegratedasymm} only the one for the downstream OS tagged
sample stands out as it disfavors \CP symmetry at more than \num{3} standard
deviations.
%
\begin{table}[!htb]
\centering
\caption{Time-integrated asymmetry of \sweighted background distributions for
downstream and long track OS and SS\pion tagged events.}
\label{tab:bkgtimeintegratedasymm}
\begin{tabular}{lr@{$\,\pm\,$}l}
\toprule
category    & \multicolumn{2}{c}{$\mathcal{A}_\text{bkg}^{\text{int}}$}\\
\midrule
DD OS       & $0.017$     & $0.005$ \\
DD SS\pion  & $-0.016$    & $0.011$ \\
LL OS       & $-0.005$    & $0.012$ \\
LL SS\pion  & $0.044$     & $0.034$ \\
\bottomrule
\end{tabular}
\end{table}
%