%!TEX root = ../main.tex
\newpage
\section{Decay time resolution and acceptance (3 pages)}
\label{sec:bd2jpsiks:decaytime}

\subsection{Decay time resolution}
\label{sec:bd2jpsiks:decaytime:resolution}

Although the determination of vertices and the measurement of momenta is
pretty accurate at \lhcb (especially thanks to the VELO) a finite resolution
remains which causes a finite decay time resolution. This becomes most obvious
when candidates with negative decay times are reconstructed. However, these
candidates can be used to determine the decay time resolution. An unbinned
likelihood fit to the decay time distribution of an unbiased \BdToJPsiKS
sample with true \jpsi candidates (selected by a fit of the invariant
$m_{\mumu}$ mass distribution) is performed. The fit model consists of a
component for the prompt peak around \SI{0}{\ps}, \ie the decay time
resolution model, a component to describe events where the wrong PV has been
associated and therefore a large difference between true and reconstructed
decay time occurs, and long lived components which are parametrised with two
exponentials with different pseudo lifetimes and which are themselves
convolved with the decay time resolution model. The decay time resolution
depends on characteristics of the event. The DTF provides predictions for the
per-event decay time resolution $\sigma_t$ which in principle could be used as
width of a Gaussian resolution model. Unfortunately, these predictions are not
perfect and a certain calibration needs to be applied. Additionally, to
account for different sources causing the decay time resolution an effective
model with two Gaussian functions which share a common mean but have different
calibrations is used. The first step is to find a reasonable calibration
model. A linear ($f_1$) and a quadratic ($f_2$) calibration model are tested:
\begin{equation}
\begin{aligned}
f_1:&\quad \sigma_{\text{true}}(\sigma_t) &=&  &b_i\,\sigma_t &+ c_i \ , \\
f_2:&\quad \sigma_{\text{true}}(\sigma_t) &=&\, \alpha_i\,\sigma_t^2 + &\beta_i\,\sigma_t &+ \gamma_i \ .
\end{aligned}
\label{eq:resolutioncalibfunctions}
\end{equation}
%
The data sample is divided into \num{20} equally filled bins of the decay time
resolution predictions $\sigma_t$. This is done separately for the downstream
and the long track sample as the decay time resolution of long track
candidates is expected to be significantly better. Under the assumption that
$\sigma_t$ is constant inside the bins average widths can be set for the two
Gaussian functions. An unbinned likelihood fit to the decay time distribution,
simultaneous in all bins, sharing all fit parameters except the widths of the
Gaussian resolution functions is performed. The widths are plotted in the
corresponding bins and a \chisq-fit with the two calibration functions is executed.

\subsection{Decay time acceptance}
\label{sec:bd2jpsiks:decaytime:acceptance}