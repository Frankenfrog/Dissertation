%!TEX root = ../main.tex
\section{Decay time resolution}
\label{sec:bd2jpsiks:decaytime:resolution}

The most obvious effect of the decay time resolution are candidates that are
reconstructed with negative decay times. However, these are ideal candidates
to determine the decay time resolution. An unbinned likelihood fit to the
decay time distribution of an unbiased \BdToJPsiKS sample with true \jpsi
candidates (selected by a fit of the invariant $m_{\mumu}$ mass distribution)
is performed. The fit model consists of a component for the prompt peak around
\SI{0}{\ps}, \ie the decay time resolution model, a component to describe
events where the wrong PV has been associated and therefore a large difference
between true and reconstructed decay time occurs, and long lived components,
which are parametrised with two exponentials with different pseudo lifetimes,
and which are themselves convolved with the decay time resolution model. The
decay time resolution depends on characteristics of the event. The DTF
provides predictions for the per-event decay time resolution $\sigma_t$, which
in principle could be used as width of a Gaussian resolution model.
Unfortunately, these predictions are not perfect and a certain calibration
needs to be applied. Additionally, to account for different sources causing
the decay time resolution an effective model with two Gaussian functions, which
share a common mean but have different calibrations, is used.

The first step is to find a reasonable calibration model. A linear ($f_1$) and
a quadratic ($f_2$) calibration model are tested:
\begin{equation}
\begin{aligned}
f_1:&\quad \sigma_{\text{true}}(\sigma_t) &=&  &b_i\,\sigma_t &+ c_i \ , \\
f_2:&\quad \sigma_{\text{true}}(\sigma_t) &=&\, \alpha_i\,\sigma_t^2 + &\beta_i\,\sigma_t &+ \gamma_i \ .
\end{aligned}
\label{eq:resolutioncalibfunctions}
\end{equation}
%
The data sample is divided into \num{20} equally filled bins of the decay time
resolution predictions $\sigma_t$. This is done separately for the downstream
and the long track sample as the decay time resolution of long track
candidates is expected to be significantly better. Under the assumption that
$\sigma_t$ is constant inside the bins average widths can be set for the two
Gaussian functions. An unbinned likelihood fit to the decay time distribution,
simultaneous to all bins, sharing all fit parameters except the widths of the
Gaussian resolution functions is performed. The widths are plotted in the
corresponding bins and a \chisq-fit with the two calibration functions is
executed. For the downstream sample the results are depicted in
\cref{fig:CalibrationOffsetResolution_DD}. \todo{replace plots with different line styles}
%
\begin{figure}[!htb]
\centering
  \includegraphics[width=0.45\textwidth]{06-Bd2JpsiKS/figs/ResolutionCalibration_1_DD.pdf}
  \includegraphics[width=0.45\textwidth]{06-Bd2JpsiKS/figs/ResolutionCalibration_2_DD.pdf}
\caption{Fit of a linear (green) and a quadratic calibration function (blue)
to the narrower (left) and to the wider width (right) of the downstream
sample. Fixing the offset of the quadratic function to zero (yellow) results
in an unphysical shape.}
\label{fig:CalibrationOffsetResolution_DD}
\end{figure}
%
Both functions fit equally well so the simpler linear model with less degrees
of freedom is preferred. The liner model is also chosen for the long track
sample.

With the calibration functions at hand a likelihood fit unbinned in decay
times and decay time resolution predictions is performed and the nominal
values of the calibration parameters are determined. The dilution factor
induced by the decay time resolution is calculated to be \num{0.986} for
downstream and \num{0.989} for long track candidates.

\section{Decay time acceptance}
\label{sec:bd2jpsiks:decaytime:acceptance}

The trigger line requirements applied in the selection of the \BdToJPsiKS
candidates partially bias the decay time distribution. The sample is divided
into an \emph{almost unbiased} subset and an \emph{exclusively biased} subset.