%!TEX root = ../main.tex

\section{Data Preparation (1 page)}
\label{sec:bd2jpsiks:datapreparation}


\BdToJPsiKS decays are reconstructed through the subsequent decays
\mbox{\JPsiToMuMu} and \KSToPiPi. Only combinations of two long or two
downstream tracks are considered for the pions. All events must have passed
either the \texttt{L0Muon} or the \texttt{L0DiMuon} trigger line. This is
implicitly given by the explicit requirement that a positive \jpsi TOS
decision of the \texttt{DiMuonHighMass} or the \texttt{TrackMuon} trigger line
in the Hlt1 stage and of the \texttt{DiMuonDetachedJPsi} trigger line in the
Hlt2 stage exists. These trigger requirements have a total signal efficiency
of about $\SI{85}{\percent}$.

In the stripping loose requirements on the quality of the \jpsi, the \KS and
the \Bd vertex are applied. The invariant masses of the $\mumu$ and the
$\pip\pim$ combination are required to be roughly consistent with the known
\jpsi and \KS masses, respectively. Moreover, the \KS candidates must have a
significant decay length and some further kinematic requirements need to be
fulfilled. Despite these rather loose selection requirements, the signal to
background ratio is already quite good. Therefore, in the offline selection
almost all cuts are tuned to have a high signal efficiency. The probability of
the muon and pion tracks to be ghost tracks (see
\cref{sec:detector:software:reconstruction}) is reduced to
\SIlist{20;30}{\percent}, respectively. The mass window around the \jpsi meson
is tightened to correspond to five standard deviations. For the \KS meson the
mass window is adapted to the track type of the pions. It contains four and
eight standard deviations for the long and the downstream candidates,
respectively. A specific treatment against misidentified \LbToJPsiL and
\BdToJPsiKstar decays is performed. The former are rejected by tighter PID
requirements on the pions if the invariant mass under a \pion\proton mass
hypothesis is compatible with the $\Lambda$ mass~\cite{PDG2014}. The latter
are suppressed with a cut on the \KS decay time in units of its uncertainty.

The invariant \Bd mass, which is restricted to candidates inside
\SIrange{5230}{5330}{\MeVcc}, is obtained from a fit to the whole decay chain
(DTF) with constraints on the \jpsi and \KS masses. The decay-time related
observables stem from a DTF with a PV constraint. To facilitate the
description of the decay time acceptance and to further suppress prompt
combinatorial background, only candidates with $t > \SI{0.3}{\ps}$ are kept.
In the last step of the selection one candidate is chosen randomly for events
where multiple candidates have survived the previously described selection.
