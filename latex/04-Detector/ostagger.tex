%!TEX root = ../main.tex

\subsubsection{Opposite-side flavour tagging}
\label{sec:detector:software:tagging:ostagging}

Opposite-side flavour-tagging algorithms~\cite{LHCb-PAPER-2011-027} infer the
flavour of the signal $B$ meson by studying the decay process of the second
$b$ hadron, which is produced from the same $\bbbar$ quark pair as the
reconstructed signal $B$ meson. Mistag probabilities are on the one hand
introduced by selecting a wrong tagging particle and on the other hand
intrinsically if the opposite-side $b$ hadron is neutral and has already mixed
at the time of decay.

In case of a semileptonic decay of the opposite-side $b$ hadron the charges of
the leptons are used by the OS electron (OS$e$) and OS muon (OS$\mu$) tagger
to determine the flavour. These two taggers provide relatively good mistag
estimates of around \SI{30}{\percent}, but have quite low tagging efficiencies
of around \SI{2}{\percent} (OS$e$) and \SI{5}{\percent} (OS$\mu$) for
charmonium respectively \SI{3.5}{\percent} (OS$e$) and \SI{8.5}{\percent}
(OS$\mu$) for open charm modes. The efficiency for muons is a factor
\numrange{2}{3} higher than for electrons due to the better reconstruction and
identification with the LHCb detector.

The OS kaon tagger selects kaons from a $\bquark\!\to\cquark\!\to\squark$
decay chain. Its tagging efficiency is around \SI{17}{\percent}
(\SI{21}{\percent}) for charmonium (open charm) modes at an average mistag
probability of approximately \SI{39}{\percent}.

The OS vertex charge tagger reconstructs the secondary vertex of the
opposite-side $b$ hadron and calculates the average charge of all tracks
associated to this vertex. The tagging efficiency and mistag probability are
comparable with the OS kaon tagger.

The recent OS charm tagger~\cite{LHCb-PAPER-2015-027} reconstructs charm
hadron candidates produced through $\bquark\!\to\cquark$ transitions of the
opposite-side $b$ hadron. The main contribution to the tagging power of the OS
charm tagger comes from partially reconstructed charm hadrons in $\Km\pip X$
final states and from \DzToKpi decays. The overall tagging efficiency of the
OS charm tagger is only \SIrange{3}{5}{\percent} with a mistag probability of
around \SI{35}{\percent}.
\todo{reference for tagging performance numbers?}

The four taggers described first are usually combined into an OS combination.
Since the release of the OS charm tagger a new OS combination is defined,
which is slightly better than the old one.