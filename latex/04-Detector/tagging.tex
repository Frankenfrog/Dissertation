%!TEX root = ../main.tex

\subsection{Flavour tagging}
\label{sec:detector:software:tagging}

For measurements of $\CP$ violation in $B$ decays it is essential to know the
initial flavour of the decaying $\bquark$ hadron candidate, \ie whether it
contained a $\bquark$ or a $\bquarkbar$ quark at production. When studying
decays of charged $B$ mesons the flavour at decay matches the production
flavour. Therefore, the flavour can unambiguously be determined from the
charges of the final state particles. Due to meson oscillations it is not as
trivial for neutral mesons. Instead, dedicated methods called flavour-tagging
algorithms are needed, which infer the initial flavour of a reconstructed
candidate from other particles inside the event. The $B$-factories \babar and
\belle have been operated at the $\FourS$ resonance, which dominantly decays into a
quantum-correlated pair of $B\Bbar$ mesons. Therefore, by analysing the decay
of the non-signal $B$ meson, \eg if it proceeds via a flavour-specific
process, the flavour of the signal $B$ meson at that time could be determined.
Such correlations are not present at proton-proton colliders like the \lhc,
where $\bquark$ quarks are dominantly produced in $\bbbar$ quark pairs via
gluon-gluon fusion. The LHCb collaboration has developed several flavour
tagging algorithms, which can be classified as \emph{same-side} (SS) and
\emph{opposite-side} (OS) taggers. A schematic overview of all current taggers
that can be exploited to tag $\Bd$ mesons is given in
\cref{fig:detector:tagging:schematics}.
\begin{figure}[htb]
\centering
%!TEX root = ../main.tex

% Colours 
\definecolor{fcdOrnA}{HTML}{331605}
\definecolor{fcdOrnB}{HTML}{662C0A}
\definecolor{fcdOrnC}{HTML}{99420F}
\definecolor{fcdOrnD}{HTML}{CC5814}
\definecolor{fcdOrnE}{HTML}{FF6E19}
\definecolor{fcdOrnF}{HTML}{FF975B}
\definecolor{fcdOrnG}{HTML}{FFAC7C}
\definecolor{fcdOrnH}{HTML}{FFC19C}
\definecolor{fcdOrnI}{HTML}{FFD6BD}
\definecolor{fcdOrnJ}{HTML}{FFEADE}
\definecolor{fcdBluA}{HTML}{052A33}
\definecolor{fcdBluB}{HTML}{0A5466}
\definecolor{fcdBluC}{HTML}{0F7E99}
\definecolor{fcdBluD}{HTML}{14A8CC}
\definecolor{fcdBluE}{HTML}{19D2FF}
\definecolor{fcdBluF}{HTML}{5BDFFF}
\definecolor{fcdBluG}{HTML}{7CE5FF}
\definecolor{fcdBluH}{HTML}{9CECFF}
\definecolor{fcdBluI}{HTML}{9CECFF}
\definecolor{fcdBluJ}{HTML}{DEF9FF}
\definecolor{fcdGrnA}{HTML}{243304}
\definecolor{fcdGrnB}{HTML}{476608}
\definecolor{fcdGrnC}{HTML}{6B990D}
\definecolor{fcdGrnD}{HTML}{A0E02D}
\definecolor{fcdGrnE}{HTML}{B2FF15}
\definecolor{fcdGrnF}{HTML}{C8FF58}
\definecolor{fcdGrnG}{HTML}{D3FF79}
\definecolor{fcdGrnH}{HTML}{DEFF9B}
\definecolor{fcdGrnI}{HTML}{E9FFBC}
\definecolor{fcdGrnJ}{HTML}{F4FFDE}
\definecolor{fcdVltA}{HTML}{310433}
\definecolor{fcdVltB}{HTML}{620866}
\definecolor{fcdVltC}{HTML}{930D99}
\definecolor{fcdVltD}{HTML}{C411CC}
\definecolor{fcdVltE}{HTML}{F514FF}
\definecolor{fcdVltF}{HTML}{F858FF}
\definecolor{fcdVltG}{HTML}{F979FF}
\definecolor{fcdVltH}{HTML}{FB9BFF}
\definecolor{fcdVltI}{HTML}{FCBCFF}
\definecolor{fcdVltJ}{HTML}{FEDDFF}

\definecolor{fcdGrayA}{HTML}{111111}
\definecolor{fcdGrayB}{HTML}{222222}
\definecolor{fcdGrayC}{HTML}{333333}
\definecolor{fcdGrayD}{HTML}{444444}
\definecolor{fcdGrayE}{HTML}{555555}
\definecolor{fcdGrayF}{HTML}{666666}
\definecolor{fcdGrayG}{HTML}{777777}
\definecolor{fcdGrayH}{HTML}{888888}
\definecolor{fcdGrayI}{HTML}{999999}
\definecolor{fcdGrayJ}{HTML}{AAAAAA}
\definecolor{fcdGrayK}{HTML}{BBBBBB}
\definecolor{fcdGrayL}{HTML}{CCCCCC}
\definecolor{fcdGrayM}{HTML}{DDDDDD}
\definecolor{fcdGrayN}{HTML}{EEEEEE}

\definecolor{fcdTropiteal}    {HTML}{00A8C6}
\definecolor{fcdTealDrop}     {HTML}{40C0CB}
\definecolor{fcdWhiteTrash}   {HTML}{F9F2E7}
\definecolor{fcdAtomicBikini} {HTML}{AEE239}
\definecolor{fcdFeebleWeek}   {HTML}{8FBE00}





\colorlet{ClrTxt}{black}
\colorlet{ClrTxtVeryDarkGray}{fcdGrayE}
\colorlet{ClrTxtDarkGray}{fcdGrayJ}
\colorlet{ClrVtxGray}{fcdGrayM}

\colorlet{ClrSigQuark}{fcdTealDrop}
\colorlet{ClrSigMeson}{fcdTropiteal}
\colorlet{ClrSigArrow}{fcdTropiteal}

\colorlet{ClrTagQuark}{fcdAtomicBikini}
\colorlet{ClrTagMeson}{fcdFeebleWeek}
\colorlet{ClrTagArrow}{fcdFeebleWeek}

\begin{tikzpicture}[
  scale=1, 
  >=stealth',
  font=\small,
  quark_sig/.style={
    align=center, 
    minimum size=3ex,
    circle,
    color=ClrSigQuark,
    fill=ClrSigQuark,
    text=ClrTxt,
    draw, 
    thick,
    inner sep=0pt,
    outer sep=0pt,
    node distance=0ex
  },
  quark_tag/.style={
    align=center, 
    minimum size=3ex,
    circle,
    color=ClrTagQuark,
    fill=ClrTagQuark,
    text=ClrTxt,
    draw, 
    thick,
    inner sep=0pt,
    outer sep=0pt,
    node distance=0ex
  },
  meson_sig/.style={
    draw, 
    align=center, 
    minimum size=4.5ex,
    circle,
    color=ClrSigMeson,
    fill=ClrSigMeson,
    text=ClrTxt,
    thick,
    inner sep=0pt,
    outer sep=1pt,
    node distance=0ex
  },
  meson_tag/.style={
    draw, 
    align=center, 
    minimum size=4.5ex,
    circle,
    color=ClrTagMeson,
    fill=ClrTagMeson,
    text=ClrTxt,
    thick,
    inner sep=0pt,
    outer sep=1pt,
    node distance=0ex
  },
  meson_comb/.style={
    shape=ellipse, 
    draw,
    fill,
    very thick,
    inner sep=0pt,
    outer sep=0pt,
    minimum width=8ex,
    minimum height=5ex},
  vertex/.style={
    shape=ellipse,
    draw,
    inner sep=1ex,
    outer sep=0pt,
    minimum width=10ex,
    minimum height=10ex,
    color=ClrVtxGray,
    fill=ClrVtxGray,
    text=ClrTxt,
    node distance=1ex
  },
  vertex_label/.style={
    inner sep=0pt,
    outer sep=0pt,
    text=ClrTxtDarkGray,
    node distance=1ex,
    font=\sffamily\small
  },
  tagger_label/.style={
    inner sep=0pt,
    outer sep=0pt,
    text=ClrTxtVeryDarkGray,
    node distance=1ex,
    font=\sffamily\small    
  },
  arrow_sig/.style={
    ->,
    very thick,
    color=ClrSigArrow
  },
  arrow_tag/.style={
    ->,
    very thick,
    color=ClrTagArrow
  }
]

%\draw[help lines] (-1,-5) grid (11,5);

\draw[dashed,color=fcdGrayE] (0,0) -- (12,0);
\node[text width=2.5cm,text=ClrTxtDarkGray,font=\sffamily\small] (SST) at (10.5,+0.3) {\hfill same side};
\node[text width=2.5cm,text=ClrTxtDarkGray,font=\sffamily\small] (OST) at (10.5,-0.3) {\hfill opposite side};


\node[draw,circle,fill,color=ClrSigQuark,inner sep=0pt,minimum size=8pt] 
  (coll) at (0,0) {};

\draw[<-,very thick] (coll) -- (+1,0);
\draw[->,very thick] (-1,0) -- (coll);

\begin{pgfonlayer}{foreground}
  
  % bbbar
  \node[quark_sig] (qrk_bbar) at (0.2,+0.8) {$\kern 0.2em\overline{\kern -0.1em b}$};
  \node[quark_sig] (qrk_b) at (0.2,-0.8) {$b$};
  
  \path (qrk_bbar) 
        to[circle connection bar switch color=from (ClrSigQuark) to (ClrSigQuark)] 
        (coll);
  \path (qrk_b) 
        to[circle connection bar switch color=from (ClrSigQuark) to (ClrSigQuark)] 
        (coll);
  
  % Signal decay
  \node[meson_sig,color=ClrSigMeson,fill=ClrSigMeson,text=ClrTxt] (SigDp) at (7.5,+2.25) {$\Dp$}  ;
  \node[meson_sig,color=ClrSigMeson,fill=ClrSigMeson,text=ClrTxt] (SigDm) [below=of SigDp] {$\Dm$}  ;
  
  % SS tagging
  \path (qrk_bbar) ++(15:3ex) node (qrk_d) [quark_tag] {$d$};
  
  \node[quark_tag] (qrk_dbar) at (0.8,+2.2) {$\kern 0.2em\overline{\kern -0.2em d}$};
  
  \node[draw,circle,fill,color=ClrTagQuark,inner sep=0pt,minimum size=4pt] 
    (vacuumexc) at ([xshift=-2ex] $(qrk_dbar)!0.5!(qrk_d)$) {};
  
  \path (qrk_d) 
        to[circle connection bar switch color=from (ClrTagQuark) to (ClrTagQuark)] 
        (vacuumexc);
  \path (qrk_dbar) 
        to[circle connection bar switch color=from (ClrTagQuark) to (ClrTagQuark)] 
        (vacuumexc);  
  \path (qrk_dbar) ++(165:3ex) node (qrk_u)    [quark_tag] {$u$};

  \node[quark_tag] (qrk_dbar2) at (1.0,+3.2) {$\kern 0.2em\overline{\kern -0.2em d}$};
  \path (qrk_dbar2) ++(165:3ex) node (qrk_ubar1)  [quark_tag] {$\kern 0.2em\overline{\kern -0.2em u}$};
  \path (qrk_ubar1) ++(165:3ex) node (qrk_ubar2)  [quark_tag] {$\kern 0.2em\overline{\kern -0.2em u}$};
  \node[draw,circle,fill,color=ClrTagQuark,inner sep=0pt,minimum size=4pt] 
    (vacuumexc2) at ([xshift=+4ex] $(qrk_dbar2)!0.5!(qrk_d)$) {};
  \path (qrk_d)
        to[circle connection bar switch color=from (ClrTagQuark) to (ClrTagQuark)] 
        (vacuumexc2);
  \path (qrk_dbar2)
        to[circle connection bar switch color=from (ClrTagQuark) to (ClrTagQuark)] 
        (vacuumexc2);

  \node[meson_tag] (SSpip) at (3.3,+2.5) {$\pi^{+}$}; 
  \node[meson_tag] (SSproton) at (3.3,+3.4) {$\antiproton$}; 
  

  % OS tagging
  \path (qrk_b)    ++(-15:3ex) node (qrk_xbar) [quark_tag] {$\kern 0.2em\overline{\kern -0.1em x}$};  

  % Tagging particles 
  \node[meson_tag] (OSlepton) at (9.5,-2.25) {$l^{-}$};
  \node[meson_tag] (OSkaon)   at (9.5,-1)    {$K^{-}$};
  
\end{pgfonlayer} 
  
  
\node[meson_comb,rotate=+15,color=ClrSigMeson,text=ClrTxt] (SigBz) at ($(qrk_bbar)!0.5!(qrk_d)$) {  }; 
\node[meson_comb,rotate=-15,color=ClrTagMeson,text=ClrTxt] (Hb) at ($(qrk_b)!0.5!(qrk_xbar)$) {}; 
\node[meson_comb,rotate=-15,color=ClrTagMeson,text=ClrTxt] (pip_SS) at ($(qrk_dbar)!0.5!(qrk_u)$)   {}; 
\node[meson_comb,rotate=-15,color=ClrTagMeson,text=ClrTxt,fit=(qrk_ubar1)(qrk_dbar2)(qrk_ubar2)] (antiproton) {}; 

  
\begin{pgfonlayer}{background}
  \node[vertex,fit=(SigBz)(Hb)(pip_SS)(antiproton),minimum width=16ex] (PV) {};
  \node[vertex_label,align=center] (PV_label) [above=of PV] {PV};
  
  \node[vertex,fit=(SigDp)(SigDm)   ,minimum width=10ex] (SigSV) {};
  \node[vertex_label,align=center] (SigSV_label) [above=of SigSV] {SV};
  
  \node[vertex,minimum width=11ex,minimum height=9ex,align=left] (OS_SV) at (4.5,-1.75) {$b \to c$\\  $b\to X l^{-}$};
  \node[vertex_label,align=center] (OS_SV_label) [above=of OS_SV] {SV};
  
  
  \node[vertex,minimum width=9ex,minimum height=7ex,align=center] (OS_TV) at (7.5,-1.25) {$c \to s$};
\end{pgfonlayer}


\begin{pgfonlayer}{foreground}
  \draw[arrow_sig] (SigBz)  -- node (Bz_label) [above] {$B^{0}$} ([xshift=+4pt] SigSV.west);
  \draw[arrow_sig] let \p1 =(SigDp.east) in (\x1-1,\y1+3)  -- (\x1+50,\y1+6);
  \draw[arrow_sig] let \p1 =(SigDp.east) in (\x1-1,\y1)  -- (\x1+50,\y1);
  \draw[arrow_sig] let \p1 =(SigDp.east) in (\x1-1,\y1-3)  -- (\x1+50,\y1-6);
  \draw[arrow_sig] let \p1 =(SigDm.east)   in (\x1-1,\y1+3)  -- (\x1+50,\y1+6);
  \draw[arrow_sig] let \p1 =(SigDm.east)   in (\x1-1,\y1)  -- (\x1+50,\y1);
  \draw[arrow_sig] let \p1 =(SigDm.east)   in (\x1-1,\y1-3)  -- (\x1+50,\y1-6);
    
  \draw[arrow_tag] (Hb)     -- node (Hb_label) [above] {$h_b$}  ([xshift=+4pt] OS_SV.west);
  \draw[arrow_tag] (OS_SV)  -- ([xshift=+4pt] OS_TV.west);
  \draw[arrow_tag] (OS_SV)  -- ([xshift=+0pt] OSlepton.west);
  \draw[arrow_tag] (OS_TV)  -- ([xshift=+0pt] OSkaon.west);
  \draw[arrow_tag] (pip_SS) -- ([xshift=+0pt] SSpip.west);
  \draw[arrow_tag] (antiproton) -- ([xshift=+0pt] SSproton.west);

  \node[tagger_label,align=left] (SSpip_label) [right=of SSpip] {SS pion};
  \node[tagger_label,align=left] (SSproton_label) [right=of SSproton] {SS proton};
  \node[tagger_label,align=left] (OSlepton_label) [right=of OSlepton] {OS muon\\ OS electron};
  \node[tagger_label,align=left] (OSkaon_label)   [right=of OSkaon] {OS kaon};
  \node[tagger_label,align=center] (OSVtcCh_label) [below=of OS_SV] {OS vertex charge\\ OS charm};

\end{pgfonlayer}

\end{tikzpicture}

\caption{Available tagging algorithms to tag \Bz mesons at the LHCb experiment.}
\label{fig:detector:tagging:schematics}
\end{figure}
By convention each flavour-tagging algorithm provides a flavour tag of $d =
\num{+1}$ if the tagger decides that it is more likely that the flavour of the
initial $B$ meson was a \Bz and of $d = \num{-1}$ if, based on the algorithm,
a \Bzb flavour is more likely. However, when no appropriate tagging particle
can be found for a reconstructed candidate, a tag decision of $d = 0$ is
assigned. The tag decisions are either based on the charge of a single
selected tagging particle or on the sign of the averaged charge of multiple
tagging particles. Besides the tag decision each tagger also provides a
prediction $\eta$ on the probability that the tag decision is wrong. This
mistag probability $\eta$ takes values between \numlist{0;0.5}, where $\eta =
0$ means that there is no uncertainty on the tag decision and $\eta = 0.5$
basically corresponds to a random choice and is associated with $d = 0$. These
predictions are based on the outcome of multivariate classifiers, which
combine kinematic and geometric properties of the tagging particle as well as
information on the event.

The performance of flavour-tagging algorithms can be quantified by the tagging
efficiency \etag, which specifies for how many reconstructed candidates a tag
decision can be made, and by the true mistag probability \mistag. The relation
between the predicted and the true mistag probability $\omega(\eta)$ is
determined in calibration studies (see
\cref{sec:dataanalysis:taggingcalibration}). From the mistag probability the
tagging dilution $D = 1 - 2\omega$ can be derived, which indicates how much a
measured amplitude is reduced with respect to the physical amplitude due to
wrong tags. The product of the tagging efficiency and the squared tagging
dilution $\effeff = \etag D^2$ is called tagging power or effective tagging
efficiency. It is widely used as figure of merit for tagging algorithms as it
states the effective loss in statistics compared to a perfectly tagged sample.
Ideally, the tagging power is calculated on a per-candidate basis by summing up the dilution of all $N$ signal candidates, with $\omega = 0.5$
($D=0$) for the untagged candidates, according to
\begin{equation}
	\effeff = \frac{1}{N}\sum_{i=1}^{N} (1 - 2\omega(\eta_i))^2\,.
\label{eq:detector:tagging:taggingpowerperevent}
\end{equation}

While further information on the basic principles of LHCb's flavour-tagging
algorithms can be found in Refs.~\cite{LHCb-CONF-2011-003,Grabalosa:1456804},
a short description of the OS and SS taggers will be given in
\cref{sec:detector:software:tagging:ostagging,sec:detecor:software:tagging:sstagger}.

%!TEX root = ../main.tex

\subsection{Opposite-side flavour tagging}
\label{sec:detector:tagging:ostagging}

Opposite-side flavour-tagging algorithms~\cite{LHCb-PAPER-2011-027} infer the
flavour of the signal $B$ meson by studying the decay process of the second
$b$ hadron, which is produced from the same $\bbbar$ quark pair as the
reconstructed signal $B$ meson. Mistag probabilities are on the one hand
introduced by selecting a wrong tagging particle and on the other hand
intrinsically if the opposite-side $b$ hadron is neutral and has already mixed
at the time of decay.

In case of a semileptonic decay of the opposite-side $b$ hadron the charges of
the leptons are used by the OS electron (OS$e$) and OS muon (OS$\mu$) tagger
to determine the flavour. These two taggers provide relatively good mistag
estimates of around \SI{30}{\percent}, but have quite low tagging efficiencies
of around \SI{2}{\percent} (OS$e$) and \SI{5}{\percent} (OS$\mu$) for
charmonium respectively \SI{3.5}{\percent} (OS$e$) and \SI{8.5}{\percent}
(OS$\mu$) for open charm modes. The efficiency for muons is a factor
\numrange{2}{3} higher than for electrons due to the better reconstruction and
identification with the LHCb detector.

The OS kaon tagger selects kaons from a $\bquark\!\to\cquark\!\to\squark$
decay chain. Its tagging efficiency is around \SI{17}{\percent}
(\SI{21}{\percent}) for charmonium (open charm) modes at an average mistag
probability of approximately \SI{39}{\percent}.

The OS vertex charge tagger reconstructs the secondary vertex of the
opposite-side $b$ hadron and calculates the average charge of all tracks
associated to this vertex. The tagging efficiency and mistag probability are
comparable with the OS kaon tagger.

The recent OS charm tagger~\cite{LHCb-PAPER-2015-027} reconstructs charm
hadron candidates produced through $\bquark\!\to\cquark$ transitions of the
opposite-side $b$ hadron. The main contribution to the tagging power of the OS
charm tagger comes from partially reconstructed charm hadrons in $\Km\pip X$
final states and from \DzToKpi decays. The overall tagging efficiency of the
OS charm tagger is only \SIrange{3}{5}{\percent} with a mistag probability of
around \SI{35}{\percent}.
\todo{reference for tagging performance numbers?}

The four taggers described first are usually combined into an OS combination.
Since the release of the OS charm tagger a new OS combination is defined,
which is slightly better than the old one.

%!TEX root = ../main.tex

\subsubsection{Same-side flavour tagging}
\label{sec:detecor:software:tagging:sstagger}

Apart from $\bquark$ quarks being produced in \bbbar quark pairs, also
$\dquark$ quarks mainly stem from \ddbar quark pairs. In the hadronisation
process of the $B$ signal candidate charged pions and protons can be produced,
which contain the other quark from the \ddbar quark pair, and whose charge is
thereby correlated with the initial flavour of the reconstructed $B$ signal
candidate. Positive pions and antiprotons are associated with \Bz mesons, and
negative pions and protons with \Bzb mesons. Additionally, \Bd mesons can
originate from the decay $B^{*+}\!\to\Bd\pip$ of excited charged $B$ mesons.
Then, the charge of the associated pion again determines the initial flavour.
For quite some time, the only available same-side flavour-tagging algorithm for
\Bz mesons was a SS\pion tagger with a cut-based approach to select the
appropriate tagging pion. It has a tagging efficiency of around
\SI{15}{\percent} for $\JPsiX$ final states at an average mistag probability
of about \SI{42}{\percent}. Recently, an improved SS\pion tagger using a
boosted decision tree (BDT) to select the tagging pion and based on the very
same principles also a SS\proton tagger have been
developed\cite{CERN-THESIS-2015-040,LHCb-PAPER-2016-039}. These two
flavour-tagging algorithms select completely disjoint tagging particles
ensured by a requirement on the distance log-likelihood (DLL) between the pion
and the proton hypothesis of the tagging particle. The tagging efficiency of
the SS\pion tagger is very high with \SIrange{70}{75}{\percent}. The SS\proton
tagger provides non-zero tags for around \SI{35}{\percent} of all
reconstructed signal candidates, of which about \SI{80}{\percent} are also
tagged by the SS\pion (BDT). However, the high tagging efficiency comes along
with rather large average mistag probabilities of around \SI{45}{\percent}.
The response of the two BDT-based SS taggers is combined into a common SS
response.