%!TEX root = ../main.tex

\subsection{Reconstruction}
\label{sec:detector:software:reconstruction}

Two things need to be done in the reconstruction: tracks need to be find and
combined and particle hypotheses have to be assigned to these tracks. The
interface for the algorithms and tools is provided by the \brunel
project~\cite{Brunel}, based on the \gaudi framework~\cite{Barrand:2001ny}.

The forward tracking algorithm starts with straight VELO tracks, which are
built from hits in the R and $\phi$ sensors of the VELO modules. These are
extrapolated to match hits in the tracking stations taking into account the
bending of the tracks by the magnet. Then, corresponding hits in the TT are
added. A second tracking algorithm directly matches independent VELO and T
tracks. It is possible that certain track segments are used for different
tracks. In that case a Clone Killer algorithm selects one of the reconstructed
tracks. Another difficulty are ghost tracks, which are randomly combined hits
that do not stem from a real physics particle.

The particle identification (PID) of charged hadrons is performed via the allocation
of rings in the RICH detectors to the tracks and calculating likelihoods for
the different particle hypotheses. The calorimeters are used to identify
electrons and neutral pions, which decay into pairs of photons. One of the
best signatures is given by the muon system, which excludes respectively
settles the muon hypothesis quite reliable.