%!TEX root = ../main.tex

\subsection{Monte Carlo simulation}
\label{sec:detector:software:simulation}

In many ways data analyses benefit from the use of Monte Carlo simulations
(MC). This reaches from the calculation of efficiencies to the development of
selection strategies or to finding appropriate parametrisations to model data
distributions. One of the advantages of MC is that, except for the need of
enough computing and storage resources, the simulated samples can be very
large, typically considerably larger than the real data sample. The main goal
of the simulation is to be as close as possible to the conditions found on
real data. Therefore, constant comparison, calibration and adjustments are
needed. Whenever a deviation attracts attention, methods to compensate the
effect are applied, \eg the performance of the particle identification system
is overestimated on MC, which can be corrected by applying a data-driven
resampling.

For \lhcb the $pp$ collisions are generated using
\pythia~\cite{Sjostrand:2006za,*Sjostrand:2007gs} with a specific
configuration~\cite{LHCb-PROC-2010-056}. The decays of the hadronic particles
are simulated with \evtgen~\cite{Lange:2001uf}. To account for the radiation
of photons in the final-state the package
\photos~\cite{Golonka:2005pn} is used. The \geant
toolkit~\cite{Allison:2006ve, *Agostinelli:2002hh} is implemented as described
in Ref.~\cite{LHCb-PROC-2011-006} to model the interaction of the generated
particles with the detector material. The digitisation is realised using
\boole~\cite{Boole}. The further processing is identical with the analysis of real
collision data, starting with the trigger implemented in \moore~\cite{Moore},
the reconstruction done via \brunel~\cite{Brunel} and the stripping using the
\davinci package~\cite{DaVinci}.

Apart from the reconstructed properties of the particles, the true information
is available as well. This allows to compare the two and study resolution and
acceptance effects. Besides signal MC, where a specific decay mode is
specified in all details in a configuration file, it is also possible to
generate inclusive MC samples, which contain a whole family of similar decay
modes, \eg decays involving a \jpsi resonance and anything else, or even
completely unbiased samples.
