%!TEX root = ../main.tex

\section{The LHCb trigger system}
\label{sec:detector:trigger}

Deliberately, the instantaneous luminosity at \lhcb is reduced to
$\SI{4e32}{\cm^{-2}s^{-1}}$, which is significantly lower than at the other
three experiments at the \lhc. Though, the partial beam loss with time can be
compensated by adjusting of the beam crossing so that a constant luminosity
level can be kept throughout the whole fill. Nevertheless, it is not possible
to store the data of all visible proton-proton collisions. Instead, a two
stage trigger system consisting of a hardware (L0) and a subsequent software
level (HLT) is deployed. At the hardware trigger stage, which runs
synchronously with the bunch crossing rate of \SI{40}{\mega\hertz}, events are
required to contain at least one muon with a high \pt (\texttt{L0Muon}), or
two muons with a minimal product of their \pT (\texttt{L0DiMuon}), or a hadron
(\texttt{L0Hadron}), a photon (\texttt{L0Photon}) or an electron
(\texttt{L0Electron}), which deposit high transverse energy in the
calorimeters. Additionally, the number of allowed hits in the SPD is limited.
These requirements reduce the data rate to \SI{1}{\mega\hertz}, at which the
full detector can be read out. The L0 signal efficiency varies a lot between
muons and hadrons. While dimuon final states are triggered with more than
\SI{90}{\percent} efficiency, for fully hadronic final states like $\Dp\Dm$
only around \SI{60}{\percent} are reached. The high level trigger (HLT) is a
\cpp application, which runs on an event filter farm of several thousand CPU
nodes. It is again split into two stages. In the HLT1 basically the decisions
of the L0 are checked. Due to the reduced data rate some more time is
available. For all events the VELO tracks are reconstructed and a partial
event reconstruction of all charged particles with $\pt > \SI{500}{\mevc}$ in
2011 and $\pt > \SI{300}{\mevc}$ in 2012 is performed. This improves the
momentum resolution and enables to calculate some invariant masses. The
\texttt{Hlt1TrackMuon} trigger line requires a high \pt muon with a \chisqip
with respect to any primary interaction greater than 16, where \chisqip is
defined as the difference in \chisq of a given PV reconstructed with and
without the considered track. %which is incompatible with originating from any PV in the event. 
The \texttt{Hlt1DiMuonHighMass} trigger line accepts events if they contain
two muons that form a good common vertex with an invariant mass above
\SI{2.7}{\gevcc}. In HLT2, a full reconstruction of the event is performed.
Therefore, it is possible to further tighten the requirements applied in HLT1.
Furthermore, for the \texttt{Hlt2DiMuonDetachedJPsi} trigger line a
requirement on the flight distance is imposed. For hadrons, it is typically
searched for two-, three- or four-track secondary vertices, which are
identified via a multivariate algorithm~\cite{BBDT}.

The total output rate after all trigger stages has been increased from
\SI{3.5}{\kilo\hertz} in 2011 to \SI{5}{\kilo\hertz} in 2012 and
\SI{12.5}{\kilo\hertz} in Run II.

In the offline selection, trigger signals are associated with reconstructed
particles. Selection requirements can therefore be made on the trigger
selection itself and on whether the decision was due to the signal candidate
(TOS), other particles produced in the $pp$ collision (TIS), or a combination
of both.
