%!TEX root = ../main.tex

\chapter{The \lhcb Experiment at the \lhc}
\label{sec:detector}

The Large Hadron Collider beauty (\lhcb) experiment is one of the four large
experiments run at CERN, the European Organisation for Nuclear Research, in
Geneva, Switzerland. The physics goal of the collaborators at \lhcb is to test
the standard model of particle physics (SM) by performing indirect searches
for new physics with hadrons containing \bquark or \cquark quarks. In contrast
to the direct searches conducted by the multipurpose experiments ATLAS (A
Toroidal LHC Apparatus)~\cite{ATLAS} and CMS (Compact Muon
Solenoid)~\cite{CMS}, \CP-violating processes and rare decay modes are
precisely measured and the results are compared with the predictions of the
SM. This allows to investigate effects caused by heavy unknown particles at
energy scales far beyond what is accessible in direct searches. To do so, it is
mandatory to have a very high vertex, momentum and decay time resolution. The
detector, with which this demanding task is accomplished, is described in
\cref{sec:detector:lhcb}. Beforehand, a description of the accelerator
complex, which provides the proton bunches to collide inside the detector, is
given (see \cref{sec:detector:lhc}). Afterwards, the trigger system (see
\cref{sec:detector:trigger}) and the specific software (see
\cref{sec:detector:software}) are described.

%!TEX root = ../main.tex

\section{The Large Hadron Collider}
\label{sec:detector:lhc}

%!TEX root = ../main.tex

\newpage

\section{The LHCb detector (4 pages)}
\label{sec:detector:lhcb}

The \lhcb detector, depicted in \cref{fig:detector:scheme}, is a single-arm
forward spectrometer. This means that the individual detector
components are sequentially arranged in the forward direction, starting from
the interaction point. The angular acceptance in the horizontal plane is
\SIrange{10}{300}{mrad} and in the vertical plane \SIrange{10}{250}{mrad}.
Thereby, a pseudorapidity range $2<\eta <5$ is covered.
\begin{figure}[htb]
\centering
\includegraphics[width=\textwidth]{04-Detector/figs/Lhcbdetektor.pdf}
\caption{Schematic view on the \lhcb detector\cite{Alves:2008zz}.}
\label{fig:detector:scheme}
\end{figure}
Instrumenting only this part of the space has been found to be an optimal
compromise between cost and output for \lhcb's desired physics program, which
is mainly to study particles containing \bquark or \cquark quarks. Simulations
of the correlation between the angular distribution of \bbbar quark pairs (see
\cref{fig:detector:bbbar}) show that the \bquark and \bquarkbar quarks are
mainly produced in quite small cones around the beam axis. Of course, half of
the \bbbar quark pairs are going backwards but about \SI{25}{\percent} of all
\bbbar quark pairs are inside the instrumented \SI{4.5}{\percent} of the whole
space.
\begin{figure}[htb]
\centering
\includegraphics[width=0.49\textwidth]{04-Detector/figs/bbbarcorrelation.pdf}
\includegraphics[width=0.49\textwidth]{04-Detector/figs/bbbaracceptance.pdf}
\caption{Correlation of angular acceptance (left) and pseudorapidity (right)
of \bbbar quark pairs~\cite{LHCb-Technical-Proposal}. The frequency of
produced \bbbar quark pairs is indicated by the bin content in the plot
showing the angular acceptance and by the colour code in the plot of the
pseudorapidities, where purple means low and red corresponds to high. The
region marked in red (left) respectively the region in the red square is
instrumented by the \lhcb detector.}
\label{fig:detector:bbbar}
\end{figure}

More details on the structure of the \lhcb detector can be found in
Ref.~\cite{Alves:2008zz} and an overview of the performance is given in
Ref.~\cite{LHCb-DP-2014-002}.

\subsection*{Vertexing and tracking}
\label{subsec:tracker}

The tracking system consists of several detector components, one of which is a
dipole magnet, which bends the tracks of charged particles with an integrated
magnetic field of \SI{4}{Tm}. To be able to study charge-dependent detection
asymmetries the polarity of the dipole magnet is reversed periodically
throughout data-taking. From the curvature radius it can be concluded on
the momentum of the track. To determine the curvature radius information from
tracking detector elements located upstream and downstream of the magnet are
needed. The $pp$ interaction region is surrounded by a silicon-strip vertex
locator (VELO)~\cite{LHCb-DP-2014-001}, which delivers the most precise
information on the position of the tracks and vertices due to being installed
very closely around the beam pipe. It is composed of 42 modules with R and
$\phi$ sensors, which measure the positions of the tracks in cylindrical
coordinates. Each module is a half disk (see \cref{fig:detector:velo}, which
can be pulled to a proximity of \SI{5}{\milli\metre} to the beam axis.
However, this is only done for stable beam conditions otherwise the modules
could be destroyed by the beam. To monitor the beam position a dedicated
detector component called Beam Conditions Monitor (BCM) is installed at two
locations in the vicinity of the beam. Via eight diamond sensors, which have
been proven to be very radiation-hard, each station determines the particle
flux and can trigger a beam dump in case of instabilities, which occur
especially at the injection of proton bunches. The importance of this system
is underlined by the fact that it has its own power supply and constantly
reports its status. If no information from the BCM is received a beam dump is
also initiated. The VELO achieves a single hit resolution of up to \SI{4}{\mum}
at an efficiency of more than \SI{99}{\percent}. The disks are arranged in a
way that guarantees that even at the outermost acceptance of \SI{300}{mrad}
tracks hit at least three VELO stations (see \cref{fig:detector:velo}).
\begin{figure}[htb]
\includegraphics[width=0.4\textwidth]{04-Detector/figs/VELO.jpg}
\includegraphics[width=0.59\textwidth]{04-Detector/figs/AnordnungVELOStationen.pdf}
\caption{View on a single VELO half disk (left) and arrangement of all VELO
modules~\cite{Alves:2008zz}.}
\label{fig:detector:velo}
\end{figure}
One of the main purposes of the VELO is to precisely determine the position of
the proton-proton interaction called primary vertex (PV) and the displaced
secondary decay vertices of long-lived particles like \Bd mesons. The impact
parameter (IP), which is the minimum distance of a track to a PV, is measured
depending on the transversal momentum \pT with a spatial resolution of
$(15+29/\pt)\mum$ (\pT given in units of \si{\gevc}).

Between VELO and dipole magnet there is another silicon-strip detector, the
tracker turicensis (TT). Like the three tracking stations located downstream
of the magnet, which are subdivided into an inner silicon-strip tracker and an
outer straw drift tube detector~\cite{LHCb-DP-2013-003}, it is built of four
layers. While the first and last layer are arranged vertically, the inner
layers are tilted by \SI{-5}{\degrees} and \SI{+5}{\degrees}. Charged
particles create electron-hole pairs in the silicon-strip detectors inducing a
measurable current. A hit efficiency of at least \SI{99.7}{\percent} and a hit
resolution of \SI{55}{\mum} and better is achieved during data-taking in 2011
and 2012~\cite{LHCb-DP-2014-002}. The straw drift tubes of the outer tracker
are filled with a gas mixture of \SI{70}{\percent} Ar and \SI{30}{\percent}
CO$_2$, which gets ionised by passing particles. Timing measurements on how
long it takes for the electrons to reach the anode in the middle of the tube
allow to reconstruct the position of the hit.

In total, the tracking system provides a relative precision on the measurement
of the momentum that varies from \SI{0.5}{\percent} at low momentum to
\SI{0.8}{\percent} at \SI{100}{\gevc}~\cite{LHCb-DP-2014-002}.

Different track types are distinguished based on from which detector
components information is available. The category with the best mass, momentum
and vertex resolution is the long category. Long tracks originate in the
vertex detector and leave hits in all subsequent tracking stations. Long-lived
particles like \KS mesons might decay outside the VELO. If their tracks are
detected in the TT and, after passing the magnet, in the tracking stations
they are referred to as downstream. These two track categories are the only
ones used in the analyses described in this thesis. Furthermore, tracks might
be classified as VELO tracks, if they have only left hits in the VELO,
upstream tracks, if in addition the TT delivers information, or T tracks, if
they are solely reconstructed in the tracking stations downstream the magnet.


\subsection*{Particle identification}
\label{subsec:teilchenidentifizierung}

Apart from detecting the tracks and reconstructing their trajectory it is
important to estimate the identity of the particles. To distinguish pions from
kaons and protons two ring-imaging Cherenkov (RICH)
detectors~\cite{LHCb-DP-2012-003} are used, which are installed between VELO
and TT respectively downstream the tracking stations. The RICH detector
upstream of the magnet is filled with $\mathrm{C_4F_{10}}$ and during Run I
additionally with Aerogel. It is designed for particles with momenta in the
range \SIrange{1}{60}{\gevc}. Higher momentum particles are detected by the
second RICH detector, which is filled with $\mathrm{CF_4}$. When particles
pass through these materials with a speed greater than the speed of light in
the medium photons are emitted. The light is guided to hybrid photo detectors
by a system of mirrors (see \cref{fig:detector:rich}). From the radius of the
light cones and the measurement of the momentum a particle hypothesis can be
constructed.
\begin{figure}[!htb]
\centering
\includegraphics[width=0.50\textwidth]{04-Detector/figs/RICHSchema.jpg}
\caption{Schematic Aufbau der beiden RICH-Detektoren~\cite{Alves:2008zz}}
\label{fig:detector:rich}
\end{figure}
While photons and electrons are identified by an electromagnetic calorimeter
(ECAL), the energy of protons, neutrons and other long-lived hadrons is
measured in a hadronic calorimeter (HCAL). To suppress background from charged
and neutral pions there is a preshower (PS) respectively a Scintillator Pad
Detector (SPD) in front of the ECAL. The thickness of the lead in the PS is
chosen as a compromise between energy resolution and trigger
performance~\cite{Preshower}. The calorimeters are built of alternating layers
of metal and plastic. Polysterene molecules in the plastic are excited by
particle showers produced in the metal plates and produce ultraviolet light
whose amount is proportional to the energy of the incident particle. The least
interacting charged particles are muons, which are identified by five stations
of multi-wire proportional chambers (MWPC) filled by a gas mixture of $\mathrm{Ar,
CO_2}$ and $\mathrm{CF_4}$. Four of them are right at the end of the detector
downstream of the calorimeters and one is located in between the second RICH
and the calorimeter system. To stop the muons \SI{80}{cm} thick layers of iron
are put between the last four muon stations. Only muons with a momentum $p >
\SI{6}{\gevc}$ pass the whole detector. The detection of the muons is based on
ionisation of the gas in the MWPCs. An electric field accelerates the ions and
electrons. The emerging current is proportional to the energy of the muon.


%!TEX root = ../main.tex

\section{The LHCb trigger system}
\label{sec:detector:trigger}

%!TEX root = ../main.tex

\section{The LHCb software}
\label{sec:detector:software}

%!TEX root = ../main.tex

\subsection{Reconstruction (1 page)}
\label{sec:detector:software:reconstruction}

\todo{explain ghost tracks}
\subsubsection{Decay Tree Fitter (0.5 pages)}
\label{sec:dataanalysis:dtf}

%!TEX root = ../main.tex

\subsection{Stripping (1 page)}
\label{sec:detector:software:stripping}


%!TEX root = ../main.tex

\subsection{Monte Carlo simulation (1 page)}
\label{sec:detector:software:simulation}

%!TEX root = ../main.tex

\subsection{Flavour tagging}
\label{sec:detector:software:tagging}
\todo{FT as subsection of software or standalone section?}

For measurements of $\CP$ violation in $B$ decays it is essential to know the
initial flavour of the decaying $\bquark$ hadron candidate, \ie whether it
contained a $\bquark$ or a $\bquarkbar$ quark at production. When studying
decays of charged $B$ mesons the flavour at decay matches the production
flavour. Therefore, the flavour can unambiguously be determined from the
charges of the final state particles. Due to meson oscillations it is not as
trivial for neutral mesons. Instead, dedicated methods called flavour-tagging
algorithms are needed, which infer the initial flavour of a reconstructed
candidate from other particles inside the event. The $B$-factories \babar and
\belle have been operated at the $\FourS$ resonance, which dominantly decays into a
quantum-correlated pair of $B\Bbar$ mesons. Therefore, by analysing the decay
of the non-signal $B$ meson, \eg if it proceeds via a flavour-specific
process, the flavour of the signal $B$ meson at that time could be determined.
Such correlations are not present at proton-proton colliders like the \lhc,
where $\bquark$ quarks are dominantly produced in $\bbbar$ quark pairs via
gluon-gluon fusion. The LHCb collaboration has developed several
flavour-tagging algorithms, which can be classified as \emph{same-side} (SS)
and \emph{opposite-side} (OS) taggers. A schematic overview of all current taggers
that can be exploited to tag $\Bd$ mesons is given in
\cref{fig:detector:tagging:schematics}.
\begin{figure}[htb]
\centering
\input{04-Detector/tikz/tikz_ft_schematics}
\caption{Available tagging algorithms to tag \Bz mesons at the LHCb experiment.}
\label{fig:detector:tagging:schematics}
\end{figure}
By convention each flavour-tagging algorithm provides a flavour tag of $d =
\num{+1}$ if the tagger decides that it is more likely that the flavour of the
initial $B$ meson was a \Bz and of $d = \num{-1}$ if, based on the algorithm,
a \Bzb flavour is more likely. However, when no appropriate tagging particle
can be found for a reconstructed candidate, a tag decision of $d = 0$ is
assigned. The tag decisions are either based on the charge of a single
selected tagging particle or on the sign of the averaged charge of multiple
tagging particles. Besides the tag decision each tagger also provides a
prediction $\eta$ on the probability that the tag decision is wrong. This
mistag estimate $\eta$ takes values between \numlist{0;0.5}, where $\eta =
0$ means that there is no uncertainty on the tag decision and $\eta = 0.5$
basically corresponds to a random choice and is associated with $d = 0$. These
predictions are based on the outcome of multivariate classifiers, which
combine kinematic and geometric properties of the tagging particle as well as
information on the event.

The performance of flavour-tagging algorithms can be quantified by the tagging
efficiency \etag, which specifies for how many reconstructed candidates a tag
decision can be made, and by the true mistag probability \mistag. The relation
between the predicted and the true mistag probability $\omega(\eta)$ is
determined in calibration studies (see
\cref{sec:dataanalysis:taggingcalibration}). From the mistag probability the
tagging dilution $D = 1 - 2\omega$ can be derived, which indicates how much a
measured amplitude is reduced with respect to the physical amplitude due to
wrong tags. The product of the tagging efficiency and the squared tagging
dilution $\effeff = \etag D^2$ is called tagging power or effective tagging
efficiency. It is widely used as figure of merit for tagging algorithms as it
states the effective loss in statistics compared to a perfectly tagged sample.
Ideally, the tagging power is calculated on a per-candidate basis by summing
up the dilution of all $N$ signal candidates according to
\begin{equation}
	\effeff = \frac{1}{N}\sum_{i=1}^{N} (1 - 2\omega(\eta_i))^2\,,
\label{eq:detector:tagging:taggingpowerperevent}
\end{equation}
with $\omega = 0.5$ ($D=0$) for the untagged candidates.

While further information on the basic principles of LHCb's flavour-tagging
algorithms can be found in Refs.~\cite{LHCb-CONF-2011-003,Grabalosa:1456804},
a short description of the OS and SS taggers will be given in
\cref{sec:detector:software:tagging:ostagging,sec:detecor:software:tagging:sstagger}.

%!TEX root = ../main.tex

\subsubsection{Opposite-side flavour tagging}
\label{sec:detector:software:tagging:ostagging}

Opposite-side flavour-tagging algorithms~\cite{LHCb-PAPER-2011-027} infer the
flavour of the signal $B$ meson by studying the decay process of the second
$b$ hadron, which is produced from the same $\bbbar$ quark pair as the
reconstructed signal $B$ meson. Mistag probabilities are on the one hand
introduced by selecting a wrong tagging particle and on the other hand
intrinsically if the opposite-side $b$ hadron is neutral and has already mixed
at the time of decay.

In case of a semileptonic decay of the opposite-side $b$ hadron the charges of
the leptons are used by the OS electron (OS$e$) and OS muon (OS$\mu$) tagger
to determine the flavour. These two taggers provide relatively good mistag
estimates of around \SI{30}{\percent}, but have quite low tagging efficiencies
of around \SI{2}{\percent} (OS$e$) and \SI{5}{\percent} (OS$\mu$) for
charmonium respectively \SI{3.5}{\percent} (OS$e$) and \SI{8.5}{\percent}
(OS$\mu$) for open charm modes. The efficiency for muons is a factor
\numrange{2}{3} higher than for electrons due to the better reconstruction and
identification with the LHCb detector.

The OS kaon tagger selects kaons from a $\bquark\!\to\cquark\!\to\squark$
decay chain. Its tagging efficiency is around \SI{17}{\percent}
(\SI{21}{\percent}) for charmonium (open charm) modes at an average mistag
probability of approximately \SI{39}{\percent}.

The OS vertex charge tagger reconstructs the secondary vertex of the
opposite-side $b$ hadron and calculates the average charge of all tracks
associated to this vertex. The tagging efficiency and mistag probability are
comparable with the OS kaon tagger.

The recent OS charm tagger~\cite{LHCb-PAPER-2015-027} reconstructs charm
hadron candidates produced through $\bquark\!\to\cquark$ transitions of the
opposite-side $b$ hadron. The main contribution to the tagging power of the OS
charm tagger comes from partially reconstructed charm hadrons in $\Km\pip X$
final states and from \DzToKpi decays. The overall tagging efficiency of the
OS charm tagger is only \SIrange{3}{5}{\percent} with a mistag probability of
around \SI{35}{\percent}.
\todo{reference for tagging performance numbers?}

The four taggers described first are usually combined into an OS combination.
Since the release of the OS charm tagger a new OS combination is defined,
which is slightly better than the old one.

%!TEX root = ../main.tex

\section{Same-side flavour tagging (1 page)}
\label{sec:tagging:sstagger}