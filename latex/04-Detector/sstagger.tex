%!TEX root = ../main.tex

\subsubsection{Same-side flavour tagging}
\label{sec:detecor:software:tagging:sstagger}

Apart from $\bquark$ quarks being produced in \bbbar quark pairs, also
$\dquark$ quarks mainly stem from \ddbar quark pairs. In the hadronisation
process of the $B$ signal candidate charged pions and protons can be produced,
which contain the other quark from the \ddbar quark pair, and whose charge is
thereby correlated with the initial flavour of the reconstructed $B$ signal
candidate. Positive pions and antiprotons are associated with \Bz mesons, and
negative pions and protons with \Bzb mesons. Additionally, \Bd mesons can
originate from the decay $B^{*+}\!\to\Bd\pip$ of excited charged $B$ mesons.
Then, the charge of the associated pion again determines the initial flavour.
For quite some time, the only available same-side flavour-tagging algorithm for
\Bz mesons was a SS\pion tagger with a cut-based approach to select the
appropriate tagging pion. It has a tagging efficiency of around
\SI{15}{\percent} for $\JPsiX$ final states at an average mistag probability
of about \SI{42}{\percent}. Recently, an improved SS\pion tagger using a
boosted decision tree (BDT) to select the tagging pion and based on the very
same principles also a SS\proton tagger have been
developed\cite{CERN-THESIS-2015-040,LHCb-PAPER-2016-039}. These two
flavour-tagging algorithms select completely disjoint tagging particles
ensured by a requirement on the distance log-likelihood (DLL) between the pion
and the proton hypothesis of the tagging particle. The tagging efficiency of
the SS\pion tagger is very high with \SIrange{70}{75}{\percent}. The SS\proton
tagger provides non-zero tags for around \SI{35}{\percent} of all
reconstructed signal candidates, of which about \SI{80}{\percent} are also
tagged by the SS\pion (BDT). However, the high tagging efficiency comes along
with rather large average mistag probabilities of around \SI{45}{\percent}.
The response of the two BDT-based SS taggers is combined into a common SS
response.