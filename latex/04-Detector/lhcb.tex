%!TEX root = ../main.tex

\newpage

\section{The LHCb detector (4 pages)}
\label{sec:detector:lhcb}

The \lhcb detector, depicted in \cref{fig:detector:scheme}, is a single-arm
forward spectrometer. This means that the individual detector
components are sequentially arranged in the forward direction, starting from
the interaction point. The angular acceptance in the horizontal plane is
\SIrange{10}{300}{mrad} and in the vertical plane \SIrange{10}{250}{mrad}.
Thereby, a pseudorapidity range $2<\eta <5$ is covered.
\begin{figure}[htb]
\centering
\includegraphics[width=\textwidth]{04-Detector/figs/Lhcbdetektor.pdf}
\caption{Schematic view on the \lhcb detector\cite{Alves:2008zz}.}
\label{fig:detector:scheme}
\end{figure}
Instrumenting only this part of the space has been found to be an optimal
compromise between cost and output for \lhcb's desired physics program, which
is mainly to study particles containing \bquark or \cquark quarks. Simulations
of the correlation between the angular distribution of \bbbar quark pairs (see
\cref{fig:detector:bbbar}) show that the \bquark and \bquarkbar quarks are
mainly produced in quite small cones around the beam axis. Of course, half of
the \bbbar quark pairs are going backwards but about \SI{25}{\percent} of all
\bbbar quark pairs are inside the instrumented \SI{4.5}{\percent} of the whole
space.
\begin{figure}[htb]
\centering
\includegraphics[width=0.49\textwidth]{04-Detector/figs/bbbarcorrelation.pdf}
\includegraphics[width=0.49\textwidth]{04-Detector/figs/bbbaracceptance.pdf}
\caption{Correlation of angular acceptance (left) and pseudorapidity (right)
of \bbbar quark pairs~\cite{LHCb-Technical-Proposal}. The frequency of
produced \bbbar quark pairs is indicated by the bin content in the plot
showing the angular acceptance and by the colour code in the plot of the
pseudorapidities, where purple means low and red corresponds to high. The
region marked in red (left) respectively the region in the red square is
instrumented by the \lhcb detector.}
\label{fig:detector:bbbar}
\end{figure}

More details on the structure of the \lhcb detector can be found in
Ref.~\cite{Alves:2008zz} and an overview of the performance is given in
Ref.~\cite{LHCb-DP-2014-002}.

\subsection*{Vertexing and tracking}
\label{subsec:tracker}

The tracking system consists of several detector components, one of which is a
dipole magnet, which bends the tracks of charged particles with an integrated
magnetic field of \SI{4}{Tm}. To be able to study charge-dependent detection
asymmetries the polarity of the dipole magnet is reversed periodically
throughout data-taking. From the curvature radius it can be concluded on
the momentum of the track. To determine the curvature radius information from
tracking detector elements located upstream and downstream of the magnet are
needed. The $pp$ interaction region is surrounded by a silicon-strip vertex
locator (VELO)~\cite{LHCb-DP-2014-001}, which delivers the most precise
information on the position of the tracks and vertices due to being installed
very closely around the beam pipe. It is composed of 42 modules with R and
$\phi$ sensors, which measure the positions of the tracks in cylindrical
coordinates. Each module is a half disk (see \cref{fig:detector:velo}, which
can be pulled to a proximity of \SI{5}{\milli\metre} to the beam axis.
However, this is only done for stable beam conditions otherwise the modules
could be destroyed by the beam. To monitor the beam position a dedicated
detector component called Beam Conditions Monitor (BCM) is installed at two
locations in the vicinity of the beam. Via eight diamond sensors, which have
been proven to be very radiation-hard, each station determines the particle
flux and can trigger a beam dump in case of instabilities, which occur
especially at the injection of proton bunches. The importance of this system
is underlined by the fact that it has its own power supply and constantly
reports its status. If no information from the BCM is received a beam dump is
also initiated. The VELO achieves a single hit resolution of up to \SI{4}{\mum}
at an efficiency of more than \SI{99}{\percent}. The disks are arranged in a
way that guarantees that even at the outermost acceptance of \SI{300}{mrad}
tracks hit at least three VELO stations (see \cref{fig:detector:velo}).
\begin{figure}[htb]
\includegraphics[width=0.4\textwidth]{04-Detector/figs/VELO.jpg}
\includegraphics[width=0.59\textwidth]{04-Detector/figs/AnordnungVELOStationen.pdf}
\caption{View on a single VELO half disk (left) and arrangement of all VELO
modules~\cite{Alves:2008zz}.}
\label{fig:detector:velo}
\end{figure}
One of the main purposes of the VELO is to precisely determine the position of
the proton-proton interaction called primary vertex (PV) and the displaced
secondary decay vertices of long-lived particles like \Bd mesons. The impact
parameter (IP), which is the minimum distance of a track to a PV, is measured
depending on the transversal momentum \pT with a spatial resolution of
$(15+29/\pt)\mum$ (\pT given in units of \si{\gevc}).

Between VELO and dipole magnet there is another silicon-strip detector, the
tracker turicensis (TT). Like the three tracking stations located downstream
of the magnet, which are subdivided into an inner silicon-strip tracker and an
outer straw drift tube detector~\cite{LHCb-DP-2013-003}, it is built of four
layers. While the first and last layer are arranged vertically, the inner
layers are tilted by \SI{-5}{\degrees} and \SI{+5}{\degrees}. Charged
particles create electron-hole pairs in the silicon-strip detectors inducing a
measurable current. A hit efficiency of at least \SI{99.7}{\percent} and a hit
resolution of \SI{55}{\mum} and better is achieved during data-taking in 2011
and 2012~\cite{LHCb-DP-2014-002}. The straw drift tubes of the outer tracker
are filled with a gas mixture of \SI{70}{\percent} Ar and \SI{30}{\percent}
CO$_2$, which gets ionised by passing particles. Timing measurements on how
long it takes for the electrons to reach the anode in the middle of the tube
allow to reconstruct the position of the hit.

In total, the tracking system provides a relative precision on the measurement
of the momentum that varies from \SI{0.5}{\percent} at low momentum to
\SI{0.8}{\percent} at \SI{100}{\gevc}~\cite{LHCb-DP-2014-002}.

Different track types are distinguished based on from which detector
components information is available. The category with the best mass, momentum
and vertex resolution is the long category. Long tracks originate in the
vertex detector and leave hits in all subsequent tracking stations. Long-lived
particles like \KS mesons might decay outside the VELO. If their tracks are
detected in the TT and, after passing the magnet, in the tracking stations
they are referred to as downstream. These two track categories are the only
ones used in the analyses described in this thesis. Furthermore, tracks might
be classified as VELO tracks, if they have only left hits in the VELO,
upstream tracks, if in addition the TT delivers information, or T tracks, if
they are solely reconstructed in the tracking stations downstream the magnet.


\subsection*{Particle identification}
\label{subsec:teilchenidentifizierung}

Apart from detecting the tracks and reconstructing their trajectory it is
important to estimate the identity of the particles. To distinguish pions from
kaons and protons two ring-imaging Cherenkov (RICH)
detectors~\cite{LHCb-DP-2012-003} are used, which are installed between VELO
and TT respectively downstream the tracking stations. The RICH detector
upstream of the magnet is filled with $\mathrm{C_4F_{10}}$ and during Run I
additionally with Aerogel. It is designed for particles with momenta in the
range \SIrange{1}{60}{\gevc}. Higher momentum particles are detected by the
second RICH detector, which is filled with $\mathrm{CF_4}$. When particles
pass through these materials with a speed greater than the speed of light in
the medium photons are emitted. The light is guided to hybrid photo detectors
by a system of mirrors (see \cref{fig:detector:rich}). From the radius of the
light cones and the measurement of the momentum a particle hypothesis can be
constructed.
\begin{figure}[!htb]
\centering
\includegraphics[width=0.50\textwidth]{04-Detector/figs/RICHSchema.jpg}
\caption{Schematic Aufbau der beiden RICH-Detektoren~\cite{Alves:2008zz}}
\label{fig:detector:rich}
\end{figure}
While photons and electrons are identified by an electromagnetic calorimeter
(ECAL), the energy of protons, neutrons and other long-lived hadrons is
measured in a hadronic calorimeter (HCAL). To suppress background from charged
and neutral pions there is a preshower (PS) respectively a Scintillator Pad
Detector (SPD) in front of the ECAL. The thickness of the lead in the PS is
chosen as a compromise between energy resolution and trigger
performance~\cite{Preshower}. The calorimeters are built of alternating layers
of metal and plastic. Polysterene molecules in the plastic are excited by
particle showers produced in the metal plates and produce ultraviolet light
whose amount is proportional to the energy of the incident particle. The least
interacting charged particles are muons, which are identified by five stations
of multi-wire proportional chambers (MWPC) filled by a gas mixture of $\mathrm{Ar,
CO_2}$ and $\mathrm{CF_4}$. Four of them are right at the end of the detector
downstream of the calorimeters and one is located in between the second RICH
and the calorimeter system. To stop the muons \SI{80}{cm} thick layers of iron
are put between the last four muon stations. Only muons with a momentum $p >
\SI{6}{\gevc}$ pass the whole detector. The detection of the muons is based on
ionisation of the gas in the MWPCs. An electric field accelerates the ions and
electrons. The emerging current is proportional to the energy of the muon.
