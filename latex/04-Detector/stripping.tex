%!TEX root = ../main.tex

\subsection{Stripping}
\label{sec:detector:software:stripping}

Once tracking and PID are done the full decay chain can be fitted. However,
the data size after reconstruction is enormous. Right now, it is inevitable to
use a centralized selection called Stripping to handle it. In the \davinci
framework~\cite{DaVinci} stripping lines are defined, which basically are a
set of requirements that describe certain decay modes. Many selection steps
can be shared between various stripping lines, which saves a lot of computing
time. For example, there are minimal requirements for stable particles to
start from. In the individual stripping lines these can then be tightened.
Only data selected by a stripping line can be analysed offline by the users
and data campaigns are usually only performed as often as once per year. This
makes the stripping so important. In the stripping the OfflineVertexFitter
(OVF) is used for the analysis of the \BdToJPsiKS decays and the
LoKiVertexFitter (LVF) for the analysis of the \BdToDD decays. In order to
correctly comprise correlations and uncertainties on vertex positions,
particle momenta, flight distances, decay times, and invariant masses, the
\mbox{DecayTreeFitter} (DTF)~\cite{Hulsbergen:2005pu} can be used in the
reconstruction of decay chains. The decay time related observables in both
analyses covered in this thesis stem from a DTF fit where a constraint on the
production vertex of the \Bd mesons is applied using the knowledge about the
position of the primary vertex. The momenta and the invariant mass of the \Bd
meson are determined with a DTF fit in which additionally the invariant masses
of the daughter hadrons are constrained to their PDG masses.