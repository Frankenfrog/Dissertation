%!TEX root = ../main.tex

\subsection{Direct \texorpdfstring{$\CP$}{CP} violation}
\label{sec:cpviolation:types:direct}

Two different type of phases can contribute to decay amplitudes, weak phases
and strong phases. Weak phases can enter through the CKM matrix and take the
opposite sign for \Af and \Abarfbar. Strong phases typically appear in
scattering processes and originate from intermediate on-shell states. They
occur with the same sign in \Af and \Abarfbar. However, only phase differences
are physically meaningful, as the SM is a gauge-invariant theory and thus
absolute phases could be removed by a rotation of the system. So, at least two
terms with different weak and strong phases need to contribute to the decay
amplitudes to have an effect. The superposition of several contributions with
individual magnitudes $A_i$, weak phases $e^{i\phi_i}$ and strong phases
$e^{i\delta_i}$ leads to
\begin{align}
\begin{split}
	\Af = &\sum_i A_i e^{i(\delta_i+\phi_i)}\,,\\
	\Abarfbar = e^{2i(\xi_f-\xi_B)}&\sum_i A_i e^{i(\delta_i-\phi_i)}\,,
\end{split}
\end{align}
where $\xi_f$ and $\xi_B$ are arbitrary phases coming from the \CP
transformation on the \Bd meson and the final state, respectively. If the
final state $f$ is a \CP eigenstate, the term $e^{2i\xi_f} = \num{\pm1}$
represents the \CP eigenvalue. Direct \CP violation is present for
\begin{align}
	\left|\frac{\Abarfbar}{\Af}\right| = \left|\frac{\sum A_i e^{i(\delta_i-\phi_i)}}{\sum A_i e^{i(\delta_i+\phi_i)}}\right| \neq 1\,.
\end{align}

This type of \CP violation is observed in charmless two-body decays of neutral
$B$ mesons~\cite{Lees:2012mma,Duh:2012ie,LHCb-PAPER-2013-018}.