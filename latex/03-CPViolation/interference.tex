%!TEX root = ../main.tex

\subsection{\texorpdfstring{$\CP$}{CP} violation in the interference of decay and decay after mixing}
\label{sec:cpviolation:types:interference}

Even if there is no direct or indirect \CP violation, it is possible that \CP
violation occurs, in the interference between decay amplitudes with and
without mixing. However, the final state has to be a \CP eigenstate, \ie
accessible for decaying \Bd and \Bdb mesons. The definition from
\cref{eq:cpviolation:lambda} slightly changes to
\begin{align}
  \lambda = \eta_{\CP} \frac qp \frac{\Abarfbar}{\Af}\,.
\end{align}
Here, $\eta_{\CP}$ is the \CP eigenvalue of the final state
\begin{align}
  \CP \ket{f_{\CP}} = \ket{\fbar_{\CP}} = \eta_{\CP} \ket{f_{\CP}} = \num{\pm1} \ket{f_{\CP}}\,.
\end{align}
The condition that any deviation from unity for $\lambda$ indicates \CP
violation holds. Indirect \CP violation ($|q/p| \neq 1$) and direct \CP
violation ($|\Abarfbar/\Af| \neq 1$) affect the magnitude of $\lambda$, while
\CP violation in the interference is associated with
\begin{align}
  \mathcal{I}m\,\lambda \neq 0\,.
\label{eq:cpviolation:imlambda}
\end{align}

The time-dependent asymmetry
\begin{equation}
  {\mathcal A}(t) \equiv
    \frac{\Gamma(\Bzb(t) \to f_{\CP}) - \Gamma(\Bz(t) \to f_{\CP})}
         {\Gamma(\Bzb(t) \to f_{\CP}) + \Gamma(\Bz(t) \to f_{\CP})}
\label{eq:cpviolation:asymmetry}
\end{equation}
can be used to measure \CP violation in the interference of decay and decay
after mixing. It compares the decay rates of initial ($t = 0$) \Bdb and \Bd
mesons. Plugging in the expressions from
\cref{eq:cpviolation:complexdecayrates} and using
\begin{align}
\begin{split}
  |g_{\pm}(t)|^2 &= \frac{e^{-\Gamma\,t}}{2}\left[\cosh\frac{\DGd\,t}{2} \pm \cos(\dmd\,t)\right]\,,\\
  g_+^{\ast}(t)g_-(t) &= \frac{e^{-\Gamma\,t}}{2} \left[-\sinh\frac{\DGd\,t}{2} - i \sin(\dmd\,t)\right]\,,
\end{split}
\end{align}
$\mathcal{A}$ can be written as
\begin{align}
  \mathcal{A}(t) = \frac{2\,\mathcal{I}m\,\lambda \sin(\dmd\,t) - (1 - |\lambda|^2)\cos(\dmd\,t)}{(1 + |\lambda|^2)\cosh(\frac{\DGd\,t}{2}) + 2\,Re\,\lambda \sinh(\frac{\DGd\,t}{2})}\,.
\end{align}
It is apparent that this asymmetry only vanishes if $|\lambda| \neq 1$ (direct
or indirect \CP violation) or if $\lambda$ has an imaginary part, which is the
condition for \CP violation stated in \cref{eq:cpviolation:imlambda}. Defining
\begin{align}
  \Sf = \frac{2\,\mathcal{I}m\,\lambda}{1 + |\lambda|^2} \quad \text{and} \quad \Cf = \frac{1 - |\lambda|^2}{1 + |\lambda|^2}\,,
\label{eq:cpviolation:SfCf_general}
\end{align}
and neglecting $\DGd$, the time-dependent asymmetry simplifies to
\begin{equation}
  \mathcal{A}(t) = \Sf\sin\dmd\,t - \Cf\cos\dmd\,t\,.
\label{eq:cpviolation:simpleasymmetry}
\end{equation}
