%!TEX root = ../main.tex

\subsection{Figures of merit}
\label{sec:dataanalysis:selection:fom}

One of the main questions when performing a selection is, how the
requirements, be it a sequence of cuts or the classifier of a multivariate
method, should be optimized. In an ideal world, one would find a selection
that keeps all signal candidates and removes all background contributions. But
this is unrealistic. Instead, suitable figures of merit have to be defined,
whose optimum should correspond with the ideal selection.  In a measurement of
\CP violation the goal is to obtain the optimal precision. Therefore, the
sensitivity on the \CP observables itself seems to be the best figure of
merit. However, there are some caveats. On the one hand there is usually more
than one observable describing the \CP violation and a strategy needs to be
found how the sensitivities of the different \CP observables can be combined
into a single figure of merit. On the other hand the absolute uncertainty
might depend on the central value and thus small values for the \CP
observables could be preferred. Finally, the full time-dependent fit has to be
performed, which is often very complex and takes a long time until it
converges. For all these reasons, alternative figure of merits are developed.
A very simple one is the pure signal efficiency $\epsilon_S$. Under the
assumptions that a high signal yield is more important than a low background
contamination and that the selection requirements are effectively suppressing
background contributions, a possible selection strategy is to judge
requirements only by their signal efficiency. In the \BdToJPsiKS analysis this
approach is chosen using the requirement that the individual cuts have to be
at least \SI{99}{\percent} signal efficient. Including the background yield to
the definition of the figure of merit should make it easier to find the
optimal cut point, as the background contamination partly influences the
achievable sensitivity. There are several possibilities how the signal yield
$S$ and the background yield $B$ can be combined: The figure of merit
\begin{align}
	Q_1 = \frac{S}{S + B}\,,
\end{align}
called purity, describes the fraction of signal candidates. It is limited to
unity, which is reached when no background candidates are left over. Using
merely the purity does not necessarily lead to an optimal selection, \eg when
lots of signal candidates would be thrown away in order to remove one last
remaining background candidate. Instead, the signal significance
\begin{align}
	Q_2 = \frac{S}{\sqrt{S + B}}\,,
\end{align}
which states by how many standard deviations the signal yield exceeds zero, is
widely used. To enhance the impact of a high signal yield, the signal
significance can be multiplied with the purity:
\begin{align}
	Q_3 = \frac{S^2}{\sqrt{(S + B)^3}}\,.
\end{align}
For decay-time-independent studies, like determinations of branching ratios,
the figure of merit $Q_3$ is appropriate. In searches for very rare decay
modes, where a certain significance expressed in number of standard deviations
$a$ is desired, Punzi's figure of merit~\cite{Punzi:2003bu}
\begin{align}
	Q_4 = \frac{\epsilon_S}{a/2 + \sqrt{B}}\,,
\end{align}
is often used.

However, for decay-time-dependent studies the figure of merit can be improved
through an extension that takes into account that the contribution of a signal
candidate to the sensitivity on \CP observables depends on its decay time. For
instance, the sine term in the decay-time-dependent asymmetry (see
\cref{eq:cpviolation:simpleasymmetry} has its maximum at $t =
\frac{\pi}{2\,\dmd} \approx \SI{3}{\ps}$. Thus, signal candidates with such
decay times have a higher impact. This can be expressed by adding up the
square of the differentiation of the log-likelihood with respect to the
parameter of interest, here $\Sf$, of all $N_S$ signal candidates:
\begin{align}
	Q_5 = \sum_{i=1}^{N_S} \left[\frac{\sin(\dmd\,t_i)}{1 + d_i\,S_f\,\sin(\dmd\,t_i)}\right]^2\,,
\end{align}
with $d_i = \num{+1}$ for \Bd and $d_i = \num{-1}$ for \Bdb. Additionally, it
has to be considered that the decay time distribution of background candidates
usually follows an exponential function with an effective lifetime that is
significantly smaller than the \Bd signal lifetime. This can be incorporated
by using a decay-time-dependent purity $f_i(t)$:
\begin{align}
	Q_6 = \sum_{i=1}^{N} f_i^2 \cdot \left[\frac{\sin(\dmd\,t_i)}{1 + d_i\,S_f\,\sin(\dmd\,t_i)}\right]^2\,.
\end{align}
Here, the sum is built over all $N$ candidates, signal and background. The
purity can be determined via a fit to the invariant mass distribution or more
specifically via sWeights (see \cref{sec:dataanalysis:selection:splot}).
Moreover, the influence of further experimentally induced dilutions can be
added, \eg from the flavour tagging (see \cref{sec:detector:software:tagging})
or the decay time resolution (see \cref{sec:dataanalysis::resolution}). Such a
generalized figure of merit for decay-time-dependent measurements of
\CP violation is derived in Ref.~\cite{FOM}.