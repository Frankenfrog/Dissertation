%!TEX root = ../main.tex

\subsection{Calibration using \texorpdfstring{$\JpsiX$}{JpsiX} channels (1 page)}
\label{sec:dataanalysis:taggingcalibration:jpsixcalibration}

For the charged $\BuToJPsiK$ decay, which is used to determine the calibration
of the OS tagging combination for the analysis of \BdToJPsiKS decays, a
comparison of the charge of the kaon with the tag decision directly tells if
the tag decision is correct or not. Binning the sample in terms of the mistag
estimate a \chisq fit using
\cref{eq:dataanalysis:taggingcalibration:generalfunction} is performed to the
($\eta$, $\omega)$ pairs and reveals the following results for the calibration
parameters:
\begin{equation}
\begin{aligned}
  p_0^{\text{OS}} &= 0.3815 \pm 0.0011 \text{\,(stat.)}
                            \pm 0.0016 \text{\,(syst.)} \, ,\\
  p_1^{\text{OS}} &= 0.978\phantom{0} \pm 0.012\phantom{0} \text{\,(stat.)}
                                      \pm 0.009\phantom{0} \text{\,(syst.)} \, ,\\
  \langle\eta^{\text{OS}}\rangle &= 0.3786 \, .
\end{aligned}
\label{eq:dataanalysis:taggingcalibration:jpsixcalibration:os_calibration_pars}
\end{equation}
Repeating the same fit with a split of the sample into the initial flavours
gives access to the asymmetry parameters $\Delta p_0$ and $\Delta p_1$, which
are determined to be
\begin{equation}
\begin{split}
  \Delta p_0^{\text{OS}} &= 0.0148 \pm 0.0016 \text{\,(stat.)} \pm  0.0008 \text{\,(syst.)} \, , \\
  \Delta p_1^{\text{OS}} &= 0.070\phantom{0} \pm 0.018\phantom{0} \text{\,(stat.)} \pm 0.004\phantom{0} \text{\,(syst.)} \,.
\end{split}
\label{eq:dataanalysis:taggingcalibration:jpsixcalibration:os_calibration_asymmetries}
\end{equation}
A cross-check of the calibration in a control sample of $\BdToJPsiKstar$
decays confirms the validity of transferring the calibration from $\Bu$ to
$\Bd$ decays.

Despite the advantages of $\BuToJPsiK$ as control channel (charged decay mode,
very high statistics), for the calibration of the cut-based SS\pion tagging
algorithm $\BdToJPsiKstar$ decays are used because differences in the
composition of the fragmentation products in the $\Bp$ and $\Bz$ hadronisation
are expected. Like for \BdToDsD a time-dependent mixing analysis is needed.
Here, a two dimensional fit to both the reconstructed decay time and mass
distributions is performed. From a simultaneous fit in five evenly filled bins
of the mistag estimate $\eta$ the calibration parameters for the SS\pion
tagging algorithm are determined to be
\begin{equation}
\begin{aligned}
  p_0^{\text{SS\pion}}        &=& \phantom{+}0.4232  &\pm\, 0.0029 \text{\,(stat.)} &\pm\, 0.0028 \text{\,(syst.)} \, ,\\
  p_1^{\text{SS\pion}}        &=& \phantom{+}1.011\phantom{0}   &\pm\, 0.064\phantom{0} \text{\,(stat.)} &\pm\, 0.031\phantom{0} \text{\,(syst.)} \, ,\\
  \Delta p_0^{\text{SS\pion}} &=& -0.0026 &\pm\, 0.0043 \text{\,(stat.)} &\pm\, 0.0027 \text{\,(syst.)} \, , \\
  \Delta p_1^{\text{SS\pion}} &=& -0.171\phantom{0}   &\pm\, 0.096\phantom{0} \text{\,(stat.)} &\pm\, 0.04\phantom{00} \text{\,(syst.)} \, , \\
  \langle\eta^{\text{SS\pion}}\rangle        &=& \phantom{+}0.425\phantom{0} & .
\end{aligned}
\label{eq:dataanalysis:taggingcalibration:jpsixcalibration:ss_calibration_pars}
\end{equation}

The systematic uncertainties for both calibrations cover two different types,
one for intrinsic uncertainties and one for the kinematic differences between
the control mode and the signal decay \BdToJPsiKS. The effect of these two
sources is of the same order.

The combined effective tagging efficiency is \SI{3.02\pm0.05}{\percent}, which
is composed of a tagging efficiency of \etag = \SI{36.54\pm0.14}{\percent} and
an effective mistag probability of $\mistag_{\textrm{eff}} =
\SI{35.62\pm0.12}{\percent}$. The major contribution comes from the OS tagging
combination, which has an inclusive tagging power of
\SI{2.63\pm0.04}{\percent}. The cut-based SS\pion tagging algorithm adds
\SI{0.376\pm0.024}{\percent}.