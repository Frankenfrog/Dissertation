%!TEX root = ../main.tex

\section{Flavour tagging calibration}
\label{sec:dataanalysis:taggingcalibration}

The output of the flavour tagging algorithms $\eta$ needs to be calibrated to
ensure that it reflects the true mistag probability $\omega$. Usually, a
linear function
\begin{align}
	\omega(\eta) = p_0 + p_1 (\eta - \langle\eta\rangle)
\label{eq:dataanalysis:taggingcalibration:generalfunction}
\end{align}
is chosen. Shifting the function by the average mistag estimate
$\langle\eta\rangle$ reduces the correlation between the calibration
parameters $p_0$ and $p_1$, which in case of a perfect calibration leads to
$p_0 = \langle\eta\rangle$ and $p_1 = 1$.

Due to different interaction rates of the tagging particles with the detector
material or detection asymmetries the performance of the flavour tagging
algorithms can be dependent on the initial flavour. This behaviour is quite
unfortunate in the measurement of \CP violation as it can dilute or enhance
the observed asymmetry. To account for these tagging asymmetries separate
parametrisations for the flavour tagging calibrations of initial \Bd and \Bzb
are implemented:
\begin{equation}
\begin{split}
  \omega^{\Bz }(\eta) = p_0^{\Bz } + p_1^{\Bz }\left(\eta - \langle\eta\rangle\right) \, , \\
  \omega^{\Bzb}(\eta) = p_0^{\Bzb} + p_1^{\Bzb}\left(\eta - \langle\eta\rangle\right) \, .
\end{split}
\label{eq:dataanalysis:taggingcalibration:individual_parametrisations}
\end{equation}
Equivalently, the calibration parameters for \Bd and \Bzb can be related
through their mean and their difference:
\begin{equation}
  p_i        = \frac{p_i^{\Bz } + p_i^{\Bzb}}{2} \, , \quad
  \Delta p_i = p_i^{\Bz } - p_i^{\Bzb} \, , \quad
  \text{with }i = 0,1 \,.
\label{eq:dataanalysis:taggingcalibration:individual_pis}
\end{equation}
The asymmetry of the mistags can then be written as
\begin{align}
	\Delta\omega(\eta) = \Delta p_0 + \Delta p_1 (\eta - \langle\eta\rangle)\,.
\label{eq:dataanalysis:taggingcalibration:mistagasymmetry}
\end{align}

For the flavour tagging calibration it is beneficial to use flavour specific
decay channels that are kinematically similar to the signal channel.
Additionally, the selection between the two channels should be as close as
possible. This allows to transfer the calibration results from the control to
the signal channel without assigning large systematic uncertainties. On the
other hand, the control channel should ideally be a mode with high statistics
to reduce the statistical uncertainties on the flavour tagging calibration
parameters. A good compromise between these two requirements is found by
choosing \BdToDsD as calibration mode for \BdToDD, while for \BdToJPsiKS the
OS tagging combination and the cut-based SS\pion tagging algorithm are
calibrated with \BuToJPsiK and \BdToJPsiKstar decays, respectively.

%!TEX root = ../main.tex

\subsection[Calibration using \texorpdfstring{$\BdToDsD$}{Bd2DsD} (2 pages)]{Calibration using \texorpdfstring{$\BdToDsD$}{Bd2DsD}}
\label{sec:dataanalysis:taggingcalibration:dsdcalibration}

\begin{table}[!htb]
\caption{Flavour tagging calibration parameters from \BdToDsD. The first
uncertainty is statistical and the second accounts for systematic
uncertainties.}
\label{tab:dataanalysis:taggingcalibration:dsdcalibration}
\centering
\begin{tabular}{lr@{$\,\pm\,$}l@{$\,\pm\,$}lr@{$\,\pm\,$}l@{$\,\pm\,$}l}
  \toprule
  Parameter           & \multicolumn{3}{c}{OS}   & \multicolumn{3}{c}{SS} \\
  \midrule
  $p_{1}$               & 1.069   & 0.072  & 0.01  & 0.842   & 0.090  & 0.01  \\
  $p_{0}$               & 0.3691  & 0.0080 & 0.01  & 0.4296  & 0.0060 & 0.009 \\
  $\langle \eta\rangle$ & \multicolumn{3}{c}{0.3627} & \multicolumn{3}{c}{0.4282} \\
  $\Delta p_{1}$        & 0.03    & 0.11   & 0.03  & 0.07    & 0.13   & 0.05  \\
  $\Delta p_{0}$        & 0.009   & 0.012  & 0.001 & -0.0065 & 0.0087 & 0.001 \\
  \bottomrule
\end{tabular}
\end{table}

%!TEX root = ../main.tex

\subsection{Calibration using \texorpdfstring{$\JPsiX$}{JpsiX} channels}
\label{sec:dataanalysis:taggingcalibration:jpsixcalibration}

For the charged $\BuToJPsiK$ decay, which is used to determine the calibration
of the OS tagging combination for the analysis of \BdToJPsiKS decays, a
comparison of the charge of the kaon with the tag decision directly tells if
the tag decision is correct or not. Binning the sample in terms of the mistag
estimate a \chisq fit using
\cref{eq:dataanalysis:taggingcalibration:generalfunction} is performed to the
($\eta$, $\omega)$ pairs and reveals the following results for the calibration
parameters:
\begin{equation}
\begin{aligned}
  p_0^{\text{OS}} &= 0.3815 \pm 0.0011 \text{\,(stat.)}
                            \pm 0.0016 \text{\,(syst.)} \, ,\\
  p_1^{\text{OS}} &= 0.978\phantom{0} \pm 0.012\phantom{0} \text{\,(stat.)}
                                      \pm 0.009\phantom{0} \text{\,(syst.)} \, ,\\
  \langle\eta^{\text{OS}}\rangle &= 0.3786 \, .
\end{aligned}
\label{eq:dataanalysis:taggingcalibration:jpsixcalibration:os_calibration_pars}
\end{equation}
Repeating the same fit with a split of the sample into the initial flavours
gives access to the asymmetry parameters $\Delta p_0$ and $\Delta p_1$, which
are determined to be
\begin{equation}
\begin{split}
  \Delta p_0^{\text{OS}} &= 0.0148 \pm 0.0016 \text{\,(stat.)} \pm  0.0008 \text{\,(syst.)} \, , \\
  \Delta p_1^{\text{OS}} &= 0.070\phantom{0} \pm 0.018\phantom{0} \text{\,(stat.)} \pm 0.004\phantom{0} \text{\,(syst.)} \,.
\end{split}
\label{eq:dataanalysis:taggingcalibration:jpsixcalibration:os_calibration_asymmetries}
\end{equation}
A cross-check of the calibration in a control sample of $\BdToJPsiKstar$
decays confirms the validity of transferring the calibration from $\Bu$ to
$\Bd$ decays.

Despite the advantages of $\BuToJPsiK$ as control channel (charged decay mode,
very high statistics), for the calibration of the cut-based SS\pion tagging
algorithm \mbox{$\BdToJPsiKstar$} decays are used because differences in the
composition of the fragmentation products in the $\Bp$ and $\Bz$ hadronisation
are expected. Like for \BdToDsD a time-dependent mixing analysis is needed.
Here, a two dimensional fit to both the reconstructed decay time and mass
distributions is performed. From a simultaneous fit in five evenly filled bins
of the mistag estimate $\eta$ the calibration parameters for the SS\pion
tagging algorithm are determined to be
\begin{equation}
\begin{aligned}
  p_0^{\text{SS\pion}}        &=& \phantom{+}0.4232  &\pm\, 0.0029 \text{\,(stat.)} &\pm\, 0.0028 \text{\,(syst.)} \, ,\\
  p_1^{\text{SS\pion}}        &=& \phantom{+}1.011\phantom{0}   &\pm\, 0.064\phantom{0} \text{\,(stat.)} &\pm\, 0.031\phantom{0} \text{\,(syst.)} \, ,\\
  \Delta p_0^{\text{SS\pion}} &=& -0.0026 &\pm\, 0.0043 \text{\,(stat.)} &\pm\, 0.0027 \text{\,(syst.)} \, , \\
  \Delta p_1^{\text{SS\pion}} &=& -0.171\phantom{0}   &\pm\, 0.096\phantom{0} \text{\,(stat.)} &\pm\, 0.04\phantom{00} \text{\,(syst.)} \, , \\
  \langle\eta^{\text{SS\pion}}\rangle        &=& \phantom{+}0.425\phantom{0} & .
\end{aligned}
\label{eq:dataanalysis:taggingcalibration:jpsixcalibration:ss_calibration_pars}
\end{equation}

The systematic uncertainties for both calibrations cover two different types,
one for intrinsic uncertainties and one for the kinematic differences between
the control mode and the signal decay \BdToJPsiKS. The effect of these two
sources is of the same order.

The combined effective tagging efficiency is \SI{3.02\pm0.05}{\percent}, which
is composed of a tagging efficiency of \etag = \SI{36.54\pm0.14}{\percent} and
an effective mistag probability of $\mistag_{\textrm{eff}} =
\SI{35.62\pm0.12}{\percent}$. The major contribution comes from the OS tagging
combination, which has an inclusive tagging power of
\SI{2.63\pm0.04}{\percent}. The cut-based SS\pion tagging algorithm adds
\SI{0.376\pm0.024}{\percent}.