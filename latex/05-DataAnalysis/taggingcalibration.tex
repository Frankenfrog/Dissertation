%!TEX root = ../main.tex

\section{Flavour-tagging calibration}
\label{sec:dataanalysis:taggingcalibration}

The output of the flavour-tagging algorithms $\eta$ needs to be calibrated to
ensure that it reflects the true mistag probability $\omega$. Usually, a
linear function
\begin{align}
	\omega(\eta) = p_0 + p_1 (\eta - \langle\eta\rangle)
\label{eq:dataanalysis:taggingcalibration:generalfunction}
\end{align}
is chosen. Shifting the function by the average mistag estimate
$\langle\eta\rangle$ reduces the correlation between the calibration
parameters $p_0$ and $p_1$, which in case of a perfect calibration are
$p_0 = \langle\eta\rangle$ and $p_1 = 1$.

Due to different interaction rates of the tagging particles with the detector
material or detection asymmetries the performance of the flavour-tagging
algorithms can be dependent on the initial flavour. This behaviour is quite
unfortunate in the measurement of \CP violation as it can dilute or enhance
the observed asymmetry. To account for these tagging asymmetries separate
parametrisations for the flavour-tagging calibrations of initial \Bd and \Bzb
are implemented:
\begin{equation}
\begin{split}
  \omega^{\Bz }(\eta) = p_0^{\Bz } + p_1^{\Bz }\left(\eta - \langle\eta\rangle\right) \, , \\
  \omega^{\Bzb}(\eta) = p_0^{\Bzb} + p_1^{\Bzb}\left(\eta - \langle\eta\rangle\right) \, .
\end{split}
\label{eq:dataanalysis:taggingcalibration:individual_parametrisations}
\end{equation}
Equivalently, the calibration parameters for \Bd and \Bzb can be related
through their mean and their difference:
\begin{equation}
  p_i        = \frac{p_i^{\Bz } + p_i^{\Bzb}}{2} \, , \quad
  \Delta p_i = p_i^{\Bz } - p_i^{\Bzb} \, , \quad
  \text{with }i = 0,1 \,.
\label{eq:dataanalysis:taggingcalibration:individual_pis}
\end{equation}
The asymmetry of the mistags can then be written as
\begin{align}
	\Delta\omega(\eta) = \Delta p_0 + \Delta p_1 (\eta - \langle\eta\rangle)\,.
\label{eq:dataanalysis:taggingcalibration:mistagasymmetry}
\end{align}

For the flavour-tagging calibration it is beneficial to use flavour-specific
decay channels that are kinematically similar to the signal channel.
Additionally, the selection between the two channels should be as close as
possible. This allows to transfer the calibration results from the control to
the signal channel without assigning large systematic uncertainties. On the
other hand, the control channel should ideally be a mode with high statistics
to reduce the statistical uncertainties on the flavour-tagging calibration
parameters. A good compromise between these two requirements is found by
choosing \BdToDsD as calibration mode for \BdToDD, while for \BdToJPsiKS the
OS tagging combination and the cut-based SS\pion tagging algorithm are
calibrated with \BuToJPsiK and \BdToJPsiKstar decays, respectively.

%!TEX root = ../main.tex

\subsection[Calibration using \texorpdfstring{$\BdToDsD$}{Bd2DsD} (2 pages)]{Calibration using \texorpdfstring{$\BdToDsD$}{Bd2DsD}}
\label{sec:dataanalysis:taggingcalibration:dsdcalibration}

\begin{table}[!htb]
\caption{Flavour tagging calibration parameters from \BdToDsD. The first
uncertainty is statistical and the second accounts for systematic
uncertainties.}
\label{tab:dataanalysis:taggingcalibration:dsdcalibration}
\centering
\begin{tabular}{lr@{$\,\pm\,$}l@{$\,\pm\,$}lr@{$\,\pm\,$}l@{$\,\pm\,$}l}
  \toprule
  Parameter           & \multicolumn{3}{c}{OS}   & \multicolumn{3}{c}{SS} \\
  \midrule
  $p_{1}$               & 1.069   & 0.072  & 0.01  & 0.842   & 0.090  & 0.01  \\
  $p_{0}$               & 0.3691  & 0.0080 & 0.01  & 0.4296  & 0.0060 & 0.009 \\
  $\langle \eta\rangle$ & \multicolumn{3}{c}{0.3627} & \multicolumn{3}{c}{0.4282} \\
  $\Delta p_{1}$        & 0.03    & 0.11   & 0.03  & 0.07    & 0.13   & 0.05  \\
  $\Delta p_{0}$        & 0.009   & 0.012  & 0.001 & -0.0065 & 0.0087 & 0.001 \\
  \bottomrule
\end{tabular}
\end{table}

%!TEX root = ../main.tex

\subsection{Calibration using \texorpdfstring{$\JpsiX$}{JpsiX} channels (2 pages)}
\label{sec:tagging:calibration:jpsixcalibration}