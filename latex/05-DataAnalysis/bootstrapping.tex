%!TEX root = ../main.tex

\section{Bootstrapping method}
\label{sec:dataanalysis:bootstrapping}

The bootstrapping method (see \eg Ref.~\cite{Behnke:2013pga}) is a frequentist
model-independent approach to estimate confidence level intervals. Toy data
samples are produced by drawing events from the nominal data sample, with
replacement, until the statistics matches the number of candidates of the
nominal data sample. This means that the same event can be drawn multiple
times. The bootstrapping method can easily be used for multiple dimensions.
Thus, it maintains correlations between the observables without any
assumptions or model dependencies. Therefore, bootstrapped samples serve as
perfect representations of the original data. Calculations or fits that are
originally done once on the nominal data sample can be repeated several times.
So, reliable estimates for standard deviations or confidence level intervals
can be derived.
