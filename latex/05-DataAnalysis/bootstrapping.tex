%!TEX root = ../main.tex

\section{Bootstrapping method (0.5 pages)}
\label{sec:dataanalysis:bootstrapping}

The bootstrapping method (see \eg Ref.~\cite{Behnke:2013pga}) is a frequentist
model-independent approach to estimate confidence level intervals. Toy data
samples are produced by drawing events from the nominal data sample, with
replacement, until the statistics matches the number of candidates of the
nominal data sample. This means that the same event can be drawn multiple
times. The toy data samples serve as test samples on which all calculations or
fits that are done on the nominal data sample can be repeated several times.
\todo{extend bootstrapping, advantages, ...}