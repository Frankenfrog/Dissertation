%!TEX root = ../main.tex

\section{Spline interpolation}
\label{sec:dataanalysis:splines}

In many cases phenomenological models are an efficient way of describing
shapes, \eg when parametrising acceptances, which are typically influenced by
more effects than could realistically be analysed separately. Interpolating
cubic splines, which are piecewise defined polynomials of degree three, are an
useful implementation~\cite{Splines}. They are parametrised by a set of knots
and coefficients at these positions and can be written as the sum over base
splines. The first and second derivatives are continuous throughout the
domain. The choice of the number and positions of the knots determines how
accurate the given shape can be described.

In this thesis cubic splines are used to parametrise the shape of mistag
distributions and to transfer a histogram of the decay-time-dependent
efficiency into an unbinned, analytically integrable representation in the
\BdToJPsiKS analysis, and to model the deviation of the decay time
distribution from a pure exponential distribution in the \BdToDD analysis.