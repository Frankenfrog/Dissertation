%!TEX root = ../main.tex

\section{Selection}
\label{sec:dataanalysis:selection}

When reconstructing decays in a hadronic environment, like at \lhcb, it is
inevitable that some of the candidates do not stem from the signal decay chain
that one wants to analyse. In fact, most of the reconstructed candidates are
usually built from random combinations of tracks that have no common physical
origin. But especially when searching for a very rare signal it is better to
be careful and at first stage rather keep an event than throwing it away. Of
course, these background candidates allocate disk space and cost computational
resources. Furthermore, the sensitivity of a measurement suffers from
background contamination. Therefore, a selection needs to be developed that
separates signal from background candidates. The simplest selection is a
requirement of the type $a < b$. Several of these cuts can be combined to a
sequence but for each variable a maximum of two requirements can be applied,
\ie a minimal and a maximal value can be defined. This means that out of the
whole phase space only a hyperrectangle is selected. But the simplicity is
also a strength of the cut-based selection. It is very fast and the selection
requirements can easily be understood and connected to event or particle
properties. Additionally, the efficiency of the requirements can be determined
individually. To account for dependencies between the variables a grid search
can be performed, in which the optimal cut values are determined recursively.
However, a cut-based selection often leads to suboptimal selection
performances.