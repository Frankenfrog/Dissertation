%!TEX root = ../main.tex
\section{Decay time resolution}
\label{sec:dataanalysis:resolution}

Uncertainties in the determination of the position of vertices and in the
measurement of momenta (although thanks to the VELO (see
\cref{sec:detector:lhcb}) pretty accurate  at $\lhcb$) lead to a finite decay
time resolution $\sigma$, which dilutes the observed $\CP$ asymmetry by a
factor
\begin{align}
  \mathcal{D} = e^{\frac{-\dmd^2\,\sigma^2}{2}} \, .
\label{eq:dataanalysis:resolution:dilution}
\end{align}
This formula is the special case for a Gaussian resolution model with width
$\sigma$. The general formula is derived in
Ref.~\cite{ResolutionDilutionFactor}. For $\Bd$ mesons the dilution induced by
the decay time resolution has only minor influence on the measurement of $\CP$
observables because the oscillation frequency of $\Bd$ mesons $\dmd =
\SI{0.5064\pm0.0019}{\hbar\invps}$~\cite{HFAG} is quite low. Even for a decay time
resolution of \SI{100}{\fs}, which would be almost two times larger than what
is usually found in analyses performed by \lhcb, the dilution factor is
greater than \SI{99}{\percent}.