%!TEX root = ../main.tex

\subsection{Unfolding data distributions using sWeights (1 page)}
\label{sec:dataanalysis:selection:splot}

The shape of invariant mass and decay time distributions of signal decays is
mostly defined by some theoretical considerations that eventually need to be
modified to account for experimental effects, like resolutions or acceptances.
However, for background contributions especially the shape of the decay time
distribution is usually difficult to motivate from first principles.
Therefore, it is beneficial to statistically remove their contribution. The
\sPlot technique~\cite{Pivk:2004ty} achieves this goal by unfolding the data
distributions using weights calculated from yields of an extended maximum
likelihood fit (see \cref{sec:dataanalysis:maximumlikelihood}). This fit is
typically performed for the invariant mass distribution where reliable
parametrisations for signal and background contributions can quite easily be
found. Based on the fit results sWeights according to
\begin{align}
	\SPlotweight(x_i) = \frac{\sum_{j = 1}^{N_s}\textbf{V}_{\text{n}j}f_j(x_i)}{\sum_{k = 1}^{N_s} N_k f_k(x_i)}
\label{eq:dataanalysis:selection:splot:weightdefinition}
\end{align}
can be calculated for each candidate. Herein, the indices $j$ and $k$ sum over
the $N_s$ categories described in the PDF $f$. The matrix \textbf{V} contains
the covariances between the yields $N$ and needs to be determined from an
individual fit in which all other floating parameters are fixed. The sWeights
fulfil the condition that their sum over one category returns the
corresponding fitted yield. The sWeights can be applied to other observables
if they are uncorrelated with the one used to obtain the sWeights. In
sweighted histograms the uncertainty on the bin content of bin $i$ is given by
\begin{align}
	\sigma(i) = \sqrt{\sum_{e \subset i} (\SPlotweight)^2}\,.
\label{eq:dataanalysis:selection:splot:weighterror}
\end{align}