%!TEX root = ../main.tex

\subsection{Multivariate selection (1 page)}
\label{sec:dataanalysis:selection:bdt}

Machine learning algorithms are more and more entering the field of particle
physics. These multivariate techniques improve the possibility to separate
signal from background contributions as they make use of correlations between
input variables. Software packages, like TMVA~\cite{Hocker:2007ht} or
scikit-learn~\cite{scikit-learn}, provide implementations of these algorithms.

The simplest classifier is a decision tree, which splits the phase space
according to repeated decisions on certain variables. Starting from a root
node the variable and cut value is determined where the best separation can be
achieved according to a criterion like the Gini index~\cite{giniindex}, $p
\cdot (1-p)$ with $p$ being the purity. At each following node the cut value
and even the variable used for the separation can be chosen dependent on the
previous decision. The depth of the tree, \ie the maximal number of
consecutive decisions, is tunable. When a stop criterion is matched, \eg the
ratio of candidates reaching a node falls below a predefined threshold, no
further decisions are applied. The tree is trained with a set of labelled
data. Each final leaf is classified as signal or background depending on the
class of the majority of training events ending in that leaf. An improvement
can be achieved by combining several decision trees to a forest. In this case,
the classification of an event follows a majority vote of the individual
trees. Additionally, the impact of the training events is altered by applying
weights using a procedure called boosting. However, the ability to easily
interpret the training result vanishes for a Boosted Decision Tree
(BDT)~\cite{Breiman,Roe}. There are several possibilities how the boosting is
realised. In the AdaBoost method~\cite{AdaBoost} events that are misidentified
in the previous tree are weighted by
\begin{align}
	\alpha = \frac{1 - \eps}{\eps}\,,
\end{align}
with $\eps < \num{0.5}$ being the misclassification rate of the previous tree.
This criterion is fulfilled as long as the decision tree performs better than
a random choice. The learning rate can be further modified by using an
exponent $\beta$ for the weight, $\alpha \to \alpha^{\beta}$. To have the same
effective number of events as before, \ie the same sum of weights, the events
are renormalised before training the next tree. The BDT output classifier $y$
is given by the weighted mean of the tree's output $h$, which is \num{+1} for
signal and \num{-1} for background,
\begin{align}
	y = \frac{1}{N_{\textrm{trees}}} \sum_i^{N_{\textrm{trees}}} \ln(\alpha_i) h_i\,.
\end{align}
This means that an event is classified more signal-like the higher the BDT
output classifier. Another boosting algorithm is the GradientBoost
method~\cite{GradientBoost}. It is based on minimising a defined loss
function, which describes the deviation between the classification and the
truth, by calculating the gradient of the loss function.

The importance of an input variable to the BDT training can be determined by
how often it is used inside the BDT and by how important the cuts are in terms
of events reaching the corresponding nodes and separation power between signal
and background.