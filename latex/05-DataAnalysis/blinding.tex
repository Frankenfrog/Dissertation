%!TEX root = ../main.tex

\section{Blinding}
\label{sec:dataanalysis:blinding}

Especially when performing precision measurements or searching for rare decays
it is advisable to blind the results throughout the analysis and even as
analyst only look at them after some thorough (ideally external) review
process. Blinding means that the central value or the final result is unknown
to all involved people. This procedure avoids the experimenter's bias, \ie the
unintended biasing of a result towards a known or expected value or towards a
(subconsciously) desired observation. The blinding transformation that has
been applied in the measurements of \CP violation in \BdToJPsiKS and \BdToDD
decays is adding a hidden offset to the fitted \CP parameters \Sf and \Cf
using the \verb|RooUnblindUniform| method of \roofit's
\verb|RooBlindTools|~\cite{roofit}. With this method the uncertainty on the
extracted parameters does not change and can still be used for optimising the
selection. Here, the offset is drawn from a uniform distribution between
\num{-2} and \num{+2} using a random number generator whose seed is generated
from a so-called \emph{blinding string}. As the physical range of
\sintwobeta is $[-1,1]$, this ensures a good opacity. A review of blind
analyses is given in Ref.~\cite{Blinding}.