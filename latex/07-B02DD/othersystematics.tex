%!TEX root = ../main.tex

%----------------------------------------------------------------------------%
\subsubsection[\texorpdfstring{$z$}{z}-scale]{\texorpdfstring{$\boldsymbol{z}$}{z}-scale}
\label{sec:b02dd:systematics:z_scale}

The decay times are determined by measuring the distance between PV and decay
vertex. So, any uncertainty on the positioning of detector elements
(especially the VELO modules) leads to biased decay times. Due to the high
boosting the main contribution to the flight distance is in $z$ direction. The
scale uncertainty in $z$ direction has been estimated to be
$\sigma_{z\text{-scale}} = \SI{0.022}{\percent}$~\cite{LHCb-ANA-2011-055}. The
influence on the measurement of the \CP observables is studied by performing
\num{1000} pseudoexperiments. For each pseudoexperiment a value for the uncertainty
on the $z$-scale is drawn from a Gaussian distribution around zero of width
$\sigma_{z\text{-scale}}$. The sum of \SI{50}{\fs} and the product of this
value with the decay time is used as width of the Gaussian function modelling
the decay time resolution in the generation. In the fit the width is set to
\SI{50}{\fs}. The product of the bias from the pull distributions of the
pseudoexperiments and the nominal statistical uncertainty is taken as
systematic uncertainty:
\begin{align*}
s_{\SDD}^{z\text{-scale}} = \num{0.0031}\ , \qquad s_{\CDD}^{z\text{-scale}} = \num{0.0028}\,.
\end{align*}

%----------------------------------------------------------------------------%
\subsubsection{Production Asymmetry}
\label{sec:b02dd:systematics:production_asymmetry}

The systematic uncertainty on the production asymmetry \prodasym{11} is
studied using \num{1000} pseudoexperiments. The nominal value is used in the
generation and the procedure described in Ref.~\cite{Karbach:1490463} is
applied in the fit: Before fitting the data sample the mean of the Gaussian
constraint for \prodasym{11} is shifted by one systematic uncertainty. The
resulting Gaussian distribution is used to draw a new value for the mean.
Then, the new Gaussian distribution is used to constrain \prodasym{11} in the
fit. Both shifts, upwards and downwards, are tested and the larger deviation
is taken as systematic uncertainty:
%
\begin{align*}
s_{\SDD}^{\prodasym{}} = \num{0.0015}\ , \qquad s_{\CDD}^{\prodasym{}} = \num{0.004}\,.
\end{align*}
%
For the production asymmetry difference $\Delta\prodasym{}$ the systematic
uncertainty is already included in the Gaussian constraint of the nominal fit.

%----------------------------------------------------------------------------%
\subsubsection{Decay Width Difference \texorpdfstring{\DGd}{Delta Gamma\_d}}
\label{sec:b02dd:systematics:deltagammad}

The decay width difference \DGd is expected to be very small and therefore
fixed to zero in the nominal fit. But experimentally it has a relatively large
uncertainty. This is taken into account by performing \num{1000}
pseudoexperiments where the current statistical precision $\sigma(\DGd) =
\SI{\pm0.007}{\invps}$~\cite{PDG2014} is used in the generation of the data
samples while it is, like in the nominal model, neglected in the fit. The mean
parameters of the pull distributions are converted into systematic
uncertainties of
\begin{align*}
s_{\SDD}^{\DGd} = \num{0.014}\ , \qquad s_{\CDD}^{\DGd} = \num{0.0021}\,.
\end{align*}

%----------------------------------------------------------------------------%
\subsubsection{\texorpdfstring{$\Bz$}{B0} Mass Difference \texorpdfstring{\dmd}{Delta m\_d}}
\label{sec:b02dd:systematics:deltamd}

The systematic uncertainty on the world average of \dmd
($\SI{\pm0.002}{\hbar\invps}$) is not covered by the Gaussian constraint that
is used in the nominal fit. Instead, it is analysed using \num{1000}
pseudoexperiments. In the generation the nominal model is used. Before
performing the fit the mean of the Gaussian distribution (its width is the
statistical precision of the world average) is shifted by one systematic
uncertainty (once up and once down) and a new value is drawn from the
distribution. This new constraint is then used in the minimisation. Looking at
the resulting pull distributions systematic uncertainties of
\begin{align*}
s_{\SDD}^{\dmd} = \num{0.0025}\ , \qquad s_{\CDD}^{\dmd} = \num{0.006}\,,
\end{align*}
are assigned.
