%!TEX root = ../main.tex

\section{Selection (5 pages)}
\label{sec:b02dd:selection}

The amount of background in \BdToDD is too high to perform a significant
measurement of \CP violation without any selection. The selection is divided
into three parts: a list of high signal efficiency requirements, a dedicated
treatment of mis-identified background and a multivariate approach to further
reduce combinatorial background.

\subsection{Vetoes}
\label{sec:b02dd:selection:vetoes}

A $K\rightarrow\pi$ mis-ID can lead to background contributions from
$\DspToKKpi$, which predominantly proceeds through $\DsTophipi$. To reduce
these $\Dsp$ contributions the kaon mass hypothesis is assigned to the pion
with the higher transverse momentum of $\DpToKpipi$ candidates. The candidate
is rejected if the invariant mass of the hypothetical kaon pair is compatible
with the $\phi$ mass of $M_{\phi} = \SI{1019.461}{\MeVcc}$~\cite{PDG2014}
within $\pm\SI{10}{\MeVcc}$ or if the invariant mass $m(\Km\Kp\pip)$ is
compatible with the \Dsp mass of $M_{\Dspm} =
\SI{1968.30}{\MeVcc}$~\cite{PDG2014} within $\pm\SI{25}{\MeVcc}$ and the pion
with the higher \pT (the one that the kaon mass hypothesis is assigned to) has
a larger $\texttt{ProbNN}K$ than $\texttt{ProbNN}\pion$ probability. When
assigning the kaon mass hypothesis to the pion with the lower \pT no vetoes
are applied as no resonant structures at the $\phi$ or the $\Dsp$ mass are
found.

To reduce $p\rightarrow\pi$ mis-ID the proton mass hypothesis is assigned to
the pion with the higher \pT of $\DpToKpipi$ candidates. The candidate is
rejected if the invariant mass of the $\kaon\proton\pion$ combination is
compatible with the \Lc mass of $M_{\Lc} =
\SI{2286.46}{\MeVcc}$~\cite{PDG2014} within $\pm\SI{25}{\MeVcc}$ and the
proton probability $\texttt{ProbNN}p$ of the pion with the higher \pT is
larger than $\texttt{ProbNN}\pion$.

\subsection{Multivariate analysis}
\label{sec:b02dd:selection:mva}