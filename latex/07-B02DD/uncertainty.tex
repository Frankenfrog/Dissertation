%!TEX root = ../main.tex

The uncertainty estimate of the nominal fit according to the \hesse matrix
which is \root's default option to calculate the uncertainties of
maximum-likelihood fits is $\num{\pm0.16}$ for \SDD and $\num{\pm0.16}$ for
\CDD. But it is a known issue that the uncertainty estimates implemented in
\root do not correctly propagate the influence of sWeights that are applied
to the likelihood fit. Additionally, it is possible that the uncertainty
estimates are asymmetric. The standard method of \root for these calculations
called \minos does not work if sWeights have to be applied. An alternative to get
proper uncertainty estimates is the bootstrap method. In this frequentist
model-independent approach a new data sample is generated by drawing events
from the nominal data sample until the number of candidates matches the
statistics in the original one (the same event can be drawn multiple times).
The nominal fit procedure, \ie performing the mass fit, calculating the
sWeights and fitting the weighted tagged decay time distribution, is executed
and the fit result is stored. The drawing and fitting is done \num{10000}
times. It turned out that half of the fits failed if (as originally planned)
the FT calibration parameters are constrained within their statistical
uncertainties determined using $\BdToDsD$ decays. So it has been decided to fix
them to their central values. The fit failure rate drops to a per-mille
effect. From the distribution of fit results (see
\cref{fig:decaytimefit:bootstrapping}) the two-side \SI{68}{\percent}
confidence intervals are used as uncertainty estimates:

\begin{align}
    \sigma^{\text{fixed FT}}_{\SDD} &= \,^{+0.162}_{-0.156} \ , \\
    \sigma^{\text{fixed FT}}_{\CDD} &= \,^{+0.175}_{-0.171} \ .
\end{align}

\begin{figure}[!htb]
\centering
% \includegraphics[width=0.48\textwidth]{06-DecayTimeFit/figs/parSigTimeSin2b_bootstrapped.pdf}
\hfill
% \includegraphics[width=0.48\textwidth]{06-DecayTimeFit/figs/parSigTimeCDD_bootstrapped.pdf}
\caption{Distributions of fit results for \SDD (left) and \CDD (right) from the
bootstrapped data samples. The central values are blinded but this just means
that the label at the x-axis is shifted. The width of the distribution is
still valid.}
\label{fig:decaytimefit:bootstrapping}
\end{figure}

The uncertainties on the flavour tagging calibration parameters are not
incorporated in the uncertainties on the $\CP$ observables. To do so, we
perform \num{10000} pseudo-experiments in which the nominal model is used to
generate the signal decay time distribution and the unblinded fit results of
the nominal fit are chosen for the \CP observables \SDD and \CDD. The flavour
tagging calibration parameters are drawn from Gaussian distributions around
their central values using the combined statistical + systematic uncertainties
in the generation. In the subsequent fit the flavour tagging calibration
parameters are fixed to their central values like in the nominal fit. This
results in pull distributions (see \cref{fig:decaytimefit:tagging:pulls})
which are broader than the standard normal distributions because pulls are
something like resolution in units of the statistical estimate. The deviation
of the width from one shows how much the statistical uncertainties are
underestimated in the likelihood fit due to not accounting for the variation
of the flavour tagging calibration parameters.
%
\begin{figure}[!htb]
% \includegraphics[width=0.49\textwidth]{06-DecayTimeFit/figs/parSigTimeSin2b_pull_tagging.pdf}
% \includegraphics[width=0.49\textwidth]{06-DecayTimeFit/figs/parSigTimeC_pull_tagging.pdf}
\caption{Pull distributions of \SDD and \CDD from a study where the flavour
tagging calibration parameters are varied in the generation and fixed in the
subsequent fit. The width corresponds to the factor by which the uncertainties
are underestimated.}
\label{fig:decaytimefit:tagging:pulls}
\end{figure}
%
So, the proper statistical uncertainties for \SDD and \CDD including the
impact of the uncertainty of the flavour tagging calibration parameters are
given by scaling the bootstrapping uncertainties by the width of the pull
distributions in \cref{fig:decaytimefit:tagging:pulls}:
%
\begin{align}
    \sigma^{\text{incl. FT}}_{\SDD} &= \,^{+0.169}_{-0.163} \ , \\
    \sigma^{\text{incl. FT}}_{\CDD} &= \,^{+0.177}_{-0.173} \ .
\end{align}
