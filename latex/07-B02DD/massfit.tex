%!TEX root = ../main.tex

\section{Mass fit}
\label{sec:b02dd:massfit}

In this section the fit of the invariant $m_{\Dp\Dm}$ mass distribution is
described, which is used to calculate signal weights via the \SPlot
method~\cite{Pivk:2004ty}, and thereby discriminates between signal and
background candidates. As the linear Pearson correlation coefficient between
the invariant mass and the decay time is determined to be $\rho =
\num{0.007}$, it is valid to apply the sWeights in the decay time fit to
obtain the \CP observables (see \cref{sec:b02dd:decaytimefit}).

The mass distribution is parametrised with a \PDF $\mathcal{P}$ consisting of
five components, \BdToDD signal, \BsToDD background, background from \BdToDsD,
background from \BsToDsD, and combinatorial background:
\begin{equation}
\resizebox{0.9\textwidth}{!}{$
  N^s \mathcal{P}^s = \yield{$s$}{\Bd} \pdf{$s$}{\Bd} + \yield{$s$}{\Bs} \pdf{$s$}{\Bs} + \yield{$s$}{\BdToDsD} \pdf{$s$}{\BdToDsD} + \yield{$s$}{\BsToDsD} \pdf{$s$}{\BsToDsD} + \yield{$s$}{Bkg} \pdf{$s$}{Bkg}$ \, .}
\end{equation}
In the extended maximum likelihood fit four disjoint categories are
simultaneously fitted. It is distinguished between the two years of
data-taking 2011 and 2012 and between the two final states $\KpipiKpipi$ and
$\KKpiKpipi$ ($s = \{2011,\Kpipi\}, \{2011, \KKpi\}, \{2012, \Kpipi\}, \{2012,
\KKpi\}$). The tagging output is not split.  In \cref{fig:massfit} the
complete data sample is plotted overlaid with the PDF projections and its
components. The individual shapes are explained in the following.

\begin{figure}[htb]
\centering
\includegraphics[width=0.85\textwidth]{07-B02DD/tikz/pdf/obsMass_summed_pull.pdf}
\caption{Plot of the reconstructed mass of the \BdToDD data sample with the
projected \PDF and pull distribution.}
\label{fig:massfit}
\end{figure}

\paragraph{$\boldsymbol{\BdToDD}$ signal:}
The $\BdToDD$ signal mass component is modelled by the sum of three Crystal
Ball functions~\cite{Skwarnicki:1986xj}, which share a common peak position
$\mu_{\Bd}$ but have different width parameters $\sigma_i$. Two of the Crystal
Ball functions have a tail towards lower masses and one has a tail towards
higher masses. The parameters $\alpha_1$ to $\alpha_3$ of the power law
functions, the ratio between the widths, and the fractions $f_1$ and $f_2$
between the Crystal Ball functions are determined from a fit to the invariant
$\Dp\Dm$ mass distribution of $\BdToDD$ signal MC in the range
\SIrange{4800}{5400}{\MeVcc}. This MC sample consists of both final states
generated in the ratio of the current world averages~\cite{PDG2016}. Apart
from the mass range the full selection is applied. Events, where photons are
missed in the reconstruction, create the very long tail towards lower masses,
which requires the third Crystal Ball function. The exponent of all power law
parts is fixed to \num{10}. The widths $\param{\sigma}{MC}{1}$ to
$\param{\sigma}{MC}{3}$ are multiplied by a common scale factor $R$ in the fit
to data to account for differences in the mass resolution between simulation
and data. The fit results are listed in \cref{tab:b02dd:massfit:mc_fitresults}
and a plot of the distribution overlaid with the projection of the PDF is
given in \cref{fig:b02dd:massfit:mc}.
\begin{figure}[htb]
\centering
\includegraphics[width=0.7\textwidth]{07-B02DD/tikz/pdf/obsMass_MC.pdf}
\caption{Mass distribution of the \BdToDD signal MC sample overlaid with the
projection of the fitted PDF. The blue dotted, green dashed and turquoise
short-dash-dotted lines represent the three Crystal Ball components.}%
\label{fig:b02dd:massfit:mc}
\end{figure}

\begin{table}[htb]
\centering
\caption{Fit results of the mass fit to \BdToDD signal MC.}%
\label{tab:b02dd:massfit:mc_fitresults}
\begin{tabular}{llr@{$\,\pm\,$}l}
  \toprule
  \multicolumn{2}{c}{Parameter}                   & \multicolumn{2}{c}{Value}  \\
  \midrule
  $\param{\mu}{MC}{\Bd}$    & ($\si{MeV/c^{2}}$)  & $5279.70$    & $0.09$      \\
  $\param{\sigma}{MC}{1}$   & ($\si{MeV/c^{2}}$)  & $8.5$        & $0.4$       \\
  $\param{\sigma}{MC}{2}$   & ($\si{MeV/c^{2}}$)  & $16$         & $5$         \\
  $\param{\sigma}{MC}{3}$   & ($\si{MeV/c^{2}}$)  & $9.0$        & $0.4$       \\
  $\param{f}{MC}{1}$        &                     & $0.48$       & $0.06$      \\
  $\param{f}{MC}{2}$        &                     & $0.0098$     & $0.0011$    \\
  $\param{\alpha}{MC}{1}$   &                     & $1.18$       & $0.08$      \\
  $\param{\alpha}{MC}{2}$   &                     & $0.12$       & $0.04$      \\
  $\param{\alpha}{MC}{3}$   &                     & $-1.46$      & $0.08$      \\
  \bottomrule
\end{tabular}
\end{table}

\paragraph{$\boldsymbol{\BsToDD}$ background:}
Apart from the $\Bd$ also the heavier $\Bs$ can decay to the $\Dp\Dm$ final state.
Almost the same parametrisation as for the $\Bd$ signal component is used, \ie
same width and tail parameters, while the peak position is shifted by the
world average $\dmBdBs = \mu_{\Bs} - \mu_{\Bd} =
\SI{87.35}{\MeVcc}$~\cite{PDG2016}.

\paragraph{$\boldsymbol{\BdToDsD}$ background:}
The vetoes applied in the selection suppress the contribution from
misidentified kaons. Nevertheless, a significant amount of \mbox{$\BdToDsD$}
decays remains in the data sample. A fit to the invariant mass distribution of
simulated $\BdToDsD$ events reconstructed as $\BdToDD$ is performed. The full
selection is applied to the simulated sample as this can change the shape of
the \BdToDsD background contribution. The sum of two Crystal Ball PDFs with
both power law exponents fixed to \num{10} is used to parametrise the
invariant mass distribution. The fit results are listed in
\cref{tab:massfit:DsDMC} and the corresponding plot is shown in
\cref{fig:massfit:DsDMC}. The fraction parameter and the tail parameters are
taken from this fit. The width parameters as well as the peak position are
floating parameters in the fit to data.

\begin{figure}[htb]
\centering
\includegraphics[width=0.6\textwidth]{07-B02DD/tikz/pdf/DsDMass_MC.pdf}
\caption{Mass distribution of the $\BdToDsD$ MC sample reconstructed as
$\BdToDD$ overlaid with the projection of the two Crystal Ball PDFs shown in
blue dashed and green dotted.}
\label{fig:massfit:DsDMC}
\end{figure}

\begin{table}[htb]
\centering
\caption{Fit results of the mass fit to \BdToDsD MC.}
\label{tab:massfit:DsDMC}
\begin{tabular}{llr@{$\,\pm\,$}l}
  \toprule
  \multicolumn{2}{c}{Parameter}                        & \multicolumn{2}{c}{Value} \\
  \midrule
  $\param{\mu}{MC}{\BdToDsD}$    & ($\si{MeV/c^{2}}$)  & $5222.2$    & $0.9$       \\
  $\param{\sigma}{MC}{1,\DsD}$   & ($\si{MeV/c^{2}}$)  & $15.0$      & $1.5$       \\
  $\param{\sigma}{MC}{2,\DsD}$   & ($\si{MeV/c^{2}}$)  & $20.7$      & $2.1$       \\
  $\param{f}{MC}{1,\DsD}$        &                     & $0.78$      & $0.13$      \\
  $\param{\alpha}{MC}{1,\DsD}$   &                     & $0.60$      & $0.09$      \\
  $\param{\alpha}{MC}{2,\DsD}$   &                     & $-1.8$      & $0.4$       \\
  \bottomrule
\end{tabular}
\end{table}

\paragraph{$\boldsymbol{\BsToDsD}$ background:}
Although only few candidates of $\BsToDsD$ decays are expected in the data
sample, a component for this contribution is included in the nominal fit. It
is parametrised with the sum of two Crystal Ball PDFs. All shape parameters
are shared with the $\Bd$ component, apart from the peak position, which is
constrained to be $\dmBdBs$ above the peak position of the $\Bd$ component.

\paragraph{Combinatorial background:}
The reconstructed mass PDF of the combinatorial background is modelled by an
exponential function with individual slopes $\param{\beta}{}{\KpipiKpipi}$ and
$\param{\beta}{}{\KKpiKpipi}$ based on the number of kaons in the final state.

\paragraph{Total fit:}
In \cref{tab:b02dd:FitResultsMass} the results of the floating shape
parameters of the mass fit to data are shown.
The $\Bd$ peak position is in good agreement with the current world average of
$\param{\mu}{WA}{\Bd} = \SI{5279.62\pm0.15}{\MeVcc}$~\cite{PDG2016}. The scale
factor $R$ is compatible with unity, which means that the mass resolution of
the signal component is well simulated. The slopes of the combinatorial
background differ significantly between the two subsamples showing the benefit
of splitting them to achieve an improved mass description.

\begin{table}[htb]
\centering
\caption{Results of the floating shape parameters in the mass fit to data.}
\label{tab:b02dd:FitResultsMass}
\centering
\begin{tabular}{llr@{$\,\pm\,$}l}
  \toprule
  \multicolumn{2}{c}{Parameter}                                & \multicolumn{2}{c}{Value}  \\
  \midrule
  $\param{\mu}{}{\Bd}$           & ($\si{MeV/c^{2}}$)          & $5279.26$    & $0.29$      \\
  $\param{R}{}{\Bd}$             &                             & $0.995$      & $0.032$     \\
  \midrule
  $\param{\mu}{}{\DsD}$          & ($\si{MeV/c^{2}}$)          & $5218.2$     & $1.1$       \\
  $\param{\sigma}{}{1,\DsD}$     & ($\si{MeV/c^{2}}$)          & $19.2$       & $2.7$       \\
  $\param{\sigma}{}{2,\DsD}$     & ($\si{MeV/c^{2}}$)          & $14.3$       & $3.1$       \\
  $\param{\beta}{}{\KpipiKpipi}$ & ($\si{1/(MeV/c^{2}})$)      & $-0.0031$    & $0.0005$    \\
  $\param{\beta}{}{\KKpiKpipi}$  & ($\si{1/(MeV/c^{2}})$)      & $-0.0041$    & $0.0006$    \\
  \bottomrule
\end{tabular}
\end{table}

The total number of $\Bd$ signal candidates is $N_{\Bd}$ = \num{1610\pm50}.
Due to the two times higher integrated luminosity and the increased production
cross-section, which in first order scales with the centre-of-mass energies
(\SI{8}{\TeV}/\SI{7}{\TeV}), one expects \num{2.3} times more signal
candidates in the 2012 subsample than in the 2011 subsample and this can
indeed be observed for the fitted yields. Additionally, there are around five
times more signal candidates in the final state with two kaons than with three
kaons, which also meets the expectations from the branching ratios.
