%!TEX root = ../main.tex

\section{Decay time fit}
\label{sec:b02dd:decaytimefit}

The conditional PDF describing the reconstructed decay time $t'$ and tag
decisions $\vect{d'} = (\dos, \dss)$, given a per-event decay time resolution
$\sigma_{t'}$ and per-event mistag probability estimates $\vect{\eta} = (\etaos,
\etass)$, is
%
\begin{equation}\label{eq:fullpdf}
  P\left(t',\vect{d'}\given \sigma_{t'},\vect{\eta}\right)
  \propto \epsilon(t') \left(\mathcal{P}(t,\vect{d'}\given \vect{\eta})
    \otimes \mathcal{R}(t'-t\given \sigma_{t'})\right)\,,
\end{equation}
%
where
\begin{equation}
  \mathcal{P}(t,\vect{d'}\given \vect{\eta}) \\
  \propto \sum_{d} \mathcal{P}(\vect{d'} \given d,\vect{\eta})
      [1 - d\, A_\text{P}] \,
      e^{-t/\tau}\left\{1 - d\, S \sin(\dm t) + d\, C \cos(\dm t)\right\}\,,
\end{equation}
and where $t$ is the true decay time, $d$ is the true production flavour,
$A_\text{P}$ is the production asymmetry, and $\mathcal{P}(\vect{d'} \given
d,\vect{\eta})$ is a two-dimensional binomial PDF describing the distribution
of tagging decisions given $\vect{\eta}$ and $d$. Normalisation factors are omitted for brevity.

%============================================================================%
%!TEX root = ../main.tex
\section{Decay time resolution}
\label{sec:dataanalysis::resolution}

Uncertainties in the determination of the position of vertices and in the
measurement of momenta (although thanks to the VELO (see
\cref{sec:detector:lhcb}) pretty accurate  at $\lhcb$) lead to a finite decay
time resolution $\sigma$, which dilutes the observed $\CP$ asymmetry by a
factor
\begin{align}
  \mathcal{D} = e^{\frac{-\dmd^2\,\sigma^2}{2}} \, .
\end{align}
This formula is the special case for a Gaussian resolution model with width
$\sigma$. The general formula is derived in
Ref.~\cite{ResolutionDilutionFactor}. For $\Bd$ mesons the dilution induced by
the decay time resolution has only minor influence on the measurement of $\CP$
observables because the oscillation frequency $\dmd$ of $\Bd$ mesons is
rather low. Even for a decay time resolution of \SI{100}{\fs} the dilution
factor is greater than \SI{99}{\percent}.

\FloatBarrier
%============================================================================%
%!TEX root = ../main.tex
\subsection{Decay time acceptance}
\label{sec:b02dd:decaytimefit:acceptance}

The trigger requirements as well as some input variables to the BDT result in
a decay-time-dependent efficiency. Additionally, the \velo reconstruction
(\ie the FastVelo algorithm~\cite{Callot:2011bza}) causes a drop in decay time
acceptance for events with large decay times. In order to correctly describe
these effects the $\Bd$ lifetime is constrained to its PDG value of $\tau =
\SI{1.519\pm0.005}{\ps}$~\cite{PDG2014} in the nominal fit and any deviation
of the decay time distribution (summed over the tags) from a pure exponential
shape is supposed to be described by cubic splines (see
\cref{sec:dataanalysis:splines}). Knots are positioned on the rising edge,
approximately at the turning point, and at the boundaries of the decay time
range, so at $\{\SIlist[list-final-separator={,
}]{0.25;0.8;2.0;10.25}{\ps}\}$. The normalisation of the splines is arbitrary
and it has been decided to fix the second to last spline coefficient to
$\num{1.0}$.

On signal MC the truth information is available so the shape of the decay time
acceptance can be separated from the exponential decay. This shape is compared
with the spline method described above. As the BDTs are trained and applied
separately for the two final states and might have different effects on the
shape of the decay time acceptance these two categories are studied
individually.

\begin{figure}[htb]
\centering
\includegraphics[width=0.48\textwidth]{07-B02DD/tikz/pdf/Acceptancespline_nolog_MC_Kpipi.pdf}
\includegraphics[width=0.48\textwidth]{07-B02DD/tikz/pdf/Acceptancespline_nolog_MC_KKpi.pdf}
\caption{Decay time acceptance of truth-matched signal MC for the $\KpipiKpipi$
final state (left) and the $\KKpiKpipi$ final state (right). The black data
points show the true decay time acceptance determined by dividing the
reconstructed by the true decay time distribution. The blue line is the spline
acceptance function and the red stripes indicate the $1\,\sigma$ error band
taking into account the statistical uncertainties.}
\label{fig:b02dd:decaytimefit:acceptance_MC}
\end{figure}

Looking at the plots in \cref{fig:b02dd:decaytimefit:acceptance_MC} it is
apparent that compared to $\BdToJPsiKS$ there is a quite large efficiency loss
at high decay times. This might be related to the fact that both $\Bd$
daughter particles ($\Dp$ and $\Dm$) are relatively long-lived. The true MC
decay time acceptance is overlaid with the shape of two spline functions.
Besides the spline function with the nominal number of four knots an
additional spline function with two more knots and slightly changed positions
$(\SIlist[list-final-separator={, }]{0.25;0.7;1.0;1.5;2.5;10.25}{\ps})$ is
plotted, which gives a better description. But it has to be considered that
the statistics of the MC sample is \num{25} times larger than the real data.
Therefore, the spline function with four knots is chosen, otherwise rather
statistical fluctuations than acceptance effects would be described. The low
statistics of the $\KKpiKpipi$ final state on real data does also not allow to
use separate spline coefficients for the two final states although with the
increased MC statistics some differences become visible.


\FloatBarrier
%============================================================================%
\subsection{External inputs}
\label{sec:b02dd:decaytimefit:constraints}

\lhcb has performed a measurement of the production asymmetry as a function
of transverse momentum and pseudorapidity using \SI{7}{\TeV}
data~\cite{LHCb-PAPER-2014-042}. Taking those distributions from \BdToDD
individual weighted averages for the 2011 and 2012 subsamples are calculated
yielding
%
\begin{equation}
  \begin{split}
    \prodasym{11} &= -0.0047 \pm 0.0106 \,\text{(stat)} \pm 0.0014 \, \text{(syst)} \,, \\
    \prodasym{12} &= -0.0071 \pm 0.0107 \,\text{(stat)} \pm 0.0014 \, \text{(syst)} \,.
  \end{split}
\end{equation}
%
As the measurement of the production asymmetry has been performed on 2011 data
only, the numbers for $\prodasym{11}$ and $\prodasym{12}$ are highly
correlated. So, the latter is modelled as $\prodasym{12} = \prodasym{11} +
\Delta\prodasym{}$ with $\Delta\prodasym{} = -0.0024 \pm 0.0018
\,\text{(syst)}$. The systematic uncertainty accounts for the difference of
the production asymmetries observed for the two data-taking conditions in the
measurement of the semileptonic $\CP$ asymmetry~\cite{LHCb-PAPER-2014-053} and
is used as the width of a Gaussian constraint. The $\Bz$ oscillation frequency and
the $\Bz$ lifetime are constrained to $\dm =
\SI{0.510\pm0.003}{\planckbar\invps}$~\cite{PDG2014} and $\tau =
\SI{1.519\pm0.005}{\ps}$~\cite{PDG2014}, respectively. The
flavour-tagging calibration parameters
(\cref{tab:dataanalysis:taggingcalibration:dsdcalibration}) are constrained
within their combined statistical and systematic uncertainties, determined in
the calibration using \BdToDsD decays. The decay time resolution parameters
(\cref{tab:b02dd:decaytimefit:resolution}) and the $\Bz$ lifetime difference
$\DG = \SI{0}{\invps}$ are fixed in the likelihood fit.

%============================================================================%
\subsection{Results}

The fit results of the $\CP$ observables from the decay time fit are
\begin{align}
\begin{split}
  \SDD                &= -0.54\,\pm\,^{0.17}_{0.16} \, , \\
  \CDD                &= \phantom{-}0.26\,\pm\,0.17 \, , \\
  \rho(\SDD,\CDD)     &= \phantom{-}0.48 \, . \\
\end{split}
\label{eq:b02dd:decaytimefit:cpresults}
\end{align}

Only after rescaling the sWeights via
\begin{align}
  w_i = w_i \frac{\sum w_i}{\sum w_i^2}\,,
\end{align}
correct asymmetric uncertainty estimates are delivered by \minos, which is
\root's standard method to analyse the likelihood shape. To check if the
coverage is guaranteed the bootstrapping method is applied. The nominal fit
procedure, \ie performing the mass fit, calculating the sWeights and fitting
the weighted tagged decay time distribution, is executed and the fit results
are stored. The drawing and fitting is done \num{10000} times. It turns out
that half of the fits fail if the flavour-tagging calibration parameters are
constrained within their statistical uncertainties. When fixing them to their
central values the fit failure rate drops to a per-mille effect. From the
distribution of fit results the two-side \SI{68}{\percent} confidence
intervals are extracted. To account for the uncertainties on the
flavour-tagging calibration parameters \num{10000} pseudoexperiments are
performed, in which the nominal model is used to generate the signal decay
time distribution and the fit results of the nominal fit are chosen for the
\CP observables \SDD and \CDD. Before generating the flavour-tagging
calibration parameters are drawn from Gaussian distributions around their
central values using the combined statistical + systematic uncertainties. In
the subsequent fit the flavour-tagging calibration parameters are fixed to
their central values, like in the fits to the bootstrapped samples. The
resulting pull distributions are broader than standard normal distributions.
The deviation of the width from unity shows how much the statistical
uncertainties are underestimated in the likelihood fit due to not accounting
for the variation of the flavour-tagging calibration parameters. So, the
statistical uncertainties for \SDD and \CDD from the bootstrapping including
the impact of the uncertainty of the flavour-tagging calibration parameters
are given by scaling the bootstrapping uncertainties by the width of the pull
distributions:
\begin{align}
    \sigma_{\SDD}(\text{bootstrapping}) &= \,^{+0.17}_{-0.16} \,, \\
    \sigma_{\CDD}(\text{bootstrapping}) &= \,^{+0.18}_{-0.17} \,.
\end{align}
These uncertainties match the nominal ones from \minos quite well. A plot of
the decay time distribution and the projection of the acceptance model are
shown in \cref{fig:b02dd:decaytimefit}. Good agreement between the latter and
the shape on signal MC (cf. \cref{fig:b02dd:decaytimefit:acceptance_MC}) can
be observed but the low statistics leading to rather large uncertainties
indicated by the error band diminishes the significance of the comparison.

\begin{figure}[htb]
\hspace*{\fill}
\begin{minipage}{0.4\textwidth}
\includegraphics[width=\textwidth]{07-B02DD/tikz/pdf/obsTime_summed_pull_logy.pdf}
\end{minipage}
\hfill
\begin{minipage}{0.5\textwidth}
\includegraphics[width=\textwidth]{07-B02DD/tikz/pdf/Acceptancespline_nolog.pdf}
\end{minipage}
\hspace*{\fill}
\caption{Plot of the decay time distribution of the background-subtracted \BdToDD
data sample with the projection of the \PDF and the pull distribution on the
left. The y-axis is plotted in logarithmic scale. Plot of the nominal decay
time acceptance model on the right. The red area indicates the 1\,$\sigma$
error band taking into account the statistical uncertainties.}
\label{fig:b02dd:decaytimefit}
\end{figure}

Apart from a quite high positive correlation between the parameters of the
acceptance spline function and the already quoted correlation of about
\num{0.5} between $\SDD$ and $\CDD$, which is expected from first principles
(see Ref.~\cite{LHCb-ANA-2011-004}), no large correlation between fitted parameters
is present as can be seen from the correlation matrix visualised in
\cref{fig:b02dd:decaytimefit:FullFitCorrMatrixHotCold}. A possible correlation
between $\dm$ and $\CDD$ is significantly reduced by the constraint applied on $\dm$,
which is a lot tighter than the sensitivity accessible from the data sample.
When releasing this constraint the correlation coefficient becomes \num{-0.8}.
But the sensitivity on $\CDD$ would significantly go down in this scenario so
the constraint on $\dm$ is maintained in the nominal setup.

\begin{figure}[htb]
\centering
\includegraphics[width=0.5\textwidth]{07-B02DD/tikz/pdf/FitResultsCorrMatrix_RedBlueDiscrete_wText.pdf}
\caption{Visualised correlation matrix of the fit parameters in the decay time
fit to data. Positive correlations are represented by the red palette on the $z$ axis,
while negative correlations are represented by the blue palette of the $z$
axis.}
\label{fig:b02dd:decaytimefit:FullFitCorrMatrixHotCold}
\end{figure}

The 1D likelihood scans in \cref{fig:b02dd:decaytimefit:1DLLScan} show a nice
parabolic shape with a clear minimum.
\begin{figure}[htb]
\centering
\includegraphics[width=0.48\textwidth]{07-B02DD/tikz/pdf/Likelihoodscan_sin2b.pdf}
\includegraphics[width=0.48\textwidth]{07-B02DD/tikz/pdf/Likelihoodscan_C.pdf}
\caption{One-dimensional likelihood profile scans for $\SDD$ and $\CDD$.}
\label{fig:b02dd:decaytimefit:1DLLScan}
\end{figure}

% The 2D likelihood scan is depicted in \cref{fig:b02dd:decaytimefit:2DLLScan}.
% \begin{figure}[tb]
% \centering
% % \includegraphics[width=0.48\textwidth]{07-B02DD/figs/2DLikelihoodscan.pdf}
% \caption{Two dimensional likelihood profile scan for $\SDD$ and $\CDD$.
% The contour line shows the $1\sigma$ confidence level.}
% \label{fig:b02dd:decaytimefit:2DLLScan}
% \end{figure}

In \cref{fig:b02dd:decaytimefit:asymmetry} the signal yield asymmetry is
plotted in eight bins of the decay time. A binned $\chisq$-fit to this signal
asymmetry is performed using
\begin{align}
{\mathcal A}^{\text{meas}}(t) = \frac{\Delta\omega + \prodasym{11}(1 - \Delta\omega) + (1 - \Delta\omega + \prodasym{11}\Delta\omega){\mathcal A}^{\text{theo}}(t)}{1 + \prodasym{11}(\SDD \sin(\dm\,t) - \CDD \cos(\dm\,t))}\,,
\label{eq:b02dd:decaytimefit:asymmetry_td}
\end{align}
which is a modified version of the theoretical signal asymmetry in
\cref{eq:cpviolation:asymmetry} and accounts for the asymmetries induced by
flavour tagging ($\Delta\omega$) and production asymmetry ($\prodasym{11}$).
The fit results
\begin{align*}
\begin{split}
  \SDD                &= -0.65\,\pm\,0.25 \,, \\
  \CDD                &= \phantom{-}0.24\,\pm\,0.26 \,,
\end{split}
\end{align*}
are compatible with those from the unbinned fit presented in
\cref{eq:b02dd:decaytimefit:cpresults} but not as sensitive.
\begin{figure}[htb]
\centering
\includegraphics[width=0.48\textwidth]{07-B02DD/tikz/pdf/Asymmetry.pdf}
\caption{Decay-time-dependent signal yield asymmetry. The solid blue curve is the
projection of the signal PDF given in \cref{eq:fullpdf} and the dashed red curve is the
pure time-dependent fit function from
\cref{eq:b02dd:decaytimefit:asymmetry_td}}
\label{fig:b02dd:decaytimefit:asymmetry}
\end{figure}

\FloatBarrier
