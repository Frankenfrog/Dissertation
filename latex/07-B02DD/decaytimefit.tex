%!TEX root = ../main.tex

\section{Decay time fit (10 pages)}
\label{sec:b02dd:decaytimefit}

%============================================================================%
%!TEX root = ../main.tex
\section{Decay time resolution}
\label{sec:dataanalysis::resolution}

Uncertainties in the determination of the position of vertices and in the
measurement of momenta (although thanks to the VELO (see
\cref{sec:detector:lhcb}) pretty accurate  at $\lhcb$) lead to a finite decay
time resolution $\sigma$, which dilutes the observed $\CP$ asymmetry by a
factor
\begin{align}
  \mathcal{D} = e^{\frac{-\dmd^2\,\sigma^2}{2}} \, .
\end{align}
This formula is the special case for a Gaussian resolution model with width
$\sigma$. The general formula is derived in
Ref.~\cite{ResolutionDilutionFactor}. For $\Bd$ mesons the dilution induced by
the decay time resolution has only minor influence on the measurement of $\CP$
observables because the oscillation frequency $\dmd$ of $\Bd$ mesons is
rather low. Even for a decay time resolution of \SI{100}{\fs} the dilution
factor is greater than \SI{99}{\percent}.

\FloatBarrier
%============================================================================%
%!TEX root = ../main.tex
\subsection{Decay time acceptance}
\label{sec:b02dd:decaytimefit:acceptance}

The trigger requirements as well as some input variables to the BDT result in
a decay-time-dependent efficiency. Additionally, the \velo reconstruction
(\ie the FastVelo algorithm~\cite{Callot:2011bza}) causes a drop in decay time
acceptance for events with large decay times. In order to correctly describe
these effects the $\Bd$ lifetime is constrained to its PDG value of $\tau =
\SI{1.519\pm0.005}{\ps}$~\cite{PDG2014} in the nominal fit and any deviation
of the decay time distribution (summed over the tags) from a pure exponential
shape is supposed to be described by cubic splines (see
\cref{sec:dataanalysis:splines}). Knots are positioned on the rising edge,
approximately at the turning point, and at the boundaries of the decay time
range, so at $\{\SIlist[list-final-separator={,
}]{0.25;0.8;2.0;10.25}{\ps}\}$. The normalisation of the splines is arbitrary
and it has been decided to fix the second to last spline coefficient to
$\num{1.0}$.

On signal MC the truth information is available so the shape of the decay time
acceptance can be separated from the exponential decay. This shape is compared
with the spline method described above. As the BDTs are trained and applied
separately for the two final states and might have different effects on the
shape of the decay time acceptance these two categories are studied
individually.

\begin{figure}[htb]
\centering
\includegraphics[width=0.48\textwidth]{07-B02DD/tikz/pdf/Acceptancespline_nolog_MC_Kpipi.pdf}
\includegraphics[width=0.48\textwidth]{07-B02DD/tikz/pdf/Acceptancespline_nolog_MC_KKpi.pdf}
\caption{Decay time acceptance of truth-matched signal MC for the $\KpipiKpipi$
final state (left) and the $\KKpiKpipi$ final state (right). The black data
points show the true decay time acceptance determined by dividing the
reconstructed by the true decay time distribution. The blue line is the spline
acceptance function and the red stripes indicate the $1\,\sigma$ error band
taking into account the statistical uncertainties.}
\label{fig:b02dd:decaytimefit:acceptance_MC}
\end{figure}

Looking at the plots in \cref{fig:b02dd:decaytimefit:acceptance_MC} it is
apparent that compared to $\BdToJPsiKS$ there is a quite large efficiency loss
at high decay times. This might be related to the fact that both $\Bd$
daughter particles ($\Dp$ and $\Dm$) are relatively long-lived. The true MC
decay time acceptance is overlaid with the shape of two spline functions.
Besides the spline function with the nominal number of four knots an
additional spline function with two more knots and slightly changed positions
$(\SIlist[list-final-separator={, }]{0.25;0.7;1.0;1.5;2.5;10.25}{\ps})$ is
plotted, which gives a better description. But it has to be considered that
the statistics of the MC sample is \num{25} times larger than the real data.
Therefore, the spline function with four knots is chosen, otherwise rather
statistical fluctuations than acceptance effects would be described. The low
statistics of the $\KKpiKpipi$ final state on real data does also not allow to
use separate spline coefficients for the two final states although with the
increased MC statistics some differences become visible.


\FloatBarrier
%============================================================================%
\subsection{External inputs}
\label{sec:b02dd:decaytimefit:constraints}

\lhcb has performed a measurement of the production asymmetry as a function
of transverse momentum and pseudorapidity using \SI{7}{\TeV}
data~\cite{LHCb-PAPER-2014-042}. Taking those distributions from \BdToDD
individual weighted averages for the 2011 and 2012 subsamples are calculated
yielding
%
\begin{equation}
  \begin{split}
    \prodasym{11} &= -0.0047 \pm 0.0106 \,\text{(stat)} \pm 0.0014 \, \text{(syst)} \,, \\
    \prodasym{12} &= -0.0071 \pm 0.0107 \,\text{(stat)} \pm 0.0014 \, \text{(syst)} \,.
  \end{split}
\end{equation}
%
As the measurement of the production asymmetry has been performed on 2011 data
only, the numbers for $\prodasym{11}$ and $\prodasym{12}$ are highly
correlated. So, the latter is modelled as $\prodasym{12} = \prodasym{11} +
\Delta\prodasym{}$ with $\Delta\prodasym{} = -0.0024 \pm 0.0018
\,\text{(syst)}$. The systematic uncertainty accounts for the difference of
the production asymmetries observed for the two data-taking conditions in the
measurement of the semileptonic $\CP$ asymmetry~\cite{LHCb-PAPER-2014-053} and
is used as width of a Gaussian constraint. The $\Bz$ oscillation frequency and
the $\Bz$ lifetime are constrained to their world averages $\dm =
\SI{0.510\pm0.004}{\planckbar\invps}$~\cite{HFAG} and $\tau =
\SI{1.519\pm0.005}{\ps}$~\cite{PDG2014}, respectively. The decay time
resolution parameters (\cref{tab:b02dd:decaytimefit:resolution}), the flavour
tagging calibration parameters
(\cref{tab:dataanalysis:taggingcalibration:dsdcalibration}) which are taken
from the \BdToDsD calibration and the $\Bz$ lifetime difference $\DG =
\SI{0}{\invps}$ are fixed in the likelihood fit.

%============================================================================%
\subsection{Statistical Uncertainty}
\label{sec:decaytimefit:uncertainty}
\todo{decide on how to assign statistical uncertainties}
%!TEX root = ../main.tex

The uncertainty estimate of the nominal fit according to the \hesse matrix
which is \root's default option to calculate the uncertainties of
maximum-likelihood fits is $\num{\pm0.16}$ for \SDD and $\num{\pm0.16}$ for
\CDD. But it is a known issue that the uncertainty estimates implemented in
\root do not correctly propagate the influence of sWeights that are applied
to the likelihood fit. Additionally, it is possible that the uncertainty
estimates are asymmetric. The standard method of \root for these calculations
called \minos does not work if sWeights have to be applied. An alternative to get
proper uncertainty estimates is the bootstrap method. In this frequentist
model-independent approach a new data sample is generated by drawing events
from the nominal data sample until the number of candidates matches the
statistics in the original one (the same event can be drawn multiple times).
The nominal fit procedure, \ie performing the mass fit, calculating the
sWeights and fitting the weighted tagged decay time distribution, is executed
and the fit result is stored. The drawing and fitting is done \num{10000}
times. It turned out that half of the fits failed if (as originally planned)
the FT calibration parameters are constrained within their statistical
uncertainties determined using $\BdToDsD$ decays. So it has been decided to fix
them to their central values. The fit failure rate drops to a per-mille
effect. From the distribution of fit results (see
\cref{fig:decaytimefit:bootstrapping}) the two-side \SI{68}{\percent}
confidence intervals are used as uncertainty estimates:

\begin{align}
    \sigma^{\text{fixed FT}}_{\SDD} &= \,^{+0.162}_{-0.156} \ , \\
    \sigma^{\text{fixed FT}}_{\CDD} &= \,^{+0.175}_{-0.171} \ .
\end{align}

\begin{figure}[!htb]
\centering
% \includegraphics[width=0.48\textwidth]{06-DecayTimeFit/figs/parSigTimeSin2b_bootstrapped.pdf}
\hfill
% \includegraphics[width=0.48\textwidth]{06-DecayTimeFit/figs/parSigTimeCDD_bootstrapped.pdf}
\caption{Distributions of fit results for \SDD (left) and \CDD (right) from the
bootstrapped data samples. The central values are blinded but this just means
that the label at the x-axis is shifted. The width of the distribution is
still valid.}
\label{fig:decaytimefit:bootstrapping}
\end{figure}

The uncertainties on the flavour tagging calibration parameters are not
incorporated in the uncertainties on the $\CP$ observables. To do so, we
perform \num{10000} pseudo-experiments in which the nominal model is used to
generate the signal decay time distribution and the unblinded fit results of
the nominal fit are chosen for the \CP observables \SDD and \CDD. The flavour
tagging calibration parameters are drawn from Gaussian distributions around
their central values using the combined statistical + systematic uncertainties
in the generation. In the subsequent fit the flavour tagging calibration
parameters are fixed to their central values like in the nominal fit. This
results in pull distributions (see \cref{fig:decaytimefit:tagging:pulls})
which are broader than the standard normal distributions because pulls are
something like resolution in units of the statistical estimate. The deviation
of the width from one shows how much the statistical uncertainties are
underestimated in the likelihood fit due to not accounting for the variation
of the flavour tagging calibration parameters.
%
\begin{figure}[!htb]
% \includegraphics[width=0.49\textwidth]{06-DecayTimeFit/figs/parSigTimeSin2b_pull_tagging.pdf}
% \includegraphics[width=0.49\textwidth]{06-DecayTimeFit/figs/parSigTimeC_pull_tagging.pdf}
\caption{Pull distributions of \SDD and \CDD from a study where the flavour
tagging calibration parameters are varied in the generation and fixed in the
subsequent fit. The width corresponds to the factor by which the uncertainties
are underestimated.}
\label{fig:decaytimefit:tagging:pulls}
\end{figure}
%
So, the proper statistical uncertainties for \SDD and \CDD including the
impact of the uncertainty of the flavour tagging calibration parameters are
given by scaling the bootstrapping uncertainties by the width of the pull
distributions in \cref{fig:decaytimefit:tagging:pulls}:
%
\begin{align}
    \sigma^{\text{incl. FT}}_{\SDD} &= \,^{+0.169}_{-0.163} \ , \\
    \sigma^{\text{incl. FT}}_{\CDD} &= \,^{+0.177}_{-0.173} \ .
\end{align}


%============================================================================%
\subsection{Results}

The fit results of the $\CP$ observables from the decay time fit are

\begin{align}
\begin{split}
  \SDD                &= -0.54\,\pm\,^{0.17}_{0.16} \ , \\
  \CDD                &= \phantom{-}0.26\,\pm\,^{0.18}_{0.17} \ , \\
  \rho(\SDD,\CDD)     &= 0.48 \ . \\
\end{split}
\label{eq:b02dd:decaytimefit:cpresults}
\end{align}

\Cref{fig:b02dd:decaytimefit} shows a plot of the decay time distribution of the
\BdToDD data sample and the projection of the acceptance model. The latter
shows (taking the rather large uncertainties into account) a good agreement
with the shape on signal MC (cf. \cref{fig:b02dd:decaytimefit:acceptance_MC}).

\begin{figure}[!htb]
\hspace*{\fill}
\begin{minipage}{0.4\textwidth}
\includegraphics[width=\textwidth]{07-B02DD/tikz/pdf/obsTime_summed_pull_logy.pdf}
\end{minipage}
\hfill
\begin{minipage}{0.4\textwidth}
\includegraphics[width=\textwidth]{07-B02DD/tikz/pdf/Acceptancespline_nolog.pdf}
\end{minipage}
\hspace*{\fill}
\caption{Plot of the decay time distribution of the background-subtracted \BdToDD
data sample with the projection of the \PDF and the pull distribution on the
left. The y-axis is plotted in logarithmic scale. Plot of the nominal decay
time acceptance model on the right. The red stripes indicate the 1\,$\sigma$
error band taking into account the statistical uncertainties.}
\label{fig:b02dd:decaytimefit}
\end{figure}

The correlation matrix is visualised in
\cref{fig:b02dd:decaytimefit:FullFitCorrMatrixHotCold}. The parameters of the
acceptance spline (fit results listed in
\cref{tab:b02dd:decaytimefit:SplineAcceptanceFitResults}) show a quite high positive
correlation among themselves. As expected (see~\cite{LHCb-ANA-2011-004}) \SDD
and \CDD have a correlation of about 0.5. Apart from these only quite small
correlations are present. Normally, one would expect a correlation between \dm and \CDD.
But the constraint which is applied on \dm is a lot tighter than the
sensitivity available from the data sample. So almost no correlation with \CDD
appears. In a cross-check where \dm is not constrained a correlation
coefficient of \num{-0.8} occurs. But because the sensitivity on \CDD goes
down significantly in this scenario the constraint on \dm is maintained in
the nominal setup even though the correlation between \SDD and \CDD would drop.

\begin{figure}[!htb]
\centering
% \includegraphics[width=0.8\textwidth]{07-B02DD/figs/FitResultsCorrMatrix_RedBlueDiscrete_wText.pdf}
\caption{Visualised correlation matrix of the fit parameters in the decay time
fit to data. Positive correlations are represented by the red palette on the $z$ axis,
while negative correlations are represented by the blue palette of the $z$
axis.}
\label{fig:b02dd:decaytimefit:FullFitCorrMatrixHotCold}
\end{figure}

\begin{table}[!htb]
\caption{Acceptance spline parameters from the decay time fit to \BdToDD data.}
\label{tab:b02dd:decaytimefit:SplineAcceptanceFitResults}
\centering
\begin{tabular}{lr@{$\,\pm\,$}l}
  \toprule
  Parameter                     & \multicolumn{2}{c}{Value} \\
  \midrule
    $h_1$  & $0.67$  & $0.06$  \\
    $h_2$  & $0.87$  & $0.11$  \\
    $h_4$  & $0.992$ & $0.010$ \\
    \bottomrule
\end{tabular}
\end{table}

The time-dependent signal asymmetry is shown in \cref{fig:b02dd:decaytimefit:asymmetry}.
\begin{figure}[!htb]
\centering
% \includegraphics[width=0.7\textwidth]{07-B02DD/figs/Asymmetry.pdf}
% \begin{tikzpicture}[scale=0.38]
% \input{07-B02DD/tikz/Asymmetry_final}
% \end{tikzpicture}
\caption{
Time-dependent signal-yield asymmetry $(N_{\Bzb} - N_{\Bz})/(N_{\Bzb} +
N_{\Bz})$. Here, $N_{\Bz} (N_{\Bzb})$ is the number of $\BdToDD$ decays with
a $\Bz$ ($\Bzb$) flavor tag. The data points are obtained with the sPlot
technique~\cite{Pivk:2004ty}, assigning signal weights to the events based on a
fit to the reconstructed mass distribution. The solid curve is the projection of
the signal PDF.}
\label{fig:b02dd:decaytimefit:asymmetry}
\end{figure}

The 1D likelihood scans can be found in \cref{fig:b02dd:decaytimefit:1DLLScan}.
\begin{figure}[!htb]
\centering
% \includegraphics[width=0.48\textwidth]{07-B02DD/figs/LLScan_1D_S.pdf}
% \includegraphics[width=0.48\textwidth]{07-B02DD/figs/LLScan_1D_C.pdf}
\caption{One dimensional likelihood profile scans for $\SDD$ and $\CDD$.}
\label{fig:b02dd:decaytimefit:1DLLScan}
\end{figure}

The 2D likelihood scan is depicted in \cref{fig:b02dd:decaytimefit:2DLLScan}.
\begin{figure}[!htb]
\centering
% \includegraphics[width=0.48\textwidth]{07-B02DD/figs/2DLikelihoodscan.pdf}
\caption{Two dimensional likelihood profile scan for $\SDD$ and $\CDD$.
The contour line shows the $1\sigma$ confidence level.}
\label{fig:b02dd:decaytimefit:2DLLScan}
\end{figure}

\FloatBarrier
