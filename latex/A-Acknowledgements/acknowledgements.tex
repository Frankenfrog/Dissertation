%!TEX root = ../main.tex

\chapter*{Acknowledgements}

\selectlanguage{ngerman}
Zunächst möchte ich mich bei Herrn Spaan bedanken. Seitdem ich im Oktober 2011
an Ihren Lehrstuhl gekommen bin, um erst meine Masterarbeit und nun diese
Doktorarbeit zu schreiben, waren Sie immer mit Rat und Tat für mich da.
Zusätzlich haben Sie uns immer wieder mit interessanten Anekdoten unterhalten.

Mein Dank gilt auch Professor Kröninger, der sich bereit erklärt hat, als
Zweitgutachter für meine Dissertation zu fungieren.

In den ersten zwei Jahren meiner Dissertation habe ich gemeinsam mit
Christophe an der Messung von CP Verletzung in \BdToJPsiKS Zerfällen
gearbeitet. Wir haben uns gegenseitig unterstützt, und so nicht nur eine
gelungene Analyse abgeliefert, sondern auch jeweils Dissertationsschriften
verfassen können. Danke schön.

\selectlanguage{english}
High energy physics is often a complex field of study, where progress is
impossible without collaborative work. Thanks to Paul, Nicoletta and Marta for
working with me on the \BdToDD analysis, especially for performing and
providing the flavour-tagging calibration.

\selectlanguage{ngerman}
Ich möchte mich auch ganz herzlich bei meinen Bürokollegen Alex, Margarete und
insbesondere Uli bedanken. Es war immer eine wunderbare Arbeitsatmosphäre, in
der ich mich wohl gefühlt habe. Probleme wurden untereinander diskutiert und
auch für Gespräche außerhalb des Unialltags war immer Zeit.

Weiterer Dank gebührt Julian für gute Ratschläge, für das Teilen von
Erfahrungen, produktive Diskussionen und allgemein Hilfe bei vielen Passagen,
die Eingang in diese Arbeit gefunden haben.

Wenn man zu lange an einem Dokument wie dieser Dissertation schreibt, bemerkt
man oftmals gar nicht, dass Bezüge fehlen oder Sachverhalte ungenügend erklärt
worden sind. Deshalb bin ich Timon, Moritz, Vanessa und Alex sehr dankbar
dafür, sich die Zeit genommen und Teile meiner Arbeit Korrektur gelesen zu
haben.

Plots problemlos ansprechend aussehen zu lassen, komplexe Systematikstudien
mit wenigen Zeilen Code erstellen und auswerten zu können, sowie etliche
nützliche Funktionen zur Verfügung zu haben, ist ein Luxus, an den man sich
all zu schnell gewöhnt, obwohl eine Menge Arbeit dahinter gesteckt hat. Vielen
Dank, Florian, dass du das DooSoftware-Framework erstellt hast.

Meine Begeisterung für die Physik wurde in der Oberstufe durch den leider viel
zu früh verstorbenen Herrn Bär, meinen Physiklehrer im Leistungskurs, geweckt.
Auch wenn Sie das nicht lesen können, vielen Dank, ohne Sie wäre ich wohl
nicht hier gelandet.