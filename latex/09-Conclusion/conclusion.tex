%!TEX root = ../main.tex

\chapter{Conclusion}
\label{sec:conclusion}

Since the discovery of \CP violation in 1964 by Christenson, Cronin, Fitch and
Turlay~\cite{CPV_discovery} there have been many experiments searching for \CP
violation, first in the sector of neutral $K$ mesons but later on also for
neutral $B$ mesons. The most significant indication of \CP violation for \Bz
mesons is found by the determination of \sintwobeta using \BdToJPsiKS decays.
But although \lhcb is not the first experiment, even not at a hadron collider,
to measure \CP violation it plays an important role in the further exploration
of the quark-mixing sector.

The measurement of \CP violation in \BdToJPsiKS
decays~\cite{LHCb-PAPER-2015-004} using proton-proton collision data
corresponding to an integrated luminosity of \SI{3}{\invfb}, which is
presented in this thesis, yields
\begin{align*}
  \SJpsiKS &=  \phantom{-}0.731 \, \pm 0.035 \, \text{(stat)} \pm 0.020 \, \text{(syst)}\,, \\
  \CJpsiKS &=  			- 0.038 \, \pm 0.032 \, \text{(stat)} \pm 0.005 \, \text{(syst)}\,,
\end{align*}
which is the most precise determination of these \CP observables at a hadron
collider to date and is almost as precise as the previous measurements by
\babar~\cite{BaBar_sin2beta} and \belle~\cite{Belle_sin2beta}. The central
values are compatible with the world averages and with the Standard Model
expectations. Thus, it serves as a benchmark measurement showing the
capability of \lhcb to perform flavour-tagged precision measurements of \CP
violation. The experimental difficulties, \eg decay time resolution,
production asymmetries and asymmetries induced by the flavour tagging, are
well under control as the result is statistically limited. The largest
systematic uncertainty on \SJpsiKS is introduced by a possible tagging
asymmetry of the background, which is not accounted for in the likelihood fit
(see \cref{sec:bd2jpsiks:systematics}). With an increased statistics this
effect can probably be analysed, controlled and suppressed better.
Furthermore, in the meantime new same-side flavour-tagging algorithms have
been developed~\cite{LHCb-PAPER-2016-039}, which are used for the first time
in the measurement of \CP violation in \BdToDD
decays~\cite{LHCb-PAPER-2016-037} yielding
\begin{align*}
  \SDD &=  -0.54 \, ^{+0.17}_{-0.16} \, \text{(stat)} \pm 0.05 \, \text{(syst)}\,, \\
  \CDD &=  \phantom{-}0.26 \, ^{+0.18}_{-0.17} \, \text{(stat)} \pm 0.02 \, \text{(syst)}\,.
\end{align*}
If the flavour-tagging performance was the same as in the \BdToJPsiKS
analysis, the 70 times lower number of available untagged signal candidates
(\num{114000} \BdToJPsiKS decays vs. \num{1610} \BdToDD decays) would only
allow a sensitivity of \num{\pm0.29} and \num{\pm0.27} for \SDD and
\CDD, respectively. However, the kinematic properties of the selected \BdToDD
candidates lead to a significantly higher tagging power and on top of that the
usage of the improved flavour-tagging algorithms increases the tagging power
by another \SI{20}{\percent}. The latter improvement can probably be exploited
in future measurements of \sintwobeta with \BdToJPsiKS decays. The value of
\effeff = \SI{8.1}{\percent} for the \BdToDD sample is the highest tagging
power to date in a tagged \CP violation measurement at \lhcb.

The main achievement by the measurement of \CP violation in \BdToDD decays is
to constrain the contribution of higher-order Standard Model corrections to be
small. The result of
\begin{align*}
	\Delta \phi = -0.16\,^{+0.19}_{-0.21}\,\si{\radian}
\end{align*}
can be transferred to the measurement of \CP violation in \BsToDsDs
decays~\cite{LHCb-PAPER-2014-051}, where \phis, the mixing phase of the \Bs
meson sector, can be determined, but only in a sum with this phase shift.

The analysis of \BdToDD decays is only the starting point for similar
measurements in other \allBToDD decay modes. First studies using \BdToDstD
decays have already been performed~\cite{BToDstDthesis}. Recent calculations
taking into account the flavour-tagging performance seen in \BdToDD and the
increase in statistics when adding data from Run~II indicate that the
sensitivity of \babar~\cite{Aubert:2008ah} and \belle~\cite{Rohrken:2012ta}
could be reached and even be topped.

The question is, what comes next? More and more data is collected at the \lhc
during Run~II and due to the higher centre-of-mass energy of currently
\SI{13}{\TeV} the \bbbar cross section, which roughly scales linearly with the
centre-of-mass energy, is even increased with respect to the proton-proton
collisions recorded during Run~I. In principle, the instantaneous luminosity
is another adjusting screw to further increase the data samples, though it is
already higher than the design value originally planned in the proposal for
the detector~\cite{LHCb-Technical-Proposal}. However, a key to significant
improvements, especially for decays with hadronic final states but also for
measurements of charmonium, is the performance of the trigger system. After
the upgrade in 2018--2020 it is planned to read out the full detector at
\SI{30}{\mega\hertz}~\cite{LHCb-TDR-016}. Right now, the signal efficiency of
the hardware trigger is not higher than \SI{50}{\percent}. Therefore, a large
potential for improvements exists. Another important aspect of tagged \CP
violation measurements is the performance of the flavour-tagging algorithms.
The higher the occupancy in the detector the more difficult it is to find the
appropriate tagging particle. There are ideas, at least for Run~II, how to
accommodate for this and the higher centre-of-mass energy helps in regaining
the flavour-tagging performance of Run~I.

Only the combination of all these efforts to increase the available amount of
data to be analysed might result in the observation of deviations from the
Standard Model expectations and thus an indirect hint for new physics. It's
not really a question of if new physics is needed but rather of how it looks
like. No dark matter candidate has been found so far. The origin of the
baryonic universe with the absence of antimatter can not be explained by the
amount of \CP violation incorporated in the SM.

\todo{Belle 2 and other future experiments like ILC?}
