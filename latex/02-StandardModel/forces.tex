%!TEX root = ../main.tex

\section{Forces and couplings (2 pages)}
\label{sec:standardmodel:forces}

The three interactions that are described in the SM are the electromagnetic,
the weak and the strong force. These differ dramatically in their strength and
the time scales, in which the interactions proceed. Strong decays typically
take \SI{e-23}{\second}, electromagnetic decays \SI{e-16}{\second}, while the
decay times of the weak interaction range between \SI{e-13}{\second} and a few
minutes.

The classical approach of the electromagnetic interaction is given by
Maxwell's equations. These are generalized into a relativistic quantum field
theory by the Quantum Electrodynamics
(QED)~\cite{Tomonaga01081946,Schwinger-QED1,*Schwinger-QED2,Feynman-QED1,*Feynman-QED2,*Feynman-QED3}.
The QED can be derived from the Lagrangian of a free fermion field
\begin{align}
	\mathcal{L}_0 = \overline{\Psi} (i\gamma^\mu \partial_\mu - m) \Psi
\end{align}
by extending the global to a local $U(1)$ symmetry. This is done by replacing
$\partial_\mu$ with the corresponding covariant derivative $D_\mu$
\begin{align}
	\partial_\mu \to D_\mu = \partial_\mu + i\,e\,A_\mu\,,
\end{align}
where the vector field $A_\mu$ can be identified as the photon, which mediates
the electromagnetic force via the coupling to the electric charge. The
dynamics enters through the introduction of a kinetic term
\begin{align}
	\mathcal{L}_A = - \frac 14 F_{\mu\nu}F^{\mu\nu}\,,
\end{align}
with the field strength tensor $F_{\mu\nu}$, which is the compressed
formulation of Maxwell's classical equations. However, in the SM the
electromagnetic interaction is unified with the weak interaction in the
electroweak $SU(2)\times U(1)$ symmetry
group~\cite{Glashow:1961tr,Salam:1964ry,Weinberg:1967tq}. The weak part
couples to the weak isospin\todo{introduce coupling constant $g_2$ here?}~and differs between left-handed and right-handed
fermion fields, where handedness gives the orientation of the spin with
respect to the momentum vector. Neutrinos have the peculiarity that they only
exist as right-handed fermions. The electroweak gauge symmetry is broken,
which becomes apparent as the photon is massless while the \Wpm and \Z bosons
are not. The Higgs mechanism~\cite{Higgs:1964pj} is responsible for this
symmetry breaking, which comes along with the need of the massive Higgs boson.
The masses of the quarks and leptons are generated through the Yukawa
interactions between the Higgs and the fermion fields. However, the
calculations do not contain any predictions for the coupling constants and
thus for the exact values of the masses. Moreover, they show that the weak
eigenstates of the down-type quarks (at least in the most common convention)
are a superposition of the mass eigenstates, where the relation is given by
the unitary Cabibbo-Kobayashi-Maskawa (CKM) matrix~\cite{Kobayashi:1973fv}.
This topic is explained in more detail in \cref{sec:cpviolation:kmmechanism}.

\todo{extend description of weak interaction, currents, GIM mechanism}

The third fundamental, strong force is described by the Quantum Chromodynamics
(QCD). It describes the binding between quarks and gluons, which are the
mediators of this interaction, through the colour charge in an $SU(3)$ gauge
symmetry group. The coupling heavily depends on the momentum scale, which in
the renormalisation theory can be understood as a running of the coupling
"constant" $\alpha_s$. Gluon polarization, which is possible as gluons carry
colour charge by themselves and therefore can couple to each other,
outperforms quark polarization effects leading to asymptotic freedom of the
quarks on very short
distances~\cite{AsymptoticFreedom_GrossWilczek,AsymptoticFreedom_Politzer}. On
the other hand quarks can not separate too much from each other or -- at least
according to one possible scenario for confinement -- a quark-antiquark pair
is produced in between. Both effects can be summarized in the quark-antiquark
potential
\begin{align}
	V_{\mathrm{QCD}} = - \frac 43 \frac{\alpha_s}{r} + k\,r\,,
\end{align}
where $r$ is the distance between the two fermions and $k \approx
\SI{1}{\GeV\fm^{-1}}$.
\todo{reference for k in QCD potential}