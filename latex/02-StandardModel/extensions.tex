%!TEX root = ../main.tex

\section{Problems and possible extensions}
\label{sec:standardmodel:extensions}

Although the SM has proven to be a very successful and predictive theory,
there are several issues that can not be explained in the SM and others that
appeal very constructed. The latter leads to the idea of a more fundamental
theory in which the SM is embedded. The concept of a unification has first
been proposed by Georgi and Glashow~\cite{Georgi:1974sy}. A first step would
be a generalization of the electroweak with the strong interaction. Then,
gravitation, the fourth fundamental force, could be included, whose effect is
almost negligible at the energy scale, which is probed in today's high energy
physic experiments, and therefore not part of the SM. Here, the difficulty is
that even nowadays gravitation is still based on Einstein's general theory of
relativity, so unlike the other theories of the SM not in a quantum mechanical
framework. The unification of the forces would probably emerge at energy
scales of \SI{e16}{\GeV}. However, quantum corrections from those mass scales
would heavily influence the Higgs mass, which is measured to be around
\SI{125}{\GeVcc}~\cite{HiggsMass}. In the SM this hierarchy problem is
solved by a fine tuning of tree-level and loop contributions, which exactly
cancel each other. Other explanations are given by extending the SM with new
symmetries, like models including
supersymmetry~\cite{Gervais:1971ji,Golfand:1971iw,Volkov:1972jx,*Volkov:1973ix}.
From a theoretical point of view it is also unsatisfactory that the SM
includes so many free parameters, like the masses of the constituents or the
number of generations. In addition, the SM only applies to the processes of
ordinary matter, which makes only about \SI{5}{\percent} of the total energy
density in the universe~\cite{Ade:2015xua}, whereas it lacks an explanation
for dark matter or dark energy. Furthermore, the amount of \CP violation in
the weak sector can not account for the baryon asymmetry in the universe, \ie
the dominance of matter without any large clusters of antimatter in the
universe. The observation of neutrino
oscillations~\cite{Fukuda:1998mi,Ahmad:2001an,*Ahmad:2002jz}, \ie the indirect
measurement of mass differences between the neutrino generations, shows that
neutrinos can not be massless as assumed in the SM. The special role of
neutrinos in the SM to appear only left-handed could be corrected if they were
Majorana particles, \ie their own antiparticles~\cite{Majorana:1937vz}.
