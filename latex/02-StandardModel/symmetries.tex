%!TEX root = ../main.tex

\section{Symmetries and conservation laws(1 page)}
\label{sec:standardmodel:symmetries}

Symmetries, global or local as well as continuous or discrete, structure the
Standard Model. According to Noether's theorem every symmetry corresponds to a
conservation law~\cite{Noether}.

As a whole the SM is constructed to be invariant under gauge transformations,
meaning that the physics is independent of the choice of the gauge. This
symmetry is only broken by the vacuum expectation value of the Higgs field,
which is the origin of the masses of the $W$ and $Z$ bosons. Gauge invariance
is associated with the conservation of charge, electrical and of colour. The
invariance under space-time translation corresponds to the conservation of
energy. However, Heisenberg's uncertainty principle~\cite{Heisenberg:1927zz}
allows the violation of the conservation of energy for a very short period of
time. This enables the existence of virtual, heavy particles in decay
processes. The invariance of a system under translation in space and rotation
leads to the conservation of momentum and angular momentum, respectively.
Moreover, the baryon number (B) and the lepton number (L) are conserved
respectively only broken by tiny non-pertubative effects. However, in the
early universe these effects might have been larger~\cite{Rubakov:1996vz},
while (B $-$ L) is an exact symmetry. The lepton family number, \ie the
individual lepton number for electrons, muons and tauons, also seems to be
conserved, at least no significant asymmetry has been found yet. But there is
no symmetry group evoking this conservation law and tests of the lepton
universality by LHCb using \mbox{$\Bu\!\to \Kp\ell^+\ell^-$} decays show an
asymmetry with a significance corresponding to \num{2.6} standard
deviations~\cite{LHCb-PAPER-2014-024}. While all the former symmetries are
absolute, \ie valid for all three interactions, there are also approximate
symmetries that only apply for the electromagnetic and strong but not for the
weak force. For example, flavour transitions are only possible in the weak
interaction. Thus, flavour symmetry is an approximate symmetry. The same
applies for the parity operation ($P$), which performs a spatial inversion of
all coordinates
\begin{align}
	P\Psi(r) = \Psi(-r)\,,
\end{align}
or in other words it transforms left-handed into right-handed fermions. In
fact, the weak interaction even maximally violates parity, \eg there are only
left-handed neutrinos and right-handed antineutrinos. Charge conjugation ($C$)
is another discrete symmetry. It changes the sign of all charges and the
magnetic moment, and thus transforms particles into their antiparticles
\begin{align}
	C \ket{p} = \ket{\bar p}\,.
\end{align}
The combination of charge conjugation and parity (\CP) is more stable,
\eg a left-handed neutrino becomes a right-handed antineutrino. Nevertheless,
it is still violated at the \num{e-4} level by the weak interaction. \CP
violation is explained in more detail in \cref{sec:cpviolation}. Not before
combining \CP with time reversal ($T$), one of the most fundamental concepts
of the SM is found, the \CPT symmetry. The \CPT
theorem~\cite{Schwinger:1951xk,Luders:1954zz,pauli1955niels} states that
particles and antiparticles have the same mass and the same lifetime.
% OZI rule~\cite{Okubo,Zweig,Iizuka:1966fk}, named after Okubo, Zweig, Iizuka