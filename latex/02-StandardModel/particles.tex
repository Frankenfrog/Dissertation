%!TEX root = ../main.tex

\section{Particles (1 page)}
\label{sec:standardmodel:particles}

In the SM 12 fermions, elementary particles with spin \sfrac{1}{2}, and the
same number of antifermions, which have the opposite charge-related quantum
numbers, as well as 12 gauge bosons with integer spin, which mediate the
forces (see \cref{sec:standardmodel:forces}), are described. The fermions are
divided into six quarks and six leptons. The quarks are further subdivided
into three generations, which each contain an up-type and a down-type quark.
The common matter, protons and neutrons, is built up from the quarks of the
first generation, the up quark (\uquark) and the down quark (\dquark). Their
heavier partners are the charm (\cquark) and the top quark (\tquark)
respectively the strange (\squark) and the bottom quark (\bquark):
\begin{align}
\begin{pmatrix}
\uquark \\ \dquark
\end{pmatrix}
\begin{pmatrix}
\cquark \\ \squark
\end{pmatrix}
\begin{pmatrix}
\tquark	\\ \bquark
\end{pmatrix}
\end{align}
Due to their confinement~\cite{Confinement} quarks are
always part of bound states, so called hadrons (terminology introduced by
L.~B.~Okun~\cite{Okun:1962kca}). A quark and an antiquark form a meson, three
quarks a baryon, and just recently evidence for the existence of four and five
quark bound states (tetraquarks respectively pentaquarks) has been
found~\cite{LHCb-PAPER-2016-018,*LHCb-PAPER-2016-019,LHCb-PAPER-2015-029,LHCb-PAPER-2016-015}.
A colour charge is associated to the quarks, which can take three different
types. However, the colour charges add up in a way that the hadrons are
colourless. The electric charge of the up-type quarks is +\sfrac{2}{3} and of
the down-type quarks $-$\sfrac{1}{3} of the elementary charge. The leptons are
also classified in three families, each consisting of a negatively charged
particle, in increasing order of mass the electron (\electron), the muon
(\muon) and the tauon (\tauon), and a corresponding neutral neutrino, which is
set to be massless:
\begin{align}
\begin{pmatrix}
\en \\ \neue
\end{pmatrix}
\begin{pmatrix}
\mun \\ \neum
\end{pmatrix}
\begin{pmatrix}
\taum \\ \neut
\end{pmatrix}
\end{align}
The gauge bosons are the massless photon (\g), the three massive mediators of
the weak force (\Wp, \Wm, \Z), and the eight gluons, which carry different
colour charge configurations. The SM has been completed by the discovery of
the Higgs boson~\cite{Higgs-Atlas,Higgs-CMS}, a massive boson with spin 0.
