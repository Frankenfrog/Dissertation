%!TEX root = ../main.tex

\chapter{Standard Model of Particle Physics}
\label{sec:standardmodel}

The Standard Model of particle physics (SM) is a renormalisable
gauge-invariant quantum field theory, which describes the fundamental
constituents of matter (see \cref{sec:standardmodel:particles}) and three of
the four fundamental interactions between them (see
\cref{sec:standardmodel:forces}). It is structured by the conservation
and (spontaneous) breaking of symmetries (see
\cref{sec:standardmodel:symmetries}). Despite being very successful in
explaining experimental data and even predicting their results, there is a
number of issues showing that extensions of the SM are required (see
\cref{sec:standardmodel:extensions}). The information given in this chapter is
mainly inspired by the description in Ref.~\cite{Brock:2011zz} and based on
explanations from Refs.~\cite{Griffiths:2008zz} and~\cite{Perkins:1982xb}.

%!TEX root = ../main.tex

\section{Particles}
\label{sec:standardmodel:particles}

In the SM 12 fermions, which are elementary particles with spin \sfrac{1}{2},
and the same number of antifermions, which have the opposite charge-related
quantum numbers, are described. The fermions are divided into six quarks and
six leptons. The quarks are further subdivided into three generations, which
each contain an up-type and a down-type quark. The common matter, protons and
neutrons, is built up from the quarks of the first generation, the up quark
(\uquark) and the down quark (\dquark). Their heavier partners are the charm
(\cquark) and the top quark (\tquark) respectively the strange (\squark) and
the bottom quark (\bquark):
\begin{align}
\begin{pmatrix}
\uquark \\ \dquark
\end{pmatrix}
\begin{pmatrix}
\cquark \\ \squark
\end{pmatrix}
\begin{pmatrix}
\tquark	\\ \bquark
\end{pmatrix}
\end{align}
Due to confinement~\cite{Confinement} quarks are always part of bound states,
so called hadrons (terminology introduced by L.~B.~Okun~\cite{Okun:1962kca}).
A quark and an antiquark form a meson, three quarks a baryon, and just
recently evidence for the existence of four and five quark bound states
(tetraquarks respectively pentaquarks) has been
found~\cite{LHCb-PAPER-2016-018,*LHCb-PAPER-2016-019,LHCb-PAPER-2015-029,LHCb-PAPER-2016-015}.
A colour charge is associated to the quarks, which can take three different
types. However, the colour charges add up in a way that the hadrons are
colourless. The electric charge of the up-type quarks is +\sfrac{2}{3} and of
the down-type quarks $-$\sfrac{1}{3} of the elementary charge. The leptons are
also classified in three families, each consisting of a negatively charged
particle, in increasing order of mass the electron (\electron), the muon
(\muon) and the tauon (\tauon), and a corresponding neutral neutrino, which is
set to be massless in the SM:
\begin{align}
\begin{pmatrix}
\en \\ \neue
\end{pmatrix}
\begin{pmatrix}
\mun \\ \neum
\end{pmatrix}
\begin{pmatrix}
\taum \\ \neut
\end{pmatrix}
\end{align}
Additionally, 12 gauge bosons with integer spin, which mediate the
forces (see \cref{sec:standardmodel:forces}), are described in the SM.
The gauge bosons are the massless photon (\g), the three massive mediators of
the weak force (\Wp, \Wm, \Z), and the eight gluons, which carry different
colour charge configurations. The SM has been completed by the discovery of
the Higgs boson~\cite{Higgs-Atlas,Higgs-CMS}, a massive boson with spin 0.

\todo{figure of SM particles?}


%!TEX root = ../main.tex

\section{Forces and couplings}
\label{sec:standardmodel:forces}

The three interactions that are described in the SM are the electromagnetic,
the weak and the strong force. These differ dramatically in their strength and
the time scales, in which the interactions proceed. Strong decays typically
take \SI{e-23}{\second}, electromagnetic decays \SI{e-16}{\second}, while the
decay time of weak interactions ranges from \SI{e-13}{\second} to a few
minutes.

The classical approach of the electromagnetic interaction is given by
Maxwell's equations. These are generalised into a relativistic quantum field
theory by the Quantum Electrodynamics
(QED)~\cite{Tomonaga01081946,Schwinger-QED1,*Schwinger-QED2,Feynman-QED1,*Feynman-QED2,*Feynman-QED3}.
The QED can be derived from the Lagrangian of a free fermion field
\begin{align}
	\mathcal{L}_0 = \overline{\Psi} (i\gamma^\mu \partial_\mu - m) \Psi
\end{align}
by extending the global to a local $U(1)$ symmetry. This is done by replacing
$\partial_\mu$ with the corresponding covariant derivative $D_\mu$
\begin{align}
	\partial_\mu \to D_\mu = \partial_\mu + i\,e\,A_\mu\,,
\end{align}
where the vector field $A_\mu$ can be identified as the photon, which mediates
the electromagnetic force via the coupling to the electric charge. The
dynamics is introduced by the kinetic term
\begin{align}
	\mathcal{L}_A = - \frac 14 F_{\mu\nu}F^{\mu\nu}\,,
\end{align}
with the field strength tensor $F_{\mu\nu}$, which is the compressed
formulation of Maxwell's classical equations. However, in the SM the
electromagnetic interaction is unified with the weak interaction in the
electroweak $SU(2)\times U(1)$ symmetry
group~\cite{Glashow:1961tr,Salam:1964ry,Weinberg:1967tq}. The weak part
couples to the weak isospin and differs between left-handed and right-handed
fermion fields, where handedness gives the orientation of the spin with
respect to the momentum vector. Neutrinos have the peculiarity that they only
exist as right-handed fermions. The electroweak gauge symmetry is broken,
which becomes apparent, as the photon is massless, while the \Wpm and \Z bosons
are not. The Higgs mechanism~\cite{Higgs:1964pj} is responsible for this
symmetry breaking, which comes along with the need for the massive Higgs boson.
The masses of the quarks and leptons are generated through the Yukawa
interactions between the Higgs and the fermion fields. However, the
calculations do not contain any predictions for the coupling constants and
thus for the numerical values of the masses. Moreover, the calculations show that the weak
eigenstates of the down-type quarks (at least in the most common convention)
are a superposition of the mass eigenstates, where the relation is given by
the unitary Cabibbo-Kobayashi-Maskawa (CKM) matrix~\cite{Kobayashi:1973fv}.
This topic is explained in more detail in \cref{sec:cpviolation:kmmechanism}.

\todo{extend description of weak interaction, currents, GIM mechanism?}

The third fundamental force, called the strong force, is characterised by the
Quantum Chromodynamics (QCD). It describes the binding between quarks and
gluons, which are the mediators of this interaction, through the colour charge
in a $SU(3)$ gauge symmetry group. The coupling heavily depends on the
momentum scale, which in the renormalisation theory can be understood as a
running of the coupling "constant" $\alpha_s$. Gluon polarisation, which is
possible as gluons carry colour charge by themselves and therefore can couple
to each other, outperforms quark polarisation effects leading to asymptotic
freedom of the quarks on very short
distances~\cite{AsymptoticFreedom_GrossWilczek,AsymptoticFreedom_Politzer}. On
the other hand, quarks can not separate too much from each other or --- at
least according to one possible scenario for confinement --- a quark-antiquark
pair is produced in between. Both effects can be summarised in the
quark-antiquark potential
\begin{align}
	V_{\mathrm{QCD}} = - \frac 43 \frac{\alpha_s}{r} + k\,r\,,
\end{align}
where $r$ is the distance between the two fermions and $k \approx
\SI{1}{\GeV\fm^{-1}}$~\cite{Perkins:1982xb}.

%!TEX root = ../main.tex

\section{Symmetries and conservation laws(1 page)}
\label{sec:standardmodel:symmetries}

Symmetries, global or local as well as continuous or discrete, structure the
Standard Model. According to Noether's theorem every symmetry corresponds to a
conservation law~\cite{Noether}.

As a whole the SM is constructed to be invariant under gauge transformations,
meaning that the physics is independent of the choice of the gauge. This
symmetry is only broken by the vacuum expectation value of the Higgs field,
which is the origin of the masses of the $W$ and $Z$ bosons. Gauge invariance
is associated with the conservation of charge, electrical and of colour. The
invariance under space-time translation corresponds to the conservation of
energy. However, Heisenberg's uncertainty principle~\cite{Heisenberg:1927zz}
allows the violation of the conservation of energy for a very short period of
time. This enables the existence of virtual, heavy particles in decay
processes. The invariance of a system under translation in space and rotation
leads to the conservation of momentum and angular momentum, repectively.
Moreover, the baryon number (B) and the lepton number (L) are conserved
respectively only broken by tiny non-pertubative effects. However, in the
early universe these effects might have been larger~\cite{Rubakov:1996vz},
while (B $-$ L) is an exact symmetry. The lepton family number, \ie the
individual lepton number for electrons, muons and tauons, also seems to be
conserved, at least no significant asymmetry has been found yet. But there is
no symmetry group evoking this conservation law and tests of the lepton
universality by LHCb using \mbox{$\Bu\!\to \Kp\ell^+\ell^-$} decays show an
asymmetry with a significance corresponding to \num{2.6} standard
deviations~\cite{LHCb-PAPER-2014-024}. While all the former symmetries are
absolute, \ie valid for all three interactions, there are also approximate
symmetries that only apply for the electromagnetic and strong but not for the
weak force. For example, flavour transitions are only possible in the weak
interaction. Thus, flavour symmetry is an approximate symmetry. The same
applies for the parity operation ($P$), which performs a spatial inversion of
all coordinates
\begin{align}
	P\Psi(r) = \Psi(-r)\,,
\end{align}
or in other words it transforms left-handed into right-handed fermions. In
fact, the weak interaction even maximally violates parity, \eg there are only
left-handed neutrinos and right-handed antineutrinos. Charge conjugation ($C$)
is another discrete symmetry. It changes the sign of all charges and the
magnetic moment, and thus transforms particles into their antiparticles
\begin{align}
	C \ket{p} = \ket{\bar p}\,.
\end{align}
The combination of charge conjugation and parity (\CP) is more stable,
\eg a left-handed neutrino becomes a right-handed antineutrino. Nevertheless,
it is still violated at the \num{e-4} level by the weak interaction. \CP
violation is explained in more detail in \cref{sec:cpviolation}. Not before
combining \CP with time reversal ($T$), one of the most fundamental concepts
of the SM is found, the \CPT symmetry. The \CPT
theorem~\cite{Schwinger:1951xk,Luders:1954zz,pauli1955niels} states that
particles and antiparticles have the same mass and the same lifetime.
% OZI rule~\cite{Okubo,Zweig,Iizuka:1966fk}, named after Okubo, Zweig, Iizuka

%!TEX root = ../main.tex

\section{Problems and possible extensions}
\label{sec:standardmodel:extensions}

Although the SM has proven to be a very successful and predictive theory,
there are several issues that can not be explained in the SM and others that
appeal very constructed. The latter leads to the idea of a more fundamental
theory in which the SM is embedded. The concept of a unification has first
been proposed by Georgi and Glashow~\cite{Georgi:1974sy}. A first step would
be a generalization of the electroweak with the strong interaction. Then,
gravitation, the fourth fundamental force, could be included, whose effect is
almost negligible at the energy scale, which is probed in today's high energy
physic experiments, and therefore not part of the SM. Here, the difficulty is
that even nowadays gravitation is still based on Einstein's general theory of
relativity, so unlike the other theories of the SM not in a quantum mechanical
framework. The unification of the forces would probably emerge at energy
scales of \SI{e16}{\GeV}. However, quantum corrections from those mass scales
would heavily influence the Higgs mass, which is measured to be around
\SI{125}{\GeVcc}~\cite{HiggsMass}. In the SM this hierarchy problem is
solved by a fine tuning of tree-level and loop contributions, which exactly
cancel each other. Other explanations are given by extending the SM with new
symmetries, like models including
supersymmetry~\cite{Gervais:1971ji,Golfand:1971iw,Volkov:1972jx,*Volkov:1973ix}.
From a theoretical point of view it is also unsatisfactory that the SM
includes so many free parameters, like the masses of the constituents or the
number of generations. In addition, the SM only applies to the processes of
ordinary matter, which makes only about \SI{5}{\percent} of the total energy
density in the universe~\cite{Ade:2015xua}, whereas it lacks an explanation
for dark matter or dark energy. Furthermore, the amount of \CP violation in
the weak sector can not account for the baryon asymmetry in the universe, \ie
the dominance of matter without any large clusters of antimatter in the
universe. The observation of neutrino
oscillations~\cite{Fukuda:1998mi,Ahmad:2001an,*Ahmad:2002jz}, \ie the indirect
measurement of mass differences between the neutrino generations, shows that
neutrinos can not be massless as assumed in the SM. The special role of
neutrinos in the SM to appear only left-handed could be corrected if they were
Majorana particles, \ie their own antiparticles~\cite{Majorana:1937vz}.
