%!TEX root = ../main.tex

\chapter{Standard Model of Particle Physics (5 pages)}
\label{sec:standardmodel}

The Standard Model of particle physics (SM) is a renormalisable
gauge-invariant quantum field theory, which describes the fundamental
constituents of matter (see \cref{sec:standardmodel:particles}) and three of
the four fundamental interactions between them (see
\cref{sec:standardmodel:forces}). It is structured by the conservation
and (spontaneous) breaking of symmetries (see
\cref{sec:standardmodel:symmetries}). Despite being very successful in
explaining experimental data and even predicting their results, there is a
number of issues showing that extensions of the SM are required (see
\cref{sec:standardmodel:extensions}). The following information is mainly
inspired by the description in Ref.~\cite{Brock:2011zz} and based on
explanations given in Refs.~\cite{Griffiths:2008zz,Perkins:1982xb}.

%!TEX root = ../main.tex

\section{Particles (1 page)}
\label{sec:standardmodel:particles}

%!TEX root = ../main.tex

\section{Forces and couplings}
\label{sec:standardmodel:forces}

%!TEX root = ../main.tex

\section{Symmetries and conservation laws}
\label{sec:standardmodel:symmetries}

Symmetries, global or local as well as continuous or discrete, structure the
Standard Model. According to Noether's theorem every symmetry corresponds to a
conservation law~\cite{Noether}.

As a whole the SM is constructed to be invariant under gauge transformations,
meaning that the physics is independent of the choice of the gauge. This
symmetry is only broken by the vacuum expectation value of the Higgs field,
which is the origin of the masses of the $W$ and $Z$ bosons. Gauge invariance
is associated with the conservation of charge, electrical and of colour. The
invariance under space-time translation corresponds to the conservation of
energy. However, Heisenberg's uncertainty principle~\cite{Heisenberg:1927zz}
allows the violation of the conservation of energy for a very short period of
time. This enables the existence of virtual, heavy particles in decay
processes. The invariance of a system under translation in space and rotation
leads to the conservation of momentum and angular momentum, respectively.
Moreover, the baryon number (B) and the lepton number (L) are conserved
respectively only broken by tiny non-pertubative effects. However, in the
early universe these effects might have been larger~\cite{Rubakov:1996vz},
while (B $-$ L) is an exact symmetry. The lepton family number, \ie the
individual lepton number for electrons, muons and tauons, also seems to be
conserved, at least no significant asymmetry has been found yet. But there is
no symmetry group evoking this conservation law and tests of the lepton
universality by LHCb using \mbox{$\Bu\!\to \Kp\ell^+\ell^-$} decays show an
asymmetry with a significance corresponding to \num{2.6} standard
deviations~\cite{LHCb-PAPER-2014-024}. While all the former symmetries are
absolute, \ie valid for all three interactions, there are also approximate
symmetries that only apply for the electromagnetic and strong but not for the
weak force. For example, flavour transitions are only possible in the weak
interaction. Thus, flavour symmetry as well as the U-spin symmetry are
approximate symmetries. The latter states that under the assumption that the
masses of up, down and strange quarks are the same, processes are invariant
under exchange of the two down-type quarks. This allows to transfer some
findings from one decay mode to another, \eg from decay modes of \Bd mesons to
\Bs mesons~\cite{Uspin1,Uspin2}. Another approximate symmetry is the parity
operation ($P$), which performs a spatial inversion of all coordinates
\begin{align}
	P\Psi(r) = \Psi(-r)\,,
\end{align}
or in other words it transforms left-handed into right-handed fermions. In
fact, the weak interaction even maximally violates parity, \eg there are only
left-handed neutrinos and right-handed antineutrinos. Charge conjugation ($C$)
is another discrete symmetry. It changes the sign of all charges and the
magnetic moment, and thus transforms particles into their antiparticles
\begin{align}
	C \ket{p} = \ket{\bar p}\,.
\end{align}
The combination of charge conjugation and parity (\CP) is more stable,
\eg a left-handed neutrino becomes a right-handed antineutrino. Nevertheless,
it is still violated at the \num{e-4} level by the weak interaction. \CP
violation is explained in more detail in \cref{sec:cpviolation}. Not before
combining \CP with time reversal ($T$), one of the most fundamental concepts
of the SM is found, the \CPT symmetry. The \CPT
theorem~\cite{Schwinger:1951xk,Luders:1954zz,pauli1955niels} states that
particles and antiparticles have the same mass and the same lifetime.
% OZI rule~\cite{Okubo,Zweig,Iizuka:1966fk}, named after Okubo, Zweig, Iizuka

%!TEX root = ../main.tex

\section{Problems and possible extensions}
\label{sec:standardmodel:extensions}

Although the SM has proven to be a very successful and predictive theory,
there are several issues that can not be explained in the SM and others that
appeal very constructed. The latter leads to the idea of a more fundamental
theory in which the SM is embedded. The concept of a unification has first
been proposed by Georgi and Glashow~\cite{Georgi:1974sy}. A first step would
be a generalization of the electroweak with the strong interaction. Then,
gravitation, the fourth fundamental force, could be included, whose effect is
almost negligible at the energy scale, which is probed in today's high energy
physic experiments, and therefore not part of the SM. Here, the difficulty is
that even nowadays gravitation is still based on Einstein's general theory of
relativity, so unlike the other theories of the SM not in a quantum mechanical
framework. The unification of the forces would probably emerge at energy
scales of \SI{e16}{\GeV}. However, quantum corrections from those mass scales
would heavily influence the Higgs mass, which is measured to be around
\SI{125}{\GeVcc}~\cite{HiggsMass}. In the SM this hierarchy problem is
solved by a fine tuning of tree-level and loop contributions, which exactly
cancel each other. Other explanations are given by extending the SM with new
symmetries, like models including
supersymmetry~\cite{Gervais:1971ji,Golfand:1971iw,Volkov:1972jx,*Volkov:1973ix}.
From a theoretical point of view it is also unsatisfactory that the SM
includes so many free parameters, like the masses of the constituents or the
number of generations. In addition, the SM only applies to the processes of
ordinary matter, which makes only about \SI{5}{\percent} of the total energy
density in the universe~\cite{Ade:2015xua}, whereas it lacks an explanation
for dark matter or dark energy. Furthermore, the amount of \CP violation in
the weak sector can not account for the baryon asymmetry in the universe, \ie
the dominance of matter without any large clusters of antimatter in the
universe. The observation of neutrino
oscillations~\cite{Fukuda:1998mi,Ahmad:2001an,*Ahmad:2002jz}, \ie the indirect
measurement of mass differences between the neutrino generations, shows that
neutrinos can not be massless as assumed in the SM. The special role of
neutrinos in the SM to appear only left-handed could be corrected if they were
Majorana particles, \ie their own antiparticles~\cite{Majorana:1937vz}
