%!TEX root = ../main.tex

\section{Backgrounds (2 pages)}
\label{sec:bd2jpsiks:backgrounds}

Although \BdToJPsiKS is an experimentally very clean decay channel care has to
be taken to properly identify, suppress or even reject, and parametrize
backgrounds. While the two muons can be identified quite effectively the pions
of the \KS decay might actually be kaons or protons which have been
mis-identified. This would lead to background contributions from \BdToJPsiKst
and \LbToJPsiL. To analyse the $\proton \to \pion$ mis-ID the proton mass
hypothesis can be assigned to one of the pions and the invariant mass of the
proton-pion pair $m_{\proton\pion}$ can be recalculated. A peak at the \Lz
mass $M_{\Lz} = \SI{1115.683}{\MeVcc}$~\cite{PDG2014} can be seen which is
reduced by applying a tighter requirement on the difference of the proton-pion
log-likelihood for candidates close to $M_{\Lz}$. With \LbToJPsiL signal MC it
is checked that after reconstruction, stripping and all offline selection
requirements, including the veto described above, the expected yield is a
sub-percent effect. The broad width of the \Kstarz does not allow an analogous
approach. Nevertheless, studies on \BdToJPsiKst MC show that the expected
contribution is even lower than for \LbToJPsiL. The main reason is the short
lifetime of the \Kstarz which is exploited by the lifetime significance cut on
the \KS.